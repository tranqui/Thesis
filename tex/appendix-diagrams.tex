%TC: macro \marginfootnote [other]
%TC: envir SCfigure [] other
%TC: macrocount beginSCfigure [figure]
\documentclass[11pt]{report}
\usepackage{preamble}
\setcounter{chapter}{3}
\renewcommand{\chaptername}{Appendix}
\renewcommand{\thechapter}{\Alph{chapter}}
\graphicspath{{../img/}}

\begin{document}
\chapter{Experimenting with diagrams}

We want to calculate
\begin{align}
  \nonumber
  \beta\Omega^{ex}_3 &=
  -\frac{1}{3!}
  \Bigg(
  \underbrace{
    3 \mayerdiagram[|.||.|][t...]{4} +
    3 \mayerdiagram[||||.|][t...]{4}
  }_{\textstyle = 3\mayerdiagram[|:||.|][t...]{4}} +
  \underbrace{
    3 \mayerdiagram[|.||||][t...]{4} +
    \mayerdiagram[||||||][t...]{4}
  }_{%
    \textstyle
    = 2\mayerdiagram[|.||||][t...]{4}
    - \mayerdiagram[|:||||][t...]{4}}
  - 2 \Delta V \mayerdiagram[|||][o..]{3}
  \Bigg) \\
  &=
  -\frac{1}{3!}
  \left(
  3 \mayerdiagram[|:||.|][t...]{4} +
  2 \mayerdiagram[|:.|||][t...]{4} -
  \mayerdiagram[|:||||][t...]{4}
  \right)
\end{align}
using $f = e - 1$ and the fact that
\begin{equation*}
  \begin{split}
    \mayerdiagram[|.||||][t...]{4}
    &=
    \mayerdiagram[|.:|||][t...]{4} -
    \mayerdiagram[|..|||][t...]{4} \\
    &=
    \mayerdiagram[|.:|||][t...]{4} +
    \Delta V \mayerdiagram[|||][o..]{3} \\
    &=
    \mayerdiagram[|:.|||][t...]{4} +
    \Delta V \mayerdiagram[|||][o..]{3}
  \end{split}
\end{equation*}
Here's my attempt:
\begin{align*}
  B_4 = -\frac{1}{4!}
  \\
  12 \left(
  \mayerdiagram[|..||..|.|][t....]{5} +
  \mayerdiagram[|..||.||.|][t....]{5}
  \right) +
  \\
  12 \left(
  \mayerdiagram[|||||..|.|][t....]{5} +
  \mayerdiagram[||.|||.|.|][t....]{5} +
  \mayerdiagram[||.||.||.|][t....]{5} +
  \mayerdiagram[||.|||||.|][t....]{5} +
  \mayerdiagram[||||||.|.|][t....]{5} +
  \mayerdiagram[||.||.||||][t....]{5}
  \right) +
  \\
  6 \left(
  \mayerdiagram[|..|||..||][t....]{5} +
  \mayerdiagram[|..||||.||][t....]{5} +
  \mayerdiagram[|..|||.|||][t....]{5} +
  \mayerdiagram[|..|||||||][t....]{5} +
  \mayerdiagram[||||||.|||][t....]{5}
  \right) +
  \\
  4 \left(
  \mayerdiagram[||.|.|.|.|][t....]{5} +
  \mayerdiagram[||||.|.|.|][t....]{5} +
  \mayerdiagram[||.|||||||][t....]{5}
  \right) +
  \\
  24 \left(
  \mayerdiagram[||.||..|.|][t....]{5} +
  \mayerdiagram[|..|||.|.|][t....]{5} +
  \mayerdiagram[|..||.||||][t....]{5} +
  \mayerdiagram[|.||||.|.|][t....]{5}
  \right) +
  \\
  3 \mayerdiagram[|||||.||.|][t....]{5} +
  \mayerdiagram[||||||||||][t....]{5}
\end{align*}
Organised by subgraph.
First, the reducible subgraphs:
\begin{subequations}
  \begin{align}
    \mayerdiagram[||||||]{4}: &
    6 \mayerdiagram[|..|||||||][t....]{5} +
    4 \mayerdiagram[||.|||||||][t....]{5} +
    \mayerdiagram[||||||||||][t....]{5}
    \nonumber \\ =&
    \mayerdiagram[|..:||||||][t....]{5} + 
    \mayerdiagram[|.::||||||][t....]{5} +
    \mayerdiagram[|:::||||||][t....]{5} -
    3 \mayerdiagram[|...||||||][t....]{5}
    \nonumber \\ =&
    \mayerdiagram[|..:||||||][t....]{5} + 
    \mayerdiagram[|.::||||||][t....]{5} +
    \mayerdiagram[|:::||||||][t....]{5} +
    3 \Delta V \mayerdiagram[||||||][o...]{4}
    \\
    \mayerdiagram[|.||.|]{4}: &
    12 \mayerdiagram[|..||.||.|][t....]{5} +
    \underbrace{6 \mayerdiagram[|..|||..||][t....]{5}}_{%
      \textstyle
      = 6 \mayerdiagram[|.|.|.||.|][t....]{5}
    } +
    12 \mayerdiagram[||.||.||.|][t....]{5} +
     3 \mayerdiagram[|||||.||.|][t....]{5}
    \nonumber \\ =&
    12 \mayerdiagram[|..:|.||.|][t....]{5} -
    12 \mayerdiagram[|...|.||.|][t....]{5} +
     6 \mayerdiagram[|.:.|.||.|][t....]{5} -
     6 \mayerdiagram[|...|.||.|][t....]{5} +
     9 \mayerdiagram[||.||.||.|][t....]{5} +
     3 \mayerdiagram[||:||.||.|][t....]{5}
    \nonumber \\ =&
    12 \mayerdiagram[|..:|.||.|][t....]{5} -
    18 \mayerdiagram[|...|.||.|][t....]{5} +
     6 \mayerdiagram[|.:.|.||.|][t....]{5} +
     9 \mayerdiagram[||.||.||.|][t....]{5} +
     3 \mayerdiagram[||:||.||.|][t....]{5}
    \nonumber \\ =&
    12 \mayerdiagram[|..:|.||.|][t....]{5} -
    18 \mayerdiagram[|...|.||.|][t....]{5} +
    6 \mayerdiagram[|.:.|.||.|][t....]{5} +
    9 \left(
    \mayerdiagram[|:.:|.||.|][t....]{5} -
    2 \mayerdiagram[|:..|.||.|][t....]{5} +
    \mayerdiagram[|...|.||.|][t....]{5}
    \right) +
    3 \left(
    \mayerdiagram[|:::|.||.|][t....]{5} -
    2 \mayerdiagram[|.::|.||.|][t....]{5} +
    \mayerdiagram[|.:.|.||.|][t....]{5}
    \right)
    \nonumber \\ =&
    3 \left(
    -2 \mayerdiagram[|..:|.||.|][t....]{5} -
    3 \mayerdiagram[|...|.||.|][t....]{5} +
    3 \mayerdiagram[|.:.|.||.|][t....]{5} +
    3 \mayerdiagram[|:.:|.||.|][t....]{5} +
    \mayerdiagram[|:::|.||.|][t....]{5} -
    2 \mayerdiagram[|.::|.||.|][t....]{5}
    \right)
    \nonumber \\ =&
    3 \left(
    \mayerdiagram[|:::|.||.|][t....]{5} -
    2 \mayerdiagram[|.::|.||.|][t....]{5} +
    3 \mayerdiagram[|:.:|.||.|][t....]{5} -
    2 \mayerdiagram[|..:|.||.|][t....]{5} +
    3 \mayerdiagram[|.:.|.||.|][t....]{5} +
    3 \Delta V \mayerdiagram[|.||.|][o...]{4}
    \right)
    \\
    \mayerdiagram[|.||||]{4}: &
    24 \mayerdiagram[|..||.||||][t....]{5} +
    12 \mayerdiagram[||.|||||.|][t....]{5} +
    12 \mayerdiagram[||.||.||||][t....]{5} +
    6 \mayerdiagram[|..||||.||][t....]{5} +
    6 \mayerdiagram[|..|||.|||][t....]{5}
  \end{align}
\end{subequations}
And the reducible subgraphs:
\begin{subequations*}
  \begin{align}
    \mayerdiagram[|..|.|]{4}: &
    12 \left(
    2 \mayerdiagram[||.||..|.|][t....]{5} +
    \mayerdiagram[|..||..|.|][t....]{5} +
    \mayerdiagram[|||||..|.|][t....]{5}
    \right)
    \nonumber \\ =&
    12 \Bigg(
    2 \mayerdiagram[|:.:|..|.|][t....]{5} -
    2 \mayerdiagram[|:..|..|.|][t....]{5} -
    2 \mayerdiagram[|..:|..|.|][t....]{5} +
    2 \mayerdiagram[|...|..|.|][t....]{5} +
    \nonumber \\ &
    \mayerdiagram[|..:|..|.|][t....]{5} -
    \mayerdiagram[|...|..|.|][t....]{5} +
    \mayerdiagram[|:::|..|.|][t....]{5} -
    \mayerdiagram[|::.|..|.|][t....]{5} -
   2\mayerdiagram[|:.:|..|.|][t....]{5} +
   2\mayerdiagram[|:..|..|.|][t....]{5} +
    \mayerdiagram[|..:|..|.|][t....]{5} -
    \mayerdiagram[|...|..|.|][t....]{5}
    \Bigg)
    \nonumber \\ =&
    12 \Bigg(
    2 \mayerdiagram[|:.:|..|.|][t....]{5} -
    2 \mayerdiagram[|:..|..|.|][t....]{5} -
    2 \mayerdiagram[|..:|..|.|][t....]{5} +
    2 \mayerdiagram[|...|..|.|][t....]{5} +
    \nonumber \\ &
    2 \mayerdiagram[|..:|..|.|][t....]{5} -
    2 \mayerdiagram[|...|..|.|][t....]{5} -
    2 \mayerdiagram[|:.:|..|.|][t....]{5} +
    2 \mayerdiagram[|:..|..|.|][t....]{5} +
    \mayerdiagram[|:::|..|.|][t....]{5} -
    \mayerdiagram[|::.|..|.|][t....]{5}
    \Bigg)
    \nonumber \\ =&
    12 \left(
    \mayerdiagram[|:::|..|.|][t....]{5} -
    \mayerdiagram[|::.|..|.|][t....]{5}
    \right)
    \\
    \mayerdiagram[.|.|.|]{4}: &
    4 \left(
    \mayerdiagram[||.|.|.|.|][t....]{5} +
    \mayerdiagram[||||.|.|.|][t....]{5}
    \right) =
    4 \mayerdiagram[||:|.|.|.|][t....]{5}
    \\
    \mayerdiagram[|..|||]{4}: &
    6 \left(
    4 \mayerdiagram[|..|||.|.|][t....]{5} +
    4 \mayerdiagram[|.||||.|.|][t....]{5} +
    2 \mayerdiagram[||.|||.|.|][t....]{5} +
    2 \mayerdiagram[||||||.|.|][t....]{5} +
    \mayerdiagram[||||||.|||][t....]{5}
    \right)
  \end{align}
\end{subequations*}
Reminder: the subgraphs contributing to $B_4$ are
\begin{equation}
  B_4 = -\frac{3}{4!}
  \left(
  3 \mayerdiagram[|.||.|][o...]{4} +
  6 \mayerdiagram[||||.|][o...]{4} +
  \mayerdiagram[||||||][o...]{4}
  \right)
\end{equation}
%% Next we calculate
%% \begin{align}
%%   \nonumber
%%   \beta\Omega^{ex}_4 &=
%%   -\frac{1}{4!}
%%   \Bigg(
%%     3 \mayerdiagram[|.||.|][t...]{4} +
%%     3 \mayerdiagram[||||.|][t...]{4}
%%     3 \mayerdiagram[|.||||][t...]{4} +
%%     \mayerdiagram[||||||][t...]{4}
%%   \Bigg) \\
%%   &=
%%   0
%% \end{align}

\end{document}
