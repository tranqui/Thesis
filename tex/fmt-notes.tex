\subsection{Fundamental measure theory (FMT)}

\subsection{Many-body correlations from fundamental measure theory}

\begin{itemize}
\item $d+1$ weight functions
\item Exact decomposition of Mayer-f bond
\end{itemize}

We have 4 scalar weight functions
\begin{subequations}
  \begin{align}
    \omega_3^i(\vec{r}) &= \Theta(R_i - r), \\
    \omega_2^i(\vec{r}) &= \delta(R_i - r), \\
    \omega_1^i(\vec{r}) &= \frac{\omega_2^i(\vec{r})}{4\pi R_i}, \\
    \omega_0^i(\vec{r}) &= \frac{\omega_2^i(\vec{r})}{4\pi R_i^2},
  \end{align}
\end{subequations}
and 2 vector weight functions
\begin{subequations}
  \begin{align}
    \vec{\omega}_2^i(\vec{r}) &=
    \frac{\vec{r}}{r} \delta(R_i - r), \\
    \vec{\omega}_1^i(\vec{r}) &=
    \frac{\vec{\omega}_1^i(\vec{r})}{4\pi R_i}.
  \end{align}
\end{subequations}

Fourier transforming weight functions:
\begin{equation}
  \widetilde{\omega}_\alpha(\vec{k}) =
  \int \omega_\alpha (\vec{r}) e^{-i \vec{k}\cdot\vec{r}} \, d\vec{r}
\end{equation}
%% \begin{align}
%%   \widetilde{\omega}_3(\vec{k}) &=
%%   4\pi \frac{\sin{kR} - kR\cos{kR}}{k^3} \\
%%   \widetilde{\omega}_2(\vec{k}) &=
%%   4\pi R^2 \frac{\sin{kR}}{kR} \\
%%   \widetilde{\vec{\omega}}_2(\vec{k}) &=
%%   - i \vec{k} \widetilde{\omega}_3(\vec{k})
%% \end{align}
\begin{equation}
  \begin{aligned}
    c_{22}(\vec{r} = \vec{r}_2 - \vec{r}_1) &=
    \frac{1}{(2\pi)^3}
    \iint
    e^{-i (\vec{k}_1\cdot\vec{r}_1 + \vec{k}_2\cdot\vec{r}_2)}
    \widetilde{\omega}_2(\vec{k}_1)
    \widetilde{\omega}_2(\vec{k}_2)
    \delta{(\vec{k}_1 + \vec{k}_2)}
    \, d\vec{k}_1 d\vec{k}_2 \\
    &=
    \frac{1}{(2\pi)^3}
    \int
    e^{i \vec{k}_1 \cdot (\vec{r}_2 - \vec{r}_1)}
    \widetilde{\omega}_2(\vec{k}_1)
    \widetilde{\omega}_2(-\vec{k}_1)
    \, d\vec{k}_1 \\
    &=
    \frac{1}{(2\pi)^3}
    \int
    e^{i \vec{k} \cdot \vec{r}}
    \widetilde{\omega}_2(\vec{k})^2
    \, d\vec{k}
    =
    \frac{8\pi^2 R^2}{(2\pi)^3}
    \int
    e^{i \vec{k} \cdot \vec{r}}
    \left( \frac{\sin{kR}}{k} \right)^2
    \, d\vec{k} \\
    &=
    \frac{R^2}{\pi}
    \int
    e^{i k r \cos\theta}
    \left( \frac{\sin{kR}}{k} \right)^2
    \, 2\pi k^2 \sin\theta \, dr d\theta \\
    &=
    2 R^2
    \int
    e^{i k r \cos\theta}
    \sin^2{(kR)}
    \, \sin\theta \, dr d\theta
  \end{aligned}
\end{equation}

Following \cite{Rosenfeld1990} we have
\begin{equation}\label{eq:fmt-direct-correlations}
  \begin{aligned}
    c^{(n)}(\vec{r}^n) &=
    - \frac{\delta^n \beta F_{ex}}{\delta \rho(\vec{r}_1)\delta \rho(\vec{r}_2) \cdots \delta \rho(\vec{r}_n)} \\
    &=
    - \sum_{\alpha_1, \alpha_2, \cdots, \alpha_n}
    \int d\vec{r}'
    \prod_{i=1}^n \Big( \omega_{\alpha_i}(\vec{r}' - \vec{r}_i) \Big)
    \partial^n_{\alpha_1, \alpha_2, \cdots, \alpha_n} \beta\Phi_{ex}(\vec{r}')
  \end{aligned}
\end{equation}
where
\begin{equation*}
  \partial^n_{\alpha_1, \alpha_2, \cdots, \alpha_n} \beta\Phi_{ex}(\vec{r}') =
  \left.
  \frac{\partial^n \beta\Phi_{ex}}{\partial n_{\alpha_1} \partial n_{\alpha_2} \cdots \partial n_{\alpha_n}}
  \right|_{\{n_\alpha\} = \{n_\alpha(\vec{r}')\}}.
\end{equation*}
At uniform density $\partial^n_{\alpha_1, \alpha_2, \cdots, \alpha_n} \beta\Phi_{ex}$ is position independent, so \eqref{eq:fmt-direct-correlations} becomes
\begin{equation}\label{eq:fmt-direct-correlations-uniform-density}
  \begin{aligned}
    c^{(n)}(\vec{r}^n) &=
    - \sum_{\alpha_1, \alpha_2, \cdots, \alpha_n}
    \partial^n_{\alpha_1, \alpha_2, \cdots, \alpha_n} \beta\Phi_{ex}
    \int d\vec{r}'
    \prod_{i=1}^n \Big( \omega_{\alpha_i}(\vec{r}' - \vec{r}_i) \Big) \\
    &=
    - \sum_{\alpha_1, \alpha_2, \cdots, \alpha_n}
    \partial^n_{\alpha_1, \alpha_2, \cdots, \alpha_n} \beta\Phi_{ex} \;
    \Big(
    \omega_{\alpha_1} \otimes \omega_{\alpha_2} \otimes \cdots \otimes \omega_{\alpha_n}
    (\vec{r}^n)
    \Big)
  \end{aligned}
\end{equation}
where the $\otimes$-notation in the latter step denotes the $n$-body convolution.
For example, the two body convolution would be written%
\todo{This is not the standard convolution! See \cite{Rosenfeld1997}.}
%\marginfootnote{The `standard' convolution, i.e.\ $f \otimes g(\vec{r} \equiv \vec{r}_1 - \vec{r}_2)$, is recovered after transforming to the new integration variable $\vec{r}'' = \vec{r}' - \vec{r}_1$.}
\begin{equation*}
  f \otimes g(\vec{r}_1, \vec{r}_2) =
  \int d\vec{r}' f(\vec{r}' - \vec{r}_1) g(\vec{r}' - \vec{r}_2).
\end{equation*}
\todo{Also, the standard convolution symbol is an asterisk, not an outer product.}
Applying the convolution theorem allows the Fourier transform of \eqref{eq:fmt-direct-correlations-uniform-density} to be written rather succinctly as
\begin{equation}
  \tilde{c}^{(n)}(\vec{k}^n) =
  - \sum_{\alpha_1, \alpha_2, \cdots, \alpha_n}
  \partial^n_{\alpha_1, \alpha_2, \cdots, \alpha_n} \beta\Phi_{ex} \;
  \left( \prod_{i=1}^n \widetilde{\omega}_{\alpha_i}(\vec{k}_i) \right)
  \delta(\vec{k}_1 + \vec{k}_2 + \cdots + \vec{k}_n).
\end{equation}
The delta function enforces the `ring' condition $\sum_{i=1}^n \vec{k}_i = 0$ which emerges from translational symmetry of the weight functions%
\marginfootnote{In the previous note this occurred by change of integration variables to a relative displacement, which was only possible because of this translational symmetry.}[-3cm],
reducing the dimensionality of the domain by $d$.
A further $d(d-1)/2$ degrees of freedom%
\marginfootnote{This many degrees of freedom can be removed for general $n \ge d$, but there are fewer for $n < d$.
  For example, $n=2$ arrangements (a dimer) are isomorphic to a line so they possess $d-1$ rotational degrees of freedom.}
can be removed by exploiting rotational symmetry.

From \cite{Rosenfeld1990}:
\begin{align}
  \widetilde{\omega}_0 &= \cos{(kR)} \\
  \widetilde{\omega}_1 &= 2\frac{\sin{(kR)}}{k} \\
  \widetilde{\omega}_0 &= \cos{(kR)}
\end{align}

White bear I:
\begin{equation}
  c^{(2)}(r)
\end{equation}

The Ornstein-Zernike equation
\begin{equation}
  h^{(2)}(\vec{r}) =
  c^{(2)}(\vec{r}) +
  \rho \int d\vec{r}' h^{(2)}(\vec{r}') c^{(2)}(\vec{r} - \vec{r}')
\end{equation}
in Fourier space
\begin{equation*}
  \tilde{h}^{(2)}(\vec{k}) =
  \tilde{c}^{(2)}(\vec{k}) +
  \rho \tilde{h}^{(2)}(\vec{k}) \tilde{c}^{(2)}(\vec{k})
\end{equation*}
after rearranging
\begin{equation}
  \tilde{h}^{(2)}(\vec{k}) =
  \frac{\tilde{c}^{(2)}(\vec{k})}{1 - \rho \tilde{c}^{(2)}(\vec{k})}
\end{equation}

The two-point static structure factor
\begin{equation}
  \begin{aligned}
    S^{(2)}(\vec{k}) &\equiv 1 + \rho \tilde{g}^{(2)}(\vec{k}) \\
    &= 1 + \rho \delta(\vec{k}) + \rho \tilde{h}^{(2)}(\vec{k})
  \end{aligned}
\end{equation}

\begin{SCfigure}[H]
  \missingfigure[figwidth=\linewidth]{}
  \caption{Static structure factor for convolution, Rosenfeld and White Bear closures.}
\end{SCfigure}

\begin{SCfigure}[H]
  \missingfigure[figwidth=\linewidth]{}
  \caption{Triplet static structure factors for convolution, Rosenfeld and White Bear closures.}
\end{SCfigure}

\section{Superposition and convolution approximations}

In the Kirkwood superposition approximation \cite{Kirkwood1935} many-body correlations are expressed as pairwise products of the two-body correlation function, i.e.
\begin{equation}
  g^{(n)}(\vec{r}^n) =
  \prod_{i < j} g^{(2)}(\vec{r}_i, \vec{r}_j),
\end{equation}
which correctly satisfies the hard-core condition, but violates the sum rule
\begin{equation}
  \begin{aligned}
    \rho^{(n)}(\vec{r}^n) &=
    \frac{1}{\Xi} \sum_{N=n}^\infty \frac{z^N}{(N-n)!} \int e^{-\beta U_N} \, d\vec{r}^{(N-n)} \\
    &=
    \frac{1}{N-n}
    \int d\vec{r}_n \left(
    \frac{1}{\Xi} \sum_{N=n}^\infty \frac{z^N}{(N-(n+1))!} \int e^{-\beta U_N} \, d\vec{r}^{(N-(n+1))}
    \right) \\
    &=
    \frac{1}{N-n}
    \int \rho^{(n+1)}(\{\vec{r}^n, \vec{r}_{n+1}\}) d\vec{r}_{n+1},
  \end{aligned}
\end{equation}
and the related convolution approximation \cite{Jackson1962,Ichimaru1970,Barrat1988}%
\todo{Check this expression is correct - it almost certainly is not.}
\begin{equation}
  S^{(n)}(\vec{k}^n) =
  (1 + \tilde{c}^{(n)}(\vec{k}^n))
  \prod_{i < j} S^{(2)}(\vec{k}_i, \vec{k}_j)
\end{equation}
satisfies the sum rule but fails to satisfy the hard-core condition.

%% In equilibrium
%% \begin{equation}
%%   c^{(n)}(\vec{r}^n) =
%%   \left.
%%   \frac{\delta^n \beta F_{ex}}{\delta \rho(\vec{r}_1)\delta \rho(\vec{r}_2) \cdots \delta \rho(\vec{r}_n)}
%%   \right|_{\rho(\vec{r})=\rho}
%% \end{equation}
