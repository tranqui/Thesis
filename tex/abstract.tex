\chapter*{Abstract}
\addcontentsline{toc}{chapter}{Abstract}

This dissertation describes an investigation into methods for advancing understanding of liquids at high densities, where dynamical processes become highly non-trivial.
Specifically, we address structure in supercooled liquids approaching their glass transition, and the kinetics of nucleating the stable cystal phase.
In both cases we describe the liquid using equilibrium physics, even though the system is metastable and not strictly in equilibrium.
The first three results chapters focus on the supercooled liquid, while the final results chapter addresses nucleation.

In the first part comprising three chapters, we combine geometric techniques with liquid state theory to develop an approach for treating complex many-particle local structures inside the bulk hard sphere liquid.
We introduce the \emph{morphometric approach}, a liquid state theory based on integral geometry, as a means of calculating many-body correlation functions.
We argue for the morphometric approach from several routes, and derive multiple specific theories for hard spheres including one suitable for producing accurate correlation functions.
We later derive the morphometric approach for hard spheres from first-principles using the virial series.
This places the approach on more rigorous ground, and suggests routes to extending the theory as part of a controlled expansion.

With the resulting many-body correlation functions, we are able to predict the concentrations of complex many-particle structures in the bulk liquid; these results are of particular relevance to theories of supercooled liquids and glasses.
We find a bimodality in the energy landscape for hard sphere local structures, where fivefold symmetric structures appear lower in free energy than fourfold symmetric structures.
In addition, we develop similar techniques for predicting the thermodynamic barriers to dynamical processes inside the bulk system.
The solution to the overarching problem of predicting structure formation inside a bulk system has potential to advance study of self-assembly, nucleation and protein folding in aqueous environments.

In a final part we address nucleation of salt crystals in drying aerosol droplets, of particular relevance to climate models and industrial spray drying applications.
Treating the droplet in the continuum limit, we solve the diffusion equation with moving boundary conditions.
By comparison with experimental data we are able to assess the accuracy of classical nucleation theory, with mixed success depending on the system.
%% We find it successfully predicts the nucleation kinetics of \ce{NaCl} aerosol droplets but fails for \ce{NaNO3}.
%% We find a complex interplay between inhomogeneity of the droplet concentration profile and droplet temperature affects the nucleation kinetics.
