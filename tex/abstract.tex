\chapter*{Abstract}
\addcontentsline{toc}{chapter}{Abstract}

This dissertation describes an investigation into methods for advancing understanding of liquids at high densities, where dynamical processes become highly non-trivial.
Specifically, we address structure in supercooled liquids approaching their glass transition, and the kinetics of nucleating the stable cystal phase.
In both cases we describe the liquid using equilibrium physics, even though the system is metastable and not strictly in equilibrium.
The first three results chapters focus on the supercooled liquid, while the final results chapter addresses nucleation.

In the first part we combine geometric techniques with liquid state theory to develop an approach for treating local structure inside the bulk hard sphere liquid.
We describe methods for calculating many-body correlation functions to do this
We are able to predict the concentrations of complex many-particle structures in the bulk liquid, of relevance to theories of supercooled liquids and glasses.
In addition, the problem we solve has potential to advance study of self-assembly and protein folding in aqueous environments.

In the final part we address nucleation of salt crystals in drying aerosol droplets, of particular relevance to climate models.
Treating the droplet in the continuum limit we solve the diffusion equation with moving boundary conditions.
By comparison with experimental data we are able to assess the accuracy of classical nucleation theory in this system.
We find it successly predicts the nucleation kinetics of \ce{NaCl} aerosol droplets but fails for \ce{NaNO3}.
We find a complex interplay between inhomogeneity of the droplet concentration profile and droplet temperature affects the nucleation kinetics.
