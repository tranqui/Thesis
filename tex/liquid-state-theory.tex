
Here we talk in general terms about descriptions of the liquid state.
Broadly speaking, in its historical development approaches can be placed inside one of two categories.
Namely, theories involving
\begin{enumerate}
  \item Local geometric approximations capturing the short range interactions (free volume theory/cell theory, scaled particle theory), and
  \item Integral equations (Ornstein-Zernike closures, density functional theory) which properly treat the long range correlations.
\end{enumerate}
These two approaches are not mutually exclusive, and hybrid theories can improve on.
For instance, fundamental measure theory (FMT) involves the synthesis of integral geometry with the formalism of density functional theory which involves minimising a functional (i.e.\ an integral equation).

Classical theory of phase transitions (Landau).
Van der waals theory.
How does the transition occur?
Metastability leads into kinetics.

Kinetics vs thermodynamics.
Thermodynamic driving force vs activation barrier.

Relaxation behaviour controlled by activation barrier.
A thermal fluctuation which takes the system over the barrier%
\marginfootnote{These fluctuations are conventionally called \emph{instantons} as they spontaneously appear and vanish just like virtual particles in fundamental physics.
  The name for this relatively straightforward phenomenon is thus a reference to a much more counterintuitive and bizarre phenomenon, because physicists are good at making helpful analogies.}
occurs with rate $\exp{-\beta \Delta U}$ \cite{Langer}.

Liquids:
Free volume theory.
Early curvature corrections.
Free volume theory depends on the free volume (duh).
This is an example of an integral geometric theory.
Morphological 

Technical - handwavey:
Motion invariance: translation and rotation invariance.
Additivity: the energy is extensive (even for small system sizes)
Continuity: the function is well behaved

\section{Liquid state theory}
A summary of results in the modern formalism as given by standard texts (notably references \cite{Hansen2013} and \cite{Santos2016}) that will be used throughout the thesis.

\section{Statistical mechanical preliminaries}

I will assume the reader has a background in physics, and is familiar enough with statistical ensembles that I do not need to explain the origin of the following equations in detail.
Briefly, these emerge by considering typical fluctuations of thermodynamic quantities for a subsystem within a macroscopic system called the \emph{ensemble}; the properties of this larger system define average quanties of the subsystem \cite{standard-text}.
Alternatively, from a Bayesian perspective the probability distributions given below emerge from maximising the entropy%
\marginfootnote{The entropy represents a thermodynamic quantity in the former picture, whereas it represents our own \emph{uncertainty} about the system in the latter.
  While these two approaches are formally equivalent, the first interpretation is more common in the physics literature.
  That being said, the derivations of the ensembles within an information theoretic framework are remarkably simple and will probably be more accessible to the non-physicist with a background in probability theory.}[-3cm]
subject to the constraint of average energy and (optionally) the average particle number \cite{Jaynes1957a,Jaynes1957b}.

A $d$-dimensional system of $N$ particles consists of $\vec{r}^N = \{\vec{r}_1, \cdots, \vec{r}_N\} \in \mathbb{R}^{dN}$ coordinates and $\vec{p}^N = \{\vec{p}_1, \cdots, \vec{p}_N\} \in \mathbb{R}^{dN}$ momenta.
The classical Hamiltonian can be decomposed into kinetic and potential terms as in
\begin{equation}
  \mathcal{H}(\vec{r}^N, \vec{p}^N) = K_N(\vec{p}^N) + U_N(\vec{r}^N)
\end{equation}
in the absence of an external field.
Further, we constrain the coordinates inside the region $\mathbb{V}^d$ of volume%
\marginfootnote{Physicists would typically use the same symbol $V$ for these two concepts.
  In keeping with the formal language of the previous chapter we would write $\vec{r}_i \in \mathbb{V}^d$ with $V = \mu_d(\mathbb{V}^d)$.}
$V$.
The \emph{canonical} ensemble describes an equilibrium system at constant temperature $T$ with probability measure
\begin{equation}
  f^{(N)}(\vec{r}^N, \vec{p}^N) \propto e^{-\beta{\mathcal{H}(\vec{r}^N, \vec{p}^N)}}
\end{equation}
where $\beta = (k_B T)^{-1}$ with Boltzmann constant $k_B$.
The proportionality constant ensures the probability distribution is properly normalised, leading to the canonical partition function
\begin{equation}
  Q_N =
  \int_{\mathbb{R}^{dN}} \int_{\mathbb{V}^{dN}}
  e^{-\beta{\mathcal{H}(\vec{r}^N, \vec{p}^N)}}
  \, d\vec{r}^N d\vec{p}^N.
\end{equation}
Or the classical kinetic energy can be written
\[ K_N(\vec{p}^N) = \sum_{i=1}^N \frac{|\vec{p}|^2}{2m_i} \]
then the momenta can be integrated out leaving only the configurational integral
\begin{equation}
  Q_N = \frac{\Lambda^{dN} Z_N}{N!}
\end{equation}
with
\begin{equation*}
  Z_N =
  \int_{\mathbb{V}^{dN}}
  \exp{-\beta U_N}
  \, d\vec{r}^N.
\end{equation*}
%% and confiprobability measure
%% \begin{equation}
%%   g^{(N)}(\vec{r}^N) = \frac{e^{-\beta U_N(\vec{r}^N)}}{Z_N}
%% \end{equation}

The grand canonical ensemble is taken.

The phase space measure is now
The partition function is given as
\begin{equation}
  \Xi
  = \sum_{N=0}^\infty \frac{z^N Z_N}{N!} \, d\vec{r}^N
  = \sum_{N=0}^\infty \frac{z^N}{N!} \int e^{-\beta U_N} \, d\vec{r}^N,
\end{equation}
where the activity is written in terms of the thermal de Broglie wavelength $\Lambda$ as $z = \exp{\beta\mu} / \Lambda^d$.
Accordingly, average quantities are found via
\begin{equation}
  \left< \cdots \right> =
  \frac{1}{\Xi} \sum_{N=0}^\infty \frac{z^N}{N!} \int \left(\cdots\right) e^{-\beta U_N} \, d\vec{r}^N,
\end{equation}

We will work almost exlusively in the \emph{grand canonical ensemble} as it is the most convenient for liquid state descriptions%
\marginfootnote{Notably the free energy is extensive without the need to invoke Stirling's approximation for $N!$, making the thermodynamics properly self-consistent even at finite system volumes.}.

\begin{equation}
  \mu_{id} = k_B T \ln{\Lambda^d \rho}
\end{equation}
\begin{equation}
  \frac{F_{id}}{N} = k_B T (\ln{\Lambda^d \rho} - 1)
\end{equation}

\subsection{Interaction potentials}
\subsection{Simple liquids}
\subsection{Hard sphere model system}

\subsubsection{Grand canonical ensemble}
\paragraph{Averages}
The partition function and free energy
\paragraph{Correlations}
Particle densities $\rho^{(n)}$ and distribution functions $g^{(n)}$.

\subsection{Homogeneous liquid state theory}

\subsubsection{Virial expansion of the equation of state}

\paragraph{Distribution function form}
Pressure in terms of pair distribution function $g(r)$.

For pair potentials:
\begin{equation}
  \frac{\beta P}{\rho} =
  1 - \frac{2 \pi \beta \rho}{3}
  \int_0^\infty v'(r) g(r) r^3 \, dr
\end{equation}
Generalisation to many-body potentials requires higher order distribution functions.
Difficulty of discontinuities overcome by introducing cavity function
\begin{equation}
  y(r) = e^{\beta v(r)} g(r)
\end{equation}
which is continuous \todo{Find out why this is continuous} leading to
\begin{equation}
  \frac{\beta P}{\rho} =
  1 + \frac{2 \pi \beta \rho}{3}
  \int_0^\infty e'(r) g(r) r^3 \, dr,
\end{equation}
where Boltzmann factor of pair potential is
\begin{equation}
  e(r) = e^{\beta v(r)}.
\end{equation}

\paragraph{Diagrammatic form}
Low density expansion of pressure in series with virial coefficients.

\begin{equation}
  \frac{\beta P}{\rho} =
  1 + \sum_{i=2}^\infty B_i(T) \rho^{i-1}
\end{equation}
where $B_i$ are the virial coefficients.

Generalisation to a binary mixture \todo{Where does this come from? What do the diagrams look like? Are they simpler than the distribution function diagrams?}: \cite{Hansen-Goos2014}
\begin{equation}
  \Phi = \sum_{n=2}^\infty \sum_{j=0}^{n}
  \frac{1}{n-1} {n \choose j} B_n^{[j]} \rho_1^{n-j} \rho_2^j
\end{equation}

\paragraph{Empirical Carnahan-Starling equation of state for hard spheres}

The excess free energy is determined from the equation of state by
\begin{equation}
  \frac{\beta F^{ex}}{N}
  = \int_0^\eta \left( \frac{\beta p}{\rho} - 1 \right) \, \frac{d\eta'}{\eta'},
\end{equation}
giving the excess chemical potential from the thermodynamic relation
\begin{equation}\label{eq:chemical-potential}
  \beta \mu^{ex}[p]
  = \beta \left( \frac{\partial F^{ex}}{\partial N} \right)_{V,T}
  = \left( \frac{\beta p}{\rho} - 1 \right)
  + \int_0^\eta \left( \frac{\beta p}{\rho} - 1 \right) \, \frac{d\eta'}{\eta'}.
\end{equation}

The Carnahan-Starling equation of state approximates the pressure for hard spheres as \cite{Carnahan1969}
\begin{equation}\label{eq:cs-pressure}
  \frac{\beta p_{cs}}{\rho} = \frac{1 + \eta + \eta^2 - \eta^3}{(1-\eta)^3},
\end{equation}
which gives the excess chemical potential using \eqref{eq:chemical-potential} as
\begin{equation}\label{eq:cs-mu}
  \beta \mu_{cs}^{ex} = \frac{8\eta - 9\eta^2 + 3\eta^3}{(1-\eta)^3}.
\end{equation}

\subsubsection{Free energy from distribution functions}
\paragraph{Distribution function theories}
\paragraph{Kirkwood superposition approximation}
\paragraph{Distribution functions from direct correlation functions: Ornstein-Zernike equation}

\subsubsection{Beyond hard spheres: perturbation theory and the mean field approximation}

\subsection{Does integral geometry help us?}

Simple example: Mayer-f two balls:
\begin{equation}
  \frac{\beta p}{\rho} - 1 =
  - \sum_{i=1}^\infty \frac{i}{i+1} \beta_i \rho^i =
  \frac{\rho}{2} \int_{\mathbb{E}_d} \mu_0 (B_d \cap g B_d) \, dg
  + \mathcal{O}(\rho^2)
\end{equation}
Volume fraction $\eta = \rho \omega_d$
\begin{equation}
  \frac{\beta p}{\rho} - 1 =
  \frac{\eta}{2 \omega_d}
  \int_{\mathbb{E}_d} \mu_0 (B_d \cap g B_d) \, dg
  + \mathcal{O}(\eta^2)
\end{equation}
For $d = 3$ this gives $4\eta + \mathcal{O}(\eta^2)$.
\begin{align*}
  \frac{\eta}{2 \omega_d}
  \int_{\mathbb{E}_d} \mu_0 (B_d \cap g B_d) \, dg &=
  \sum_{i=0}^{d}
  {d \brack i}^{-1}
  \mu_i(B_d) \mu_{d-i}(B_d) \\
  &=
  \frac{\eta}{2 \omega_d}
  \sum_{i=0}^{d}
  {d \brack i}^{-1}
  {d \brack i} \omega_i
  {d \brack d-i} \omega_{d-i} \\
  &=
  \frac{\eta}{2 \omega_d}
  \sum_{i=0}^{d}
  {d \brack i} \omega_i \omega_{d-i} \\
  &=
  \begin{cases}
    \frac{\eta}{2 \omega_2}
    (2 \omega_0 \omega_2 + \frac{\pi}{2} \omega_1^2)
    & \qquad d=2 \\
    \frac{\eta}{\omega_3}
    (\omega_0 \omega_3 + 2 \omega_1 \omega_2)
    & \qquad d=3
  \end{cases} \\
  &=
  \begin{cases}
    2 \eta & \qquad d=2 \\
    4 \eta & \qquad d=3
  \end{cases}
\end{align*}
using \eqref{eq:flag-coefficients-symmetry} and \eqref{eq:intrinsic-volume-ball-flag}.
\begin{equation*}
\end{equation*}

We can do much better in $d=1$, for hard rods.
We have the pressure defined by
\begin{equation}
  \frac{\beta p}{\rho} =
  1 + \sum_{n=2}^\infty c_n (n-1) \rho^{n-1} \Gamma_n
\end{equation}
with $c_n = 1/n(n-1)$ \todo{I think you can get $c_n$ from the $d=0$ case: check this} giving
\begin{equation}
  \Gamma_n = \int_{\mathbb{E}_d^n} \mu_0 (B^d \cap_{i=2}^n g_i B^d) \, d^n g
\end{equation}
For $d=1$ we have
\begin{equation}
  \int_{\mathbb{E}} \mu_k (A \cap g B^1) \, dg =
  \begin{cases}
    \mu_0(A) \mu_1(B^1) + \mu_1(A) & \qquad k=0 \\
    \mu_1(A) \mu_1(B^1) & \qquad k=1
  \end{cases}
\end{equation}
which gives
\begin{equation}
  \begin{split}
    \Gamma_n &= \int_{\mathbb{E}_d^{n-1}} \left(
    \mu_0 (B^1 \cap_{i=2}^{n-1} g_i B^1) \mu_1(B^1) +
    \mu_1 (B^1 \cap_{i=2}^{n-1} g_i B^1)
    \right) \, d^{n-1} g \\
    &= \Gamma_{n-1} \mu_1(B^1) + \mu_1(B^1)^{n-1}.
  \end{split}
\end{equation}
Applying this formula iteratively gives
\begin{equation}
  \Gamma_n = n \mu_1(B^1)^{n-1}.
\end{equation}
\begin{equation}
  \frac{\beta p}{\rho} =
  1 + \sum_{n=2}^\infty \rho^{n-1} \mu_1(B^1)^{n-1}
\end{equation}
But $\rho = \eta / \mu_1(B^1)$ so we have
\begin{equation}
  \frac{\beta p}{\rho} =
  1 + \sum_{n=2}^\infty \eta^{n-1} = 
  \sum_{n=0}^\infty \eta^{n} = \frac{1}{1 - \eta}
\end{equation}
which is convergent for all $|\eta| < 1$ although only positive volume fractions i.e.\ in the range $\eta \in [0,1]$ have any physical meaning.

Defining the more general integral
\begin{equation}
  \Gamma_n^{(k)} = \int_{\mathbb{E}_d^n} \mu_k (B^d \cap_{i=2}^n g_i B^d) \, d^n g
\end{equation}
For $d=3$ we have
\begin{equation}
  \int_{\mathbb{E}} \mu_k (A \cap g B^3) \, dg =
  \begin{cases}
    \omega_3 \mu_0(A) + \omega_2 \mu_1(A) +
    \omega_1 \mu_2(A) + \omega_0 \mu_3(A)
    & \qquad k=0 \\
    \omega_3 \mu_1(A) + \frac{\pi}{2} \omega_2 \mu_2(A)
    + 2 \omega_1 \mu_3(A)
    & \qquad k=1 \\
    \omega_3 \mu_2(A) + 2 \omega_2 \mu_3(A) & \qquad k=2 \\
    \omega_3 \mu_3(A) & \qquad k=3
  \end{cases}
\end{equation}
so the lowest-order integrals for $n=2$ evaluate to
\begin{align}
  \Gamma_2^{(0)} &= 2 \omega_0 \omega_3 + 4 \omega_1 \omega_2 \\
  \Gamma_2^{(1)} &= 4 \omega_1 \omega_3 + \pi \omega_2^2 \\
  \Gamma_2^{(2)} &= 4 \omega_2 \omega_3 \\
  \Gamma_2^{(3)} &= \omega_3^2.
\end{align}
Iterating the integrals gives (evaluated in order of simplicity)
\begin{align}
  \Gamma_n^{(3)} &=
  \omega_3 \Gamma_{n-1}^{(3)}
  = \omega_3^n \\
  \Gamma_n^{(2)} &=
  \omega_3 \Gamma_{n-1}^{(2)} + 2 \omega_2 \Gamma_{n-1}^{(3)}
  \nonumber \\ &=
  \omega_3 \Gamma_{n-1}^{(2)} + 2 \omega_2 \omega_3^{n-1}
  = 2 n \omega_2 \omega_3^{n-1} \\
  \Gamma_n^{(1)} &=
  \omega_3 \Gamma_{n-1}^{(1)} +
  \frac{\pi}{2} \omega_2 \Gamma_{n-1}^{(2)} +
  2 \omega_1 \Gamma_{n-1}^{(3)}
  \nonumber \\ &=
  \omega_3 \Gamma_{n-1}^{(1)} +
  (n-1) \pi \omega_2^2 \omega_3^{n-2} +
  2 \omega_1 \omega_3^{n-1}
  \nonumber \\ &=
  \frac{n(n-1)}{2} \pi \omega_2^2 \omega_3^{n-2} +
  2n \omega_1 \omega_3^{n-1}
  \\
  \Gamma_n^{(0)} &=
  \omega_3 \Gamma_{n-1}^{(0)} +
  \omega_2 \Gamma_{n-1}^{(1)} +
  \omega_1 \Gamma_{n-1}^{(2)} +
  \omega_0 \Gamma_{n-1}^{(3)}
  \nonumber \\ &=
  \omega_3 \Gamma_{n-1}^{(0)} +
  \frac{(n-1)(n-2)}{2} \pi \omega_2^3 \omega_3^{n-3} +
  2(n-1) \omega_1 \omega_2 \omega_3^{n-2} +
  \nonumber \\ & \quad
  2(n-1) \omega_1 \omega_2 \omega_3^{n-2} +
  \omega_0 \omega_3^{n-1}
  \nonumber \\ &=
  \frac{n(n-1)(n-2))}{6} \pi \omega_2^3 \omega_3^{n-3} +
  2 n(n-1) \omega_1 \omega_2 \omega_3^{n-2} +
  n \omega_0 \omega_3^{n-1}
\end{align}
This gives
\begin{equation}
  \begin{split}
    \frac{\beta p}{\rho} &=
    1 + \sum_{n=2}^\infty \frac{1}{n} \frac{\eta^{n-1}}{\omega_3^{n-1}} \Gamma_n
    =
    1 + \sum_{n=1}^\infty \frac{1}{n+1} \frac{\eta^n}{\omega_3^n} \Gamma_{n+1}
    \\ &=
    1 + \sum_{n=1}^\infty \frac{1}{n+1} \frac{\eta^n}{\omega_3^n}
    \left(
    \frac{n(n-1)(n+1))}{6} \pi \omega_2^3 \omega_3^{n-2} +
    2 n(n+1) \omega_1 \omega_2 \omega_3^{n-1} +
    (n+1) \omega_0 \omega_3^{n}
    \right)
    \\ &=
    1 + \sum_{n=1}^\infty \eta^n
    \left(
    \frac{n(n-1)}{6} \pi \frac{\omega_2^3}{\omega_3^2} +
    2 n \frac{\omega_1 \omega_2}{\omega_3} +
    \omega_0
    \right)
    \\ &=
    \pi \frac{\omega_2^3}{6\omega_3^2}
    \sum_{n=0}^\infty
    n(n-1) \eta^n
    +
    2 \frac{\omega_1 \omega_2}{\omega_3}
    \sum_{n=0}^\infty n \eta^n
    +
    \left( 1 + \omega_0 \sum_{n=1}^\infty \eta^n \right)
    \\ &=
    \pi \frac{\omega_2^3}{6\omega_3^2}
    \eta^2
    \sum_{n=0}^\infty
    n(n-1) \eta^{n-2}
    +
    2 \frac{\omega_1 \omega_2}{\omega_3}
    \eta
    \sum_{n=0}^\infty n \eta^{n-1}
    +
    \sum_{n=0}^\infty \eta^n
    \\ &=
    \pi \frac{\omega_2^3}{6\omega_3^2}
    \eta^2
    \frac{2}{(1-\eta)^3}
    +
    2 \frac{\omega_1 \omega_2}{\omega_3}
    \eta
    \frac{1}{(1-\eta)^2}
    +
    \frac{1}{1-\eta}
    \\ &=
    \pi \frac{\pi^3}{6 (16 \pi^2 / 9)}
    \eta^2
    \frac{2}{(1-\eta)^3}
    +
    2 \frac{2 \pi}{4 \pi / 3}
    \eta
    \frac{1}{(1-\eta)^2}
    +
    \frac{1}{1-\eta}
    \\ &=
    \frac{3 \pi^2}{16}
    \eta^2
    \frac{1}{(1-\eta)^3}
    +
    3
    \eta
    \frac{1}{(1-\eta)^2}
    +
    \frac{1}{1-\eta}
    \\ &=
    \frac{A\eta^2}{(1-\eta)^3}
    +
    \frac{3\eta (1-\eta)}{(1-\eta)^3}
    +
    \frac{(1-\eta)^2}{(1-\eta)^3}
    \\ &=
    \frac{A \eta^2
    +
    3\eta - 3\eta^2
    +
    1-2\eta + \eta^2}{(1-\eta)^3}
    \\ &=
    \frac{1 + \eta + (A - 2) \eta^2}{(1-\eta)^3}
  \end{split}
\end{equation}
\begin{align*}
  A = 3 &= \frac{3\pi^2}{16} + B \\
  B &= 3\left( 1 - \frac{\pi^2}{16} \right) \\
  B &= \frac{48 - 3\pi^2}{16} \\
  B &= 3 - \frac{\pi \omega_2^2}{\omega_3^2} \\
\end{align*}
For $d=2$ we have
\begin{equation}
  \int_{\mathbb{E}} \mu_k (A \cap g B^2) \, dg =
  \begin{cases}
    \omega_2 \mu_0(A) + \omega_1 \mu_1(A) + \omega_0 \mu_2(A) & \qquad k=0 \\
    \omega_2 \mu_1(A) + \frac{\pi}{2} \omega_1 \mu_2(A) & \qquad k=1 \\
    \omega_2 \mu_2(A) & \qquad k=2
  \end{cases}
\end{equation}
so the lowest-order integrals for $n=2$ evaluate to
\begin{align}
  \Gamma_2^{(0)} &= 2 \omega_0 \omega_2 + \omega_1^2 \\
  \Gamma_2^{(1)} &= \pi \omega_1 \omega_2 \\
  \Gamma_2^{(2)} &= \omega_2^2.
\end{align}
Iterating the integrals gives (evaluated in order of simplicity)
\begin{align}
  \Gamma_n^{(2)} &=
  \omega_2 \Gamma_{n-1}^{(2)}
  = \omega_2^n \\
  \Gamma_n^{(1)} &=
  \omega_2 \Gamma_{n-1}^{(1)}
  + \frac{\pi}{2} \omega_1 \Gamma_{n-1}^{(2)}
  \nonumber \\ &=
  \omega_2 \Gamma_{n-1}^{(1)}
  + \frac{\pi}{2} \omega_1 \omega_2^{n-1}
  = \frac{n \pi}{2} \omega_1 \omega_2^{n-1} \\
  \Gamma_n^{(0)} &=
  \omega_2 \Gamma_{n-1}^{(0)}
  + \omega_1 \Gamma_{n-1}^{(1)}
  + \omega_0 \Gamma_{n-1}^{(2)}
  \nonumber \\ &=
  \omega_2 \Gamma_{n-1}^{(0)}
  + \frac{(n-1) \pi}{2} \omega_1^2 \omega_2^{n-2}
  + \omega_0 \omega_2^{n-1}
  \nonumber \\ &=
  \sum_{m=1}^n \omega_2^{n-m} \left(
  \frac{(m-1)\pi}{2} \omega_1^2 \omega_2^{m-2}
  + \omega_0 \omega_2^{m-1}
  \right)
  \nonumber \\ &=
  \omega_2^{n-2}
  \sum_{m=1}^n \left(
  \frac{(m-1)\pi}{2} \omega_1^2
  + \omega_0 \omega_2
  \right)
  \nonumber \\ &=
  \omega_2^{n-2}
  \left(
  \frac{n(n-1)\pi}{4} \omega_1^2
  + n \omega_0 \omega_2
  \right)
\end{align}
This gives
\begin{equation}
  \begin{split}
    \frac{\beta p}{\rho} &=
    1 + \sum_{n=2}^\infty \frac{1}{n} \frac{\eta^{n-1}}{\omega_2^{n-1}} \Gamma_n
    \\ &=
    1 + \sum_{n=1}^\infty \frac{1}{n+1} \frac{\eta^n}{\omega_2^n} \Gamma_{n+1}
    \\ &=
    1 + \sum_{n=1}^\infty \frac{1}{n+1} \frac{\eta^n}{\omega_2^n}
    \omega_2^{n-1}
    \left(
    \frac{n(n+1)\pi}{4} \omega_1^2
    + (n+1) \omega_0 \omega_2
    \right)
    \\ &=
    1 + \sum_{n=1}^\infty \eta^n
    \left(
    \frac{n\pi}{4} \frac{\omega_1^2}{\omega_2}
    + \omega_0
    \right)
    \\ &=
    1 + \sum_{n=1}^\infty (n+1) \eta^n =
    \sum_{n=0}^\infty n \eta^{n-1}
    = \frac{1}{(1-\eta)^2}
  \end{split}
\end{equation}
which is absolutely convergent for all $|\eta| < 1$, although only values in the range $\eta \in [0,1]$ have any physical meaning.

\subsection{Inhomogeneous liquid state theory}

\subsubsection{Solvation physics}
\subsubsection{Density functional theory (DFT)}

Classic texts are \cite{Evans1979,Evans1992}, and a more recent review \cite{Roth2010}.
Also mention \cite{Lutsko2010} for more of a focus on crystallisation.
This exposition follows \cite{Roth2010} mainly.

Density functional theory traces back to 

Rigorously prove that \cite{Evans1979,Evans1992}
\begin{equation}
  \Omega[\{\rho_i\}] =
  \mathcal{F}[\{\rho_i\}]
  + \sum_{i=1}^\nu \int d^d \vec{r} \rho_i(\vec{r}) (\phi(\vec{r}) - \mu_i)
\end{equation}

\begin{equation}
  \left.
  \frac{\delta \Omega[\{\rho_i\}]}{\delta \rho_i(\vec{r})}
  \right|_{\{\rho_i(\vec{r}) = \rho_i^0(\vec{r})\}}
  = 0.
\end{equation}

Split into ideal and excess parts
\begin{equation}
  \mathcal{F}[\{\rho_i\}] =
  \mathcal{F}_{id}[\{\rho_i\}] + \mathcal{F}_{ex}[\{\rho_i\}]
\end{equation}
where the ideal part is
\begin{equation}
  \beta \mathcal{F}_{id}[\{\rho_i\}] =
  \sum_{i=1}^\nu \int d^d \vec{r} \rho_i(\vec{r})
  (\ln{\Lambda_i^d \rho_i(\vec{r})} - 1)
\end{equation}

\begin{itemize}
\item Contrast mechanical problem (e.g.\ simulations) with inverse problem
\item Summarise successes
\end{itemize}

\paragraph{Fundamental measure theory (FMT)}
\begin{itemize}
\item $d+1$ weight functions
\item Exact decomposition of Mayer-f bond
\end{itemize}

We have 4 scalar weight functions
\begin{subequations}
  \begin{align}
    \omega_3^i(\vec{r}) &= \Theta(R_i - r), \\
    \omega_2^i(\vec{r}) &= \delta(R_i - r), \\
    \omega_1^i(\vec{r}) &= \frac{\omega_2^i(\vec{r})}{4\pi R_i}, \\
    \omega_0^i(\vec{r}) &= \frac{\omega_2^i(\vec{r})}{4\pi R_i^2},
  \end{align}
\end{subequations}
and 2 vector weight functions
\begin{subequations}
  \begin{align}
    \vec{\omega}_2^i(\vec{r}) &=
    \frac{\vec{r}}{r} \delta(R_i - r), \\
    \vec{\omega}_1^i(\vec{r}) &=
    \frac{\vec{\omega}_1^i(\vec{r})}{4\pi R_i}.
  \end{align}
\end{subequations}

\subsubsection{Heterogeneous approaches to the homogeneous liquid}
\paragraph{Potential distribution theorem}
\paragraph{Scaled particle theory}
