\section{Liquid state theory}

Here we talk in general terms about descriptions of the liquid state.
Broadly speaking, in its historical development approaches can be placed inside one of two categories.
Namely, theories involving
\begin{enumerate}
  \item Local geometric approximations capturing the short range interactions (free volume theory/cell theory, scaled particle theory), and
  \item Integral equations (Ornstein-Zernike closures, density functional theory) which properly treat the long range correlations.
\end{enumerate}
These two approaches are not mutually exclusive, and hybrid theories can improve on.
For instance, fundamental measure theory (FMT) involves the synthesis of integral geometry with the formalism of density functional theory which involves minimising a functional (i.e.\ an integral equation).

Classical theory of phase transitions (Landau).
Van der waals theory.
How does the transition occur?
Metastability leads into kinetics.

Kinetics vs thermodynamics.
Thermodynamic driving force vs activation barrier.

Relaxation behaviour controlled by activation barrier.
A thermal fluctuation which takes the system over the barrier%
\marginfootnote{These fluctuations are conventionally called \emph{instantons} as they spontaneously appear and vanish just like virtual particles in fundamental physics.
  The name for this relatively straightforward phenomenon is thus a reference to a much more counterintuitive and bizarre phenomenon, because physicists are good at making helpful analogies.}
occurs with rate $\exp{-\beta \Delta U}$ \cite{Langer}.

Liquids:
Free volume theory.
Early curvature corrections.
Free volume theory depends on the free volume (duh).
This is an example of an integral geometric theory.
Morphological 

Technical - handwavey:
Motion invariance: translation and rotation invariance.
Additivity: the energy is extensive (even for small system sizes)
Continuity: the function is well behaved

A summary of results in the modern formalism as given by standard texts (notably references \cite{Hansen2013} and \cite{Santos2016}) that will be used throughout the thesis.

\subsection{Interaction potentials}
\subsection{Simple liquids}
\subsection{Hard sphere model system}

\subsubsection{Grand canonical ensemble}
\paragraph{Averages}
The partition function and free energy
\paragraph{Correlations}
Particle densities $\rho^{(n)}$ and distribution functions $g^{(n)}$.

\subsection{Homogeneous liquid state theory}

\subsubsection{Virial expansion of the equation of state}

\paragraph{Distribution function form}
Pressure in terms of pair distribution function $g(r)$.

For pair potentials:
\begin{equation}
  \frac{\beta P}{\rho} =
  1 - \frac{2 \pi \beta \rho}{3}
  \int_0^\infty v'(r) g(r) r^3 \, dr
\end{equation}
Generalisation to many-body potentials requires higher order distribution functions.
Difficulty of discontinuities overcome by introducing cavity function
\begin{equation}
  y(r) = e^{\beta v(r)} g(r)
\end{equation}
which is continuous \todo{Find out why this is continuous} leading to
\begin{equation}
  \frac{\beta P}{\rho} =
  1 + \frac{2 \pi \beta \rho}{3}
  \int_0^\infty e'(r) g(r) r^3 \, dr,
\end{equation}
where Boltzmann factor of pair potential is
\begin{equation}
  e(r) = e^{\beta v(r)}.
\end{equation}

\paragraph{Diagrammatic form}
Low density expansion of pressure in series with virial coefficients.

\begin{equation}
  \frac{\beta P}{\rho} =
  1 + \sum_{i=2}^\infty B_i(T) \rho^{i-1}
\end{equation}
where $B_i$ are the virial coefficients.

Generalisation to a binary mixture \todo{Where does this come from? What do the diagrams look like? Are they simpler than the distribution function diagrams?}: \cite{Hansen-Goos2014}
\begin{equation}\label{eq:virial-series-binary}
  \Phi = \sum_{n=2}^\infty \sum_{j=0}^{n}
  \frac{1}{n-1} {n \choose j} B_n^{[j]} \rho_1^{n-j} \rho_2^j
\end{equation}

\paragraph{Empirical Carnahan-Starling equation of state for hard spheres}

The excess free energy is determined from the equation of state by
\begin{equation}
  \frac{\beta F^{ex}}{N}
  = \int_0^\eta \left( \frac{\beta p}{\rho} - 1 \right) \, \frac{d\eta'}{\eta'},
\end{equation}
giving the excess chemical potential from the thermodynamic relation
\begin{equation}\label{eq:chemical-potential}
  \beta \mu^{ex}[p]
  = \beta \left( \frac{\partial F^{ex}}{\partial N} \right)_{V,T}
  = \left( \frac{\beta p}{\rho} - 1 \right)
  + \int_0^\eta \left( \frac{\beta p}{\rho} - 1 \right) \, \frac{d\eta'}{\eta'}.
\end{equation}

The Carnahan-Starling equation of state approximates the pressure for hard spheres as \cite{Carnahan1969}
\begin{equation}\label{eq:cs-pressure}
  \frac{\beta p_{cs}}{\rho} = \frac{1 + \eta + \eta^2 - \eta^3}{(1-\eta)^3},
\end{equation}
which gives the excess chemical potential using \eqref{eq:chemical-potential} as
\begin{equation}\label{eq:cs-mu}
  \beta \mu_{cs}^{ex} = \frac{8\eta - 9\eta^2 + 3\eta^3}{(1-\eta)^3}.
\end{equation}

\subsubsection{Free energy from distribution functions}
\paragraph{Distribution function theories}
\paragraph{Kirkwood superposition approximation}
\paragraph{Distribution functions from direct correlation functions: Ornstein-Zernike equation}

\subsubsection{Beyond hard spheres: perturbation theory and the mean field approximation}

\subsection{Inhomogeneous liquid state theory}

\subsubsection{Solvation physics}

\section{Functional calculus}

The chain rule of functional calculus is
\todo{Insert citation to some calculus of variations reference, and Hansen2013/Bob's reviews for application to liquids.}
\begin{equation*}
  \frac{\delta f}{\delta u(\vec{r})} =
  \int
  \frac{\delta f}{\delta v(\vec{r}')}
  \frac{\delta v(\vec{r}')}{\delta u(\vec{r})}
  \, d\vec{r}'
\end{equation*}
and inverse derivatives are found via
\begin{equation*}
  \int
  \frac{\delta u(\vec{r})}{\delta v(\vec{r}')}
  \frac{\delta v(\vec{r}')}{\delta u(\vec{r})}
  \, d\vec{r} =
  \delta(\vec{r} - \vec{r}')
\end{equation*}

\section{Correlation functions}

\begin{equation}
  H^{(n)}(\vec{r}^n) =
  \left\langle
  \prod_{i=1}^n
  \Big[ \rho(\vec{r}_i) - \rho^{(1)}(\vec{r}_i) \Big]
  \right\rangle
  \qquad \forall \; n \ge 2
\end{equation}
\begin{equation}
  \begin{aligned}
    H^{(2)}(\vec{r}_1, \vec{r}_2) &=
    \left\langle
    \left[ \rho(\vec{r}_1) - \rho^{(1)}(\vec{r}_1) \right]
    \left[ \rho(\vec{r}_2) - \rho^{(1)}(\vec{r}_2) \right]
    \right\rangle \\
    &=
    \big\langle \rho(\vec{r}_1) \rho(\vec{r}_2) \big\rangle -
    \rho^{(1)}(\vec{r}_1) \rho^{(1)}(\vec{r}_2) \\
    &=
    \rho^{(2)}(\vec{r}_1, \vec{r}_2) +
    \rho^{(1)}(\vec{r}_1) \delta(\vec{r}_1 - \vec{r}_2) -
    \rho^{(1)}(\vec{r}_1) \rho^{(1)}(\vec{r}_2) \\
    &=
    \rho^{(1)}(\vec{r}_1) \rho^{(1)}(\vec{r}_2) h^{(2)}(\vec{r}_1, \vec{r}_2)
    +
    \rho^{(1)}(\vec{r}_1) \delta(\vec{r}_1 - \vec{r}_2)
  \end{aligned}
\end{equation}
where
\begin{equation}
  h^{(n)}(\vec{r}^n) \equiv g^{(n)}(\vec{r}^n) - 1
\end{equation}

Density-density correlation functions
\begin{equation}
  H^{(n)}(\vec{r}^n) =
  \left\langle
  \prod_{i=1}^n
  \Big[ \rho(\vec{r}_i) - \big\langle\rho(\vec{r}_i)\big\rangle \Big]
  \right\rangle
  \qquad \forall \; n \ge 2
\end{equation}
They are obtained from the total grand potential by repeat functional differentiation, as in
\begin{equation}
  \begin{aligned}
  H^{(n)}(\vec{r}^n) &=
  - \frac{\delta^n \beta \Omega}{\delta \beta\psi(\vec{r}_1) \delta \beta\psi(\vec{r}_2) \cdots \delta \beta\psi(\vec{r}_n)} \\
  &=
  \frac{\delta^{n-1} \rho(\vec{r}_1)}{\delta \beta\psi(\vec{r}_2) \delta \beta\psi(\vec{r}_3) \cdots \delta \beta\psi(\vec{r}_n)}.
  \end{aligned}
\end{equation}
I.e.\ the grand potential is the generating functional for the density-density correlation functions.

\section{Density functional theory}

\subsection{Functional form of thermodynamic potentials}

From the Legendre transform of $\beta \Omega$
\begin{equation}
  \begin{aligned}
    \Omega[\rho(\vec{r})] &=
    F[\rho(\vec{r})] -
    \int \rho(\vec{r}) \psi(\vec{r}) \, d\vec{r} \\
    &=
    F_{id}[\rho(\vec{r})] +
    F_{ex}[\rho(\vec{r})] -
    \int \rho(\vec{r}) \psi(\vec{r}) \, d\vec{r}
  \end{aligned}
\end{equation}
after defining intrinsic chemical potential
\begin{equation}\label{eq:intrinsic-chemical-potential}
  \psi(\vec{r}) = \mu - V_{ext}(\vec{r})
\end{equation}

For the ideal gas
\begin{align*}
  \Xi_{id} &=
  \sum_{N=0}^\infty
  \frac{(e^{\beta\mu} Z_1)^N}{N!}
  = \exp{\left( Z_1 e^{\beta \mu} \right)} \\
  Z_1 &= \int e^{-\beta V_{ext}(\vec{r})} \, d\vec{r} \\
  \beta\Omega_{id}[V_{ext}] &=
  - \ln{\Xi_{id}}
  =
  - e^{\beta\mu} \int e^{-\beta V_{ext}(\vec{r})} \, d\vec{r} \\
  \rho^{(1)}(\vec{r}) &=
  \frac{\langle N \rangle e^{-\beta V_{ext}(\vec{r})}}
       {\int e^{-\beta V_{ext}(\vec{r}')} \, d\vec{r}'} =
  \frac{\langle N \rangle}{Z_1} e^{-\beta V_{ext}(\vec{r})}
  \\
  \beta V_{ext}(\vec{r})
  &=
  -\ln
  \left(
  \frac{\rho^{(1)}(\vec{r})}{\langle N \rangle}
  \int e^{-\beta V_{ext}(\vec{r}')} \, d\vec{r}'
  \right)
  \\
  e^{\beta \mu} &= \lambda^d \rho
  \\
  F[\rho]
  &=
  - \lambda^d \rho
  \int e^{-\beta V_{ext}(\vec{r})} \, d\vec{r} -
  \int
  \rho(\vec{r})
  (V_{ext}(\vec{r}) - \mu)
  \, d\vec{r}
  \\
  &=
  - \lambda^d \rho
  \int e^{-\beta V_{ext}(\vec{r})} \, d\vec{r} -
  \int
  \rho(\vec{r})
  V_{ext}(\vec{r})
  \, d\vec{r} -
  \int
  \rho(\vec{r}) \mu
  \, d\vec{r}
  \\
  &=
  - \lambda^d \rho
  \int e^{-\beta V_{ext}(\vec{r})} \, d\vec{r} -
  \int
  \rho(\vec{r})
  \ln {\frac{Z_1}{\langle N \rangle} \rho(\vec{r})}
  \, d\vec{r} +
  \int
  \rho(\vec{r}) \mu
  \, d\vec{r}
  \\
  &=
  - \lambda^d \rho
  \int
  \frac{\rho^{(1)}(\vec{r})}{\langle N \rangle}
  \left( \int e^{-\beta V_{ext}(\vec{r}')} \, d\vec{r}' \right)
  \, d\vec{r} -
  \int
  \rho(\vec{r})
  \ln{\left(
  \frac{\rho^{(1)}(\vec{r})}{\langle N \rangle}
  \int e^{-\beta V_{ext}(\vec{r}')} \, d\vec{r}'
  \right)}
  (V_{ext}(\vec{r}) - \mu)
  \, d\vec{r}
  \\
  \beta\Omega_{id}[\rho] &=
  F[\rho] + \int \rho(\vec{r}) (V_{ext}(\vec{r}) - \mu) \\
  &=
  F[\rho] + \int \rho(\vec{r}) (V_{ext}(\vec{r}) - \mu) \\
\end{align*}
Following \cite{Ashcroft1996}, we write the single-particle density as
\begin{equation*}
  \rho^{(1)}(\vec{r}) =
  \frac{1}{\Xi}
  \Tr{ \left(
    e^{-\beta U_N(\vec{r}^N)} \sum_{i=1}^N \delta(\vec{r} - \vec{r}')
    \right)
  }
\end{equation*}
which in the absence of particle interactions reduces to
\begin{equation*}
  \rho^{(1)}(\vec{r}) =
  \frac{\langle N \rangle e^{-\beta V_{ext}(\vec{r})}}
       {\int e^{-\beta V_{ext}(\vec{r}')} \, d\vec{r}'}
\end{equation*}
We note that the energetic contribution to the free energy arising solely from the external potential is
\begin{equation*}
  \int \rho(\vec{r}') V_{ext}(\vec{r}') \, d\vec{r}'
\end{equation*}
The intrinsic free energy of the ideal gas is then obtained by subtracting the contributions from the external potential, as in
\begin{equation*}
  \begin{aligned}
  F_{id}[\rho] &=
  F[\rho] -
  \int \rho(\vec{r}') V_{ext}(\vec{r}') \, d\vec{r}' \\
  &=
  F[\rho] -
  \int \rho(\vec{r}') V_{ext}(\vec{r}') \, d\vec{r}' \\
  \end{aligned}
\end{equation*}

Derivatives of the ideal gas free energy.
The ideal gas free energy is given as
\begin{equation*}
  \beta F_{id}[\rho(\vec{r})] =
  \int \rho(\vec{r}) (\ln{\lambda^d \rho(\vec{r})} - 1) \, d\vec{r},
\end{equation*}
so the first functional derivative is
\begin{equation*}
  \frac{\delta \beta F_{id}}{\delta \rho(\vec{r})} =
  \ln{\lambda^d \rho(\vec{r})}.
\end{equation*}
To obtain the higher order functional derivatives it is helpful to write this as an integral with a delta function
\begin{equation*}
  \frac{\delta \beta F_{id}}{\delta \rho(\vec{r})} =
  \int \delta{(\vec{r}' - \vec{r})}
  \ln{\lambda^d \rho(\vec{r}')} \, d\vec{r}',
\end{equation*}
so we can obtain the second derivative as
\begin{equation*}
  \frac{\delta^2 \beta F_{id}}{\delta \rho(\vec{r}) \delta \rho(\vec{r}')} =
  \frac{\delta(\vec{r}'-\vec{r})}{\rho(\vec{r})}.
\end{equation*}
Iterating this procedure gives us the $n$th functional derivative as
\begin{equation}
  \begin{aligned}
    \frac{\delta^n \beta F_{id}}{\delta \rho(\vec{r}_1) \delta \rho(\vec{r}_2) \cdots \delta \rho(\vec{r}_n)} &=
    \frac{\partial^{n-1} (\ln{\lambda^d \rho(\vec{r})})}{\partial \rho(\vec{r})^{n-1}}
    \prod_{i=2}^n \delta(\vec{r}_i - \vec{r}_1) \\
    &=
    (-1)^{n-1}
    \frac{(n-2)!}{\rho(\vec{r})^{n-1}}
    \prod_{i=2}^n \delta(\vec{r}_i - \vec{r}_1),
  \end{aligned}
\end{equation}
where the last line is valid for all $n \ge 2$.

\subsection{Correlation functions}

Excess free energy is the generating functional for direct correlations
\begin{equation}\label{eq:direct-correlations}
  c^{(n)}(\vec{r}^n) =
  - \frac{\delta^n \beta F_{ex}}{\delta \rho(\vec{r}_1)\delta \rho(\vec{r}_2) \cdots \delta \rho(\vec{r}_n)}
\end{equation}

We have to compute a derivative like
\begin{equation}\label{eq:intrinsic-chemical-potential-derivative-1}
  \frac{\delta}{\delta\rho(\vec{r})}
  \left(
  \int \psi(\vec{r}') \rho(\vec{r}') \, d\vec{r}'
  \right)
  =
  \psi(\vec{r}) +
  \int
  \rho(\vec{r}') \frac{\delta\psi(\vec{r}')}{\delta\rho(\vec{r})}
  \, d\vec{r}'.
\end{equation}
The functional derivative on the right-hand side of \eqref{eq:intrinsic-chemical-potential-derivative-1} is a little odd.
In general the external potential $V_{ext}(\vec{r})$%
\marginfootnote{And thus $\psi(\vec{r})$ through \eqref{eq:intrinsic-chemical-potential}}
is fixed so we consider the density profile as relaxing in response to perturbations from the potential i.e.\ terms like \[ \frac{\delta \rho(\vec{r})}{\delta \psi(\vec{r}')}. \]
However, here the \emph{inverse} derivative appears.
This must satisfy the inversion formula
\begin{equation}\label{eq:intrinsic-chemical-potential-inverse-derivative}
  \int
  \frac{\delta \rho(\vec{r}_1)}{\delta \psi(\vec{r}')}
  \frac{\delta \psi(\vec{r}')}{\delta \rho(\vec{r}_2)}
  \, d\vec{r}' =
  \delta(\vec{r}_1 - \vec{r}_2),
\end{equation}
which will be very important for obtaining Ornstein-Zernike relations later.
Considering the external field as the control parameter, and noting the definition of $\psi(\vec{r})$ in \eqref{eq:intrinsic-chemical-potential} we take
\begin{equation*}
  \frac{\delta\psi(\vec{r}')}{\delta\rho(\vec{r})} = 0.
\end{equation*}
This expression satisfies \eqref{eq:intrinsic-chemical-potential-inverse-derivative}, being zero in general except where $\psi(\vec{r}')$ is used as an intermediate function in the (functional) chain rule expression.
With this expression \eqref{eq:intrinsic-chemical-potential-derivative-2} becomes
\begin{equation}\label{eq:intrinsic-chemical-potential-derivative-2}
  \frac{\delta}{\delta\rho(\vec{r})}
  \left(
  \int \psi(\vec{r}') \rho(\vec{r}') \, d\vec{r}'
  \right)
  =
  \psi(\vec{r}).
\end{equation}

In equilibrium
\begin{equation}
  \left.
  \frac{\delta \Omega[\rho]}{\delta\rho(\vec{r})}
  \right|_{\rho(\vec{r}) = \rho^{(1)}(\vec{r})}
  = 0
\end{equation}
and consequently all higher-order derivatives must be zero, i.e.\
\begin{equation}
  \frac{\delta^n \Omega[\rho]}{\delta\rho(\vec{r}_1) \delta\rho(\vec{r}_2) \cdots \delta\rho(\vec{r}_n)} = 0 \qquad \forall \; n \ge 1
\end{equation}
therefore we have
\begin{equation}
  \frac{\delta^n F[\rho]}{\delta \rho(\vec{r}_1) \delta \rho(\vec{r}_2) \cdots \delta \rho(\vec{r}_n)} -
  \frac{\delta}{\delta\rho(\vec{r})}
  \left(
  \int \psi(\vec{r}') \rho(\vec{r}') \, d\vec{r}'
  \right)
  = 0
\end{equation}
or using \eqref{eq:direct-correlations} this becomes
\begin{equation*}
  c^{(n)}(\vec{r}^n) =
  \frac{\delta^n \beta F_{id}[\rho]}{\delta \rho(\vec{r}_1) \delta \rho(\vec{r}_2) \cdots \delta \rho(\vec{r}_n)} -
  \frac{\delta^n}{\delta \rho(\vec{r}_1) \delta \rho(\vec{r}_2) \cdots \delta \rho(\vec{r}_n)}
  \left(
  \int \beta\psi(\vec{r}') \rho(\vec{r}') \, d\vec{r}'
  \right)
\end{equation*}
At the one-body level we have:
\begin{equation}\label{eq:c1}
  \begin{aligned}
    c^{(1)}(\vec{r}) &=
    \frac{\delta \beta F_{id}}{\delta \rho(\vec{r})} -
    \frac{\delta}{\delta \rho(\vec{r})^{(1)}}
    \left(
    \int \beta\psi(\vec{r}') \rho(\vec{r}') \, d\vec{r}'
    \right) \\
    &=
    \ln{\lambda^d \rho(\vec{r})} -
    \beta\psi(\vec{r})
  \end{aligned}
\end{equation}
Which gives the equilibrium density as
\begin{equation}
  \rho^{(1)}(\vec{r}) = \lambda^{-d} \exp{\left(\beta\psi(\vec{r}) + c^{(1)}(\vec{r})\right)}
\end{equation}
We obtain the two-body correlations by functionally differentiating \eqref{eq:c1}, as in
\begin{equation}\label{eq:c2}
  \begin{aligned}
    c^{(2)}(\vec{r}_1, \vec{r}_2) &=
    \frac{\delta c^{(1)}(\vec{r}_1)}{\delta \rho^{(1)}(\vec{r}_2)} \\
    &=
    \frac{\delta(\vec{r}_2 - \vec{r}_1)}{\rho^{(1)}(\vec{r}_1)} -
    \frac{\delta\beta\psi(\vec{r}_1)}{\delta \rho(\vec{r}_2)}
  \end{aligned}
\end{equation}
From \eqref{eq:intrinsic-chemical-potential-inverse-derivative} we get
\begin{equation*}
  \begin{aligned}
    \delta(\vec{r}_1 - \vec{r}_2) &=
    \int
    \frac{\delta \rho(\vec{r}_1)}{\delta \psi(\vec{r}')}
    \frac{\delta \psi(\vec{r}')}{\delta \rho(\vec{r}_2)}
    \, d\vec{r}' \\
    &=
    \int
    H^{(2)}(\vec{r}_1, \vec{r}')
    \left(
    \frac{\delta(\vec{r}' - \vec{r}_2)}{\rho^{(1)}(\vec{r}')} -
    c^{(2)}(\vec{r}', \vec{r}_2)
    \right)
    \, d\vec{r}' \\
    &=
    \rho^{(1)}(\vec{r}_1)
    \left(
    h^{(2)}(\vec{r}_1, \vec{r}_2) -
    c^{(2)}(\vec{r}_1, \vec{r}_2)
    \right) +
    \delta(\vec{r}_1 - \vec{r}_2) - \\
    &\qquad
    \rho^{(1)}(\vec{r}_1)
    \int
    \rho^{(1)}(\vec{r}')
    h^{(2)}(\vec{r}_1, \vec{r}')
    c^{(2)}(\vec{r}', \vec{r}_2)
    \, d\vec{r}'
  \end{aligned}
\end{equation*}
which rearranges to give the Ornstein-Zernike equation
\begin{equation}
  h^{(2)}(\vec{r}_1, \vec{r}_2) =
  c^{(2)}(\vec{r}_1, \vec{r}_2) +
  \int
  \rho^{(1)}(\vec{r}')
  h^{(2)}(\vec{r}_1, \vec{r}')
  c^{(2)}(\vec{r}', \vec{r}_2)
  \, d\vec{r}'.
\end{equation}
This is a classic result in liquid state theory (cf.\ Refs.\ \cite{Ornstein1914,Hansen2010,Evans1979}) though normally it is expressed for the uniform liquid $\rho^{(1)}(\vec{r}) = \rho$ for spherically symmetric pair potentials thus
\begin{equation}
  h^{(2)}(r) =
  c^{(2)}(r) +
  \rho
  \int
  h^{(2)}(r)
  c^{(2)}(|\vec{r}' - \vec{r}|)
  \, d\vec{r}',
\end{equation}
where $r = |\vec{r}_2 - \vec{r}_1|$.
The integral is a convolution, so it simplifies under Fourier transform to
\begin{equation}
  \tilde{h}(\vec{k}) =
  \tilde{c}(\vec{k}) +
  \rho \tilde{h}(\vec{k}) \tilde{c}(\vec{k})
\end{equation}
or rearranging for
\begin{equation}
  \tilde{h}(\vec{k}) =
  \frac{\tilde{c}(\vec{k})}{1 - \rho \tilde{c}(\vec{k})}.
\end{equation}
Note that the static structure factor is defined as the Fourier transform of the two-body distribution function, i.e.\
\begin{equation}
  \begin{aligned}
    S(\vec{k}) \equiv \tilde{g}^{(2)}(\vec{k}) &=
    \delta(\vec{k}) + \tilde{h}(\vec{k}) \\
    &=
    \delta(\vec{k}) +
    \frac{\tilde{c}(\vec{k})}{1 - \rho \tilde{c}(\vec{k})}.
  \end{aligned}
\end{equation}

\subsection{Generalised Ornstein-Zernike equations}

This section follows \cite{Barrat1988}.

Thus for higher $n$ we have
\begin{equation}
  \begin{aligned}
    c^{(n)}(\vec{r}^n) &=
    \frac{\delta^{n-1} c^{(1)}(\vec{r}_1)}{\delta \rho(\vec{r}_2) \delta \rho(\vec{r}_3) \cdots \delta \rho(\vec{r}_n)} \\
    &=
    (-1)^n
    \frac{(n-2)!}{\rho(\vec{r}_1)^{n-1}}
    \prod_{i=2}^n \delta(\vec{r}_i - \vec{r}_1) -
    \frac{\delta^{n-1} \beta\psi(\vec{r}_1)}{\delta \rho(\vec{r}_2) \delta \rho(\vec{r}_3) \cdots \delta \rho(\vec{r}_n)}
  \end{aligned}
\end{equation}
or
\begin{equation}
  \frac{\delta^{n-1} \beta\psi(\vec{r}_1)}{\delta \rho(\vec{r}_2) \delta \rho(\vec{r}_3) \cdots \delta \rho(\vec{r}_n)} =
  (-1)^n
  \frac{(n-2)!}{\rho(\vec{r}_1)^{n-1}}
  \prod_{i=2}^n \delta(\vec{r}_i - \vec{r}_1) -
  c^{(n)}(\vec{r}^n)
\end{equation}

\begin{equation*}
  H^{(n)}(\vec{r}^n) =
  \frac{\delta^{n-1} \rho(\vec{r}_1)}{\delta \beta\psi(\vec{r}_2) \delta \beta\psi(\vec{r}_3) \cdots \delta \beta\psi(\vec{r}_n)}.
\end{equation*}

Defining
\begin{equation*}
  K^{(n)}(\vec{r}^n) =
  \frac{\delta^{n-1} \beta\psi(\vec{r}_1)}{\delta \rho(\vec{r}_2) \delta \rho(\vec{r}_3) \cdots \delta \rho(\vec{r}_n)}
\end{equation*}
we have
\begin{equation*}
  \frac{\delta K^{(n)}(\vec{r}^n)}{\delta \rho(\vec{r}_{n+1})} =
  K^{(n+1)}(\vec{r}^{n+1}).
\end{equation*}
and
\begin{equation*}
  \begin{aligned}
    \frac{\delta H^{(n)}(\vec{r}^n)}{\delta \rho(\vec{r}_{n+1})}
    &=
    \int
    \frac{\delta H^{(n)}(\vec{r}^n)}{\delta \psi(\vec{r}')}
    \frac{\delta \psi(\vec{r}')}{\delta \psi(\vec{r}_{n+1})}
    \, d\vec{r}' \\
    &=
    \int
    H^{(n+1)}(\vec{r}^n, \vec{r}')
    K^{(2)}(\vec{r}', \vec{r}_{n+1})
    \, d\vec{r}' \\
    &=
    H^{(n+1)} \otimes K^{(2)}(\vec{r}^{n+1}).
  \end{aligned}
\end{equation*}
In this form the Ornstein-Zernike equation can be written.
\begin{equation*}
  \begin{aligned}
    \delta(\vec{r}_1 - \vec{r}_2) &=
    \int
    \frac{\delta \rho(\vec{r}_1)}{\delta \psi(\vec{r}')}
    \frac{\delta \psi(\vec{r}')}{\delta \rho(\vec{r}_2)}
    \, d\vec{r}' \\
    &=
    \int
    H^{(2)}(\vec{r}_1, \vec{r}') K^{(2)}(\vec{r}', \vec{r}_2)
    \, d\vec{r}' \\
    &=
    H^{(2)} \otimes K^{(2)} (\vec{r}^2)
  \end{aligned}
\end{equation*}
Taking functional derivatives of this expression gives us a hierarchy of generalised Ornstein-Zernike equations.
For example, the next equation in the hierarchy is
\begin{equation*}
  H^{(2)} \otimes K^{(3)} (\vec{r}^3) +
  H^{(3)} \otimes K^{(2)} \otimes K^{(2)} (\vec{r}^3) = 0
\end{equation*}
The next functional derivative
\begin{equation}
  \begin{aligned}
  H^{(2)} \otimes K^{(4)} (\vec{r}^4) & \\
  + \; 2 H^{(3)} \otimes K^{(3)} \otimes K^{(2)} (\vec{r}^4) & \\
  + \; H^{(4)} \otimes K^{(2)} \otimes K^{(2)} \otimes K^{(2)} (\vec{r}^4)
  &= 0
  \end{aligned}
\end{equation}
And the next one%
\todo{Can we find a general formula? I notice some constraints on the indices: the sum of the indices $\{m\}$ in the $K^{(m)}$ terms must add up to $n-1$ so that the right number of independent variables are returned (the extra one is provided by the $H^{(l)}$ function giving the $\vec{r}^n$ total.}
\begin{equation}
  \begin{aligned}
    H^{(2)} \otimes K^{(5)} (\vec{r}^5) & \\
    + \; 3 H^{(3)} \otimes K^{(3)} \otimes K^{(3)} (\vec{r}^5) & \\
    + \; 2 H^{(3)} \otimes K^{(4)} \otimes K^{(2)} (\vec{r}^5) & \\
    + \; 5 H^{(4)} \otimes K^{(3)} \otimes K^{(2)} \otimes K^{(2)} (\vec{r}^5) & \\
    + \; H^{(5)} \otimes K^{(2)} \otimes K^{(2)} \otimes K^{(2)} \otimes K^{(2)} (\vec{r}^5)
    &= 0
  \end{aligned}
\end{equation}

\subsubsection{Density functional theory (DFT)}

Classic texts are \cite{Evans1979,Evans1992}, and a more recent review \cite{Roth2010}.
Also mention \cite{Lutsko2010} for more of a focus on crystallisation.
This exposition follows \cite{Roth2010} mainly.

Density functional theory traces back to 

Rigorously prove that \cite{Evans1979,Evans1992}
\begin{equation}
  \Omega[\{\rho_i\}] =
  \mathcal{F}[\{\rho_i\}]
  + \sum_{i=1}^\nu \int d^d \vec{r} \rho_i(\vec{r}) (\phi(\vec{r}) - \mu_i)
\end{equation}

\begin{equation}
  \left.
  \frac{\delta \Omega[\{\rho_i\}]}{\delta \rho_i(\vec{r})}
  \right|_{\{\rho_i(\vec{r}) = \rho_i^0(\vec{r})\}}
  = 0.
\end{equation}

Split into ideal and excess parts
\begin{equation}
  \mathcal{F}[\{\rho_i\}] =
  \mathcal{F}_{id}[\{\rho_i\}] + \mathcal{F}_{ex}[\{\rho_i\}]
\end{equation}
where the ideal part is
\begin{equation}
  \beta \mathcal{F}_{id}[\{\rho_i\}] =
  \sum_{i=1}^\nu \int d^d \vec{r} \rho_i(\vec{r})
  (\ln{\Lambda_i^d \rho_i(\vec{r})} - 1)
\end{equation}

\begin{itemize}
\item Contrast mechanical problem (e.g.\ simulations) with inverse problem
\item Summarise successes
\end{itemize}

\subsubsection{Heterogeneous approaches to the homogeneous liquid}
\paragraph{Potential distribution theorem}
\paragraph{Scaled particle theory}
