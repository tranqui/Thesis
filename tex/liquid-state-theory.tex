\section{Liquid state theory}
A summary of results given in standard texts (notably references \cite{Hansen2013} and \cite{Santos2016}) that will be used throughout the thesis.

\subsection{Statistical mechanical preliminaries}
\subsubsection{Interaction potentials}
\subsubsection{Simple liquids}
\subsubsection{Hard sphere model system}

\subsubsection{Grand canonical ensemble}
\paragraph{Averages}
The partition function and free energy
\paragraph{Correlations}
Particle densities $\rho^{(n)}$ and distribution functions $g^{(n)}$.

\subsection{Homogeneous liquid state theory}

\subsubsection{Virial expansion of the equation of state}

\paragraph{Distribution function form}
Pressure in terms of pair distribution function $g(r)$.

For pair potentials:
\begin{equation}
  \frac{\beta P}{\rho} =
  1 - \frac{2 \pi \beta \rho}{3}
  \int_0^\infty v'(r) g(r) r^3 \, dr
\end{equation}
Generalisation to many-body potentials requires higher order distribution functions.
Difficulty of discontinuities overcome by introducing cavity function
\begin{equation}
  y(r) = e^{\beta v(r)} g(r)
\end{equation}
which is continuous leading to
\begin{equation}
  \frac{\beta P}{\rho} =
  1 + \frac{2 \pi \beta \rho}{3}
  \int_0^\infty e'(r) g(r) r^3 \, dr,
\end{equation}
where Boltzmann factor of pair potential is
\begin{equation}
  e(r) = e^{\beta v(r)}.
\end{equation}

\paragraph{Diagrammatic form}
Low density expansion of pressure in series with virial coefficients.

\begin{equation}
  \frac{\beta P}{\rho} =
  1 + \sum_{i=2}^\infty B_i(T) \rho^{i-1}
\end{equation}
where $B_i$ are the virial coefficients.

Generalisation to a binary mixture: \cite{Hansen-Goos2014}
\begin{equation}
  \Phi = \sum_{n=2}^\infty \sum_{j=0}^{n}
  \frac{1}{n-1} {n \choose j} B_n^{[j]} \rho_1^{n-j} \rho_2^j
\end{equation}

\paragraph{Empirical Carnahan-Starling equation of state for hard spheres}

The excess free energy is determined from the equation of state by
\begin{equation}
  \frac{\beta F^{ex}}{N}
  = \int_0^\eta \left( \frac{\beta p}{\rho} - 1 \right) \, \frac{d\eta'}{\eta'},
\end{equation}
giving the excess chemical potential from the thermodynamic relation
\begin{equation}\label{eq:chemical-potential}
  \beta \mu^{ex}[p]
  = \beta \left( \frac{\partial F^{ex}}{\partial N} \right)_{V,T}
  = \left( \frac{\beta p}{\rho} - 1 \right)
  + \int_0^\eta \left( \frac{\beta p}{\rho} - 1 \right) \, \frac{d\eta'}{\eta'}.
\end{equation}

The Carnahan-Starling equation of state approximates the pressure for hard spheres as \cite{Carnahan1969}
\begin{equation}\label{eq:cs-pressure}
  \frac{\beta p_{cs}}{\rho} = \frac{1 + \eta + \eta^2 - \eta^3}{(1-\eta)^3},
\end{equation}
which gives the excess chemical potential using \eqref{eq:chemical-potential} as
\begin{equation}\label{eq:cs-mu}
  \beta \mu_{cs}^{ex} = \frac{8\eta - 9\eta^2 + 3\eta^3}{(1-\eta)^3}.
\end{equation}

\subsubsection{Free energy from distribution functions}
\paragraph{Distribution function theories}
\paragraph{Kirkwood superposition approximation}
\paragraph{Distribution functions from direct correlation functions: Ornstein-Zernike equation}

\subsubsection{Beyond hard spheres: perturbation theory and the mean field approximation}

\subsection{Inhomogeneous liquid state theory}

\subsubsection{Solvation physics}
\subsubsection{Density functional theory (DFT)}

\begin{itemize}
\item Contrast mechanical problem (e.g.\ simulations) with inverse problem
\item Summarise successes
\end{itemize}

\paragraph{Fundamental measure theory (FMT)}
\begin{itemize}
\item $d+1$ weight functions
\item Exact decomposition of Mayer-f bond
\end{itemize}

\subsubsection{Heterogeneous approaches to the homogeneous liquid}
\paragraph{Potential distribution theorem}
\paragraph{Scaled particle theory}
