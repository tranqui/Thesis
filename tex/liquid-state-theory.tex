\section{Statistical physics of fluids}

%% \subsection{Notes}

%% Here we talk in general terms about descriptions of the liquid state.
%% Broadly speaking, in its historical development approaches can be placed inside one of two categories.
%% Namely, theories involving
%% \begin{enumerate}
%%   \item Local geometric approximations capturing the short range interactions (free volume theory/cell theory, scaled particle theory), and
%%   \item Integral equations (Ornstein-Zernike closures, density functional theory) which properly treat the long range correlations.
%% \end{enumerate}
%% These two approaches are not mutually exclusive, and hybrid theories can improve on.
%% For instance, fundamental measure theory (FMT) involves the synthesis of integral geometry with the formalism of density functional theory which involves minimising a functional (i.e.\ an integral equation).

%% Classical theory of phase transitions (Landau).
%% Van der waals theory.
%% How does the transition occur?
%% Metastability leads into kinetics.

%% Kinetics vs thermodynamics.
%% Thermodynamic driving force vs activation barrier.

%% Relaxation behaviour controlled by activation barrier.
%% A thermal fluctuation which takes the system over the barrier%
%% \marginfootnote{These fluctuations are conventionally called \emph{instantons} as they spontaneously appear and vanish just like virtual particles in fundamental physics.
%%   The name for this relatively straightforward phenomenon is thus a reference to a much more counterintuitive and bizarre phenomenon, because physicists are good at making helpful analogies.}
%% occurs with rate $\exp{-\beta \Delta U}$ \cite{Langer}.

%% Liquids:
%% Free volume theory.
%% Early curvature corrections.
%% Free volume theory depends on the free volume (duh).
%% This is an example of an integral geometric theory.
%% Morphological

\subsection{Statistical mechanics}

In this section we \emph{briefly} introduce the statistical ensembles used throughout the rest of the thesis.
These emerge by considering typical fluctuations of thermodynamic quantities for a subsystem within a macroscopic system called the \emph{ensemble}; the properties of this larger system define average quanties of the subsystem \cite{Landau2008}.
Alternatively, from a Bayesian perspective the probability distributions given below emerge from maximising the entropy%
\marginfootnote{The entropy represents a thermodynamic quantity in the former picture, whereas it represents our own \emph{uncertainty} about the system in the latter.}
subject to the constraint of average energy and (optionally) the average particle number \cite{JaynesPR1957,JaynesPR1957a}.

A $d$-dimensional system of $N$ particles consists of $\vec{r}^N = \{\vec{r}_1, \cdots, \vec{r}_N\} \in \mathbb{R}^{dN}$ coordinates and $\vec{p}^N = \{\vec{p}_1, \cdots, \vec{p}_N\} \in \mathbb{R}^{dN}$ momenta.
The classical Hamiltonian can be decomposed into kinetic and potential terms as in
\begin{equation}
  \mathcal{H}_N(\vec{r}^N, \vec{p}^N)
  =
  K_N(\vec{p}^N) + U_N(\vec{r}^N)
\end{equation}
in the absence of an external field.
Further, we constrain the coordinates inside the volume $V$%
\marginfootnote{We use the usual physicist abuse of notation where $V$ refers to both a region in space $V \subset \mathbb{R}^d$, and also the volume of this space $V = V_d(V)$.}.
The \emph{canonical} ensemble describes an equilibrium system at constant temperature $T$ with probability measure
\begin{equation}
  f^{(N)}(\vec{r}^N, \vec{p}^N) \propto e^{-\beta \mathcal{H}_N}
\end{equation}
where $\beta = (k_B T)^{-1}$ with Boltzmann constant $k_B$ and temperature $T$.
The proportionality constant ensures the probability distribution is properly normalised, leading to the canonical partition function
\begin{equation}
  Q_N
  =
  \int_{\mathbb{R}^{dN}} \int_{V^N}
  e^{-\beta\mathcal{H}_N}
  d\vec{r}^N d\vec{p}^N.
\end{equation}
Classically, the kinetic energy is simply
\begin{equation*}
  K_N(\vec{p}^N) = \sum_{i=1}^N \frac{|\vec{p}_i|^2}{2m_i}
\end{equation*}
which can be integrated leaving
\begin{equation}
  Q_N = \frac{Z_N}{\Lambda^{dN} N!}
\end{equation}
where $\Lambda$ is the thermal de Broglie wavelength, and the configurational integral is given by
\begin{equation}\label{eq:canonical-partition}
  Z_N
  =
  \int_{V^N}
  e^{-\beta U_N}
  d\vec{r}^N.
\end{equation}
Average constants with $N$ fixed are obtained through
\begin{equation*}\label{eq:canonical-average}
  \left< \cdots \right>_N
  =
  \frac{1}{Z_N}
  \int_{V^N} \left(\cdots\right) e^{-\beta U_N} d\vec{r}^N,
\end{equation*}
and the Helmholtz free energy is obtained through
\begin{equation*}
  \beta F = -\ln{Z_N}.
\end{equation*}

We will work almost exclusively in the \emph{grand canonical ensemble}, where particle number varies according to a chemical potential $\mu$, which is convenient for liquid state descriptions%
\marginfootnote{Notably the free energy is extensive without invoking Stirling's approximation for $N!$, making the thermodynamics properly self-consistent even with small system sizes.}.
The corresponding partition function features summation over $N$, as in
\begin{equation}\label{eq:grand-canonical-partition}
  \Xi
  =
  \sum_{N=0}^\infty \frac{z^N}{N!} Z_N
  =
  \sum_{N=0}^\infty \frac{z^N}{N!}
  \int_{V^N} e^{-\beta U_N} d\vec{r}^N,
\end{equation}
where the activity is $z = \exp{(\beta\mu)} / \Lambda^d$.
Accordingly, average quantities are found via
\begin{equation}\label{eq:grand-canonical-average}
  \left< \cdots \right>
  =
  \frac{1}{\Xi} \sum_{N=0}^\infty \frac{z^N}{N!}
  \int_{V^N} \left(\cdots\right) e^{-\beta U_N} d\vec{r}^N,
\end{equation}
and the corresponding free energy (or \emph{grand potential}) is obtained via
\begin{equation*}
  \beta \Omega = -\ln{\Xi}.
\end{equation*}
For a homogeneous system this reduces to the standard result
\begin{equation}\label{eq:homogeneous-grand-potential}
  \Omega_\mathrm{hom} = - p V.
\end{equation}
Thermodynamic quantities are easily calculated for the ideal gas, e.g.\
\begin{equation*}
  \beta\Omega = - \frac{e^{\beta\mu^\mathrm{id}}}{\Lambda^d} V.
\end{equation*}
Comparing the homogeneous result \eqref{eq:homogeneous-grand-potential} with the ideal gas law $\beta p = \rho$ gives the chemical potential of an ideal gas as
\begin{equation}\label{eq:ideal-chemical-potential}
  \beta \mu^\mathrm{id} = \ln{(\Lambda^d \rho)}.
\end{equation}
From the Legendre transform of the grand potential
\begin{equation}\label{eq:grand-potential-legendre-transform}
  \Omega = F - \mu N
\end{equation}
we obtain the free energy density of an ideal gas as
\begin{equation}\label{eq:ideal-free-energy-density}
  \frac{\beta F^\mathrm{id}}{V} = \rho (\ln{(\Lambda^d \rho)} - 1).
\end{equation}
Finally, for interacting systems the chemical potential and free energy are typically separated into \emph{ideal} and \emph{excess} parts, as in
\begin{align*}
  \beta \mu &= \beta \mu^\mathrm{id} + \beta \mu^\mathrm{ex} \\
  \beta F &= \beta F^\mathrm{id} + \beta F^\mathrm{ex}
\end{align*}
with the ideal contributions as expressed above.

\subsection{Liquid structure}

Interparticle interactions induce spatial structure in the liquid which are characterised by several (equivalent) hierarchies of correlation functions.
The most natural description of structure starts from the \emph{$n$-particle density}
\begin{equation}\label{eq:n-particle-density-pdf}
  \mathrm{Prob}\left[ \textit{any } n \textrm{ particles in volume } d\vec{r}^n \right]
  :=
  \rho^{(n)}(\vec{r}^n) \, d\vec{r}^n.
\end{equation}
This is formally obtained by integrating the full (configurational) probability distribution over the remaining degrees of freedom.
For the single-component system this yields~\cite{Hansen2013}
\begin{equation}\label{eq:n-particle-density}
  \rho^{(n)}(\vec{r}^n)
  =
  \frac{1}{\Xi}
  \sum_{N=n}^\infty \frac{z^N}{(N-n)!}
  \int_{V^N} e^{-\beta U_N} \, d\vec{r}^{(N-n)}.
\end{equation}
Importantly, the single-particle density is simply the equilibrium density profile i.e.\
\begin{equation}\label{eq:single-particle-density}
  \rho^{(1)}(\vec{r})
  =
  \langle \rho(\vec{r}) \rangle
  =
  \left\langle \sum_{i=1}^N \delta(\vec{r} - \vec{r}_i) \right\rangle.
\end{equation}
The $n$-particle density is an intuitive descriptor for liquid structure because it generalises the probability density function for a closed system, i.e.\
\begin{equation*}
  \mathrm{Prob}\left[ N \textrm{ particles in volume } d\vec{r}^n \right]
  :=
  \frac{e^{-\beta U_N}}{Z_N} \, d\vec{r}^N,
\end{equation*}
to a subset of particles in an open system.
$\rho^{(n)}$ thus provides the correct procedure for coarse-graining onto selected degrees of freedom within a bulk system.
The analogy with canonical ensemble fails in that $\rho^{(n)}$ is unnormalised so it is not strictly a probability density function; integrating \eqref{eq:n-particle-density} over the remaining degrees of freedom yields%
\marginfootnote{In keeping with the analogy to canonical ensemble we treat this integral as a partition function, and so account for indistinguishability of the $n$ particles by dividing through by $n!$.}
\begin{equation}\label{eq:n-particle-density-normalisation}
  \frac{1}{n!}
  \int_{V^n} \rho^{(n)}(\vec{r}^n) \, d\vec{r}^n
  =
  \left\langle \frac{N!}{n! (N-n)!} \right\rangle
\end{equation}
i.e.\ the average binomial coefficient.
The $n$-particle density scales proportionally to $\rho^n$ so it is usual to remove this by defining the \emph{$n$-particle distribution function} as
\begin{equation}\label{eq:n-particle-distribution}
  g^{(n)}(\vec{r}^n)
  :=
  \frac{\rho^{(n)}(\vec{r}^n)}{\rho^n},
\end{equation}
which provides our first (and primary) hierarchy of correlation functions.

Physically, particles become decorrelated when they are separated by macroscopic distances%
\marginfootnote{This limit behaviour is only valid for `normal' liquid behaviour far from the critical point where the correlation length diverges.}.
This property manifests in the distribution functions via a \emph{product property} where \cite{UhlenbeckJMP1963}
\begin{equation*}
  g^{(n)}(\vec{r}^n)
  \simeq
  g^{(s)}(\vec{r}^s) \, g^{(n-s)}(\vec{r}^{n-s})
\end{equation*}
in the limit where the $s$ particles become macroscopically separated from the remaining $(n-s)$ particles.
This property causes the distribution functions to decay to their ideal gas value $g^{(n)}(\vec{r}^n) \to 1$ in the limit of infinite separations between all particles.
Moreover, the product property suggests that there is a great deal of redundancy inside the distribution functions; in certain applications it is convenient to introduce an additional hierarchy of correlation functions which only capture the excess correlations.
If we imagine the normalisation of the distribution functions $g^{(n)}$ as \emph{moments} of an unspecified probability distribution, then we can formally imagine a dual set of correlation functions $h^{(n)}$ which generate the \emph{cumulants}.
Formally, this relationship is expressed \cite{Santos2016}
\begin{equation}\label{eq:correlation-moment-generating-function}
  1
  + \sum_{n=1}^\infty \frac{\epsilon^n}{n!}
  \int_{V^n} g^{(n)}(\vec{r}^n) \, d\vec{r}^n
  =
  \exp{
    \left(
    \sum_{n=1}^\infty \frac{\epsilon^n}{n!}
    \int_{V^n} h^{(n)}(\vec{r}^n) \, d\vec{r}^n
    \right)
  },
\end{equation}
with $\epsilon$ as a formal expansion parameter of the moment generating function.
In addition, we require that these new functions share the same symmetries as $g^{(n)}$ e.g.\ permutation invariance of the arguments.
These conditions specify a new hierarchy: the \emph{cluster correlation functions}%
\marginfootnote{These are so-named because they possess a \emph{cluster property} where they decay to zero in the limit where any particles become macroscopically separated \cite{UhlenbeckJMP1963}.
  This feature directly emerges from, and is dual to, the product property for $g^{(n)}$.}
where the first few terms are given by \cite{UhlenbeckJMP1963}
\begin{subequations}
  \begin{align}
    h^{(1)}(\vec{r})
    =& \,
    g^{(1)}(\vec{r}) = 1,
    \\
    h^{(2)}(\vec{r}^2)
    =& \,
    g^{(2)}(\vec{r}_1, \vec{r}_2)
    - g^{(1)}(\vec{r}_1) g^{(1)}(\vec{r}_2),
    \label{eq:pair-cluster-correlation-function}
    \\
    h^{(3)}(\vec{r}^3)
    =& \,
    g^{(3)}(\vec{r}_1, \vec{r}_2, \vec{r}_3)
    - g^{(2)}(\vec{r}_1, \vec{r}_2) g^{(1)}(\vec{r}_3)
    - g^{(2)}(\vec{r}_1, \vec{r}_3) g^{(1)}(\vec{r}_2)
    \nonumber \\ & \,
    - g^{(2)}(\vec{r}_2, \vec{r}_3) g^{(1)}(\vec{r}_1)
    + g^{(1)}(\vec{r}_1) g^{(1)}(\vec{r}_2) g^{(1)}(\vec{r}_3),
  \end{align}
\end{subequations}
The pair cluster correlation function%
\marginfootnote{This is often called simply the \emph{total correlation function}, especially in the context of integral equation theories (cf.\ section \ref{sec:oz-equation}).}
$h^{(2)}(\vec{r}_1, \vec{r}_2) = g^{(2)}(\vec{r}_1, \vec{r}_2) - 1$ is the main function we will use from this hierarchy.

An important response function for liquid structure is the isothermal compressibility
\begin{equation*}\label{eq:isothermal-compressibility}
  \kappa_T
  :=
  %% - \frac{1}{V}
  %% \left( \frac{\partial V}{\partial p} \right)_{N,T}.
  \frac{1}{\rho}
  \left( \frac{\partial \rho}{\partial p} \right)_{V,T}.
\end{equation*}
Using standard thermodynamic manipulations we can obtain the equivalent expression \begin{equation*}
  \kappa_T
  =
  \frac{1}{\rho^2}
  \left( \frac{\partial \rho}{\partial \mu} \right)_{V,T}
\end{equation*}
or defining the dimensionless \emph{isothermal susceptibility} as \begin{equation*}\label{eq:isothermal-susceptibility}
  \chi_T
  :=
  \rho k_B T \kappa_T
  =
  \frac{1}{\rho}
  \left( \frac{\partial \rho}{\partial (\beta \mu)} \right)_{V,T}.
\end{equation*}
It is straightforward to evaluate this through the grand canonical average \eqref{eq:grand-canonical-average} of density $\rho = \langle N \rangle / V$, obtaining
\begin{equation}
  \chi_T
  =
  \frac{ \langle N^2 \rangle - \langle N \rangle^2 }{\langle N \rangle}.
\end{equation}
From the normalisation of $\rho^{(2)}$ \eqref{eq:n-particle-density-normalisation} we find
\begin{equation}\label{eq:compressibility-h2}
  \begin{split}
    \chi_T
    &=
    1
    + \frac{1}{\langle N \rangle}
    \int_{V^2} \rho^{(2)}(\vec{r}_1, \vec{r}_2) \, d\vec{r}_1 d\vec{r}_2
    - \langle N \rangle
    \\ &=
    1
    + \rho \int_V h^{(2)}(\vec{r}) \, d\vec{r}
  \end{split}
\end{equation}
where the latter step is valid for the homogeneous liquid where $g^{(2)}(\vec{r}_1, \vec{r}_2) = g^{(2)}(\vec{r}_2 - \vec{r}_1)$ and we used the pair cluster correlation function \eqref{eq:pair-cluster-correlation-function}.

The above are the real space structural descriptors, but there are an equivalent set of Fourier space measures.
The static structure factor
Note that the static structure factor is defined from the Fourier transform of the pair distribution function, i.e.\
\begin{equation}\label{eq:static-structure-factor}
  \begin{split}
    S^{(2)}(\vec{k})
    :=&
    1 + \rho \tilde{g}^{(2)}(\vec{k})
    \\ =&
    1 + \rho \tilde{h}^{(2)}(\vec{k}) + \rho \delta(\vec{k})
  \end{split}
\end{equation}
where the tilde over a function denotes its Fourier transform.
In terms of the structure factor \eqref{eq:compressibility-h2} is written succinctly as%
\marginfootnote{The Dirac delta function at the origin in $S^{(2)}(\vec{k})$ is often left out in order to regularise the function, in which case the right-hand side can be written more simply as $S^{(2)}(0)$.}
\begin{equation}
  \chi_T = \lim_{\vec{k} \to 0} S^{(2)}(\vec{k}).
\end{equation}

Kirkwood superposition and convolution approximations

\subsection{Correlation functions I}

\todo{Finish this section.}
Density-density correlation functions
\begin{equation}\label{eq:density-density-correlations}
  H^{(n)}(\vec{r}^n)
  =
  \left\langle
  \prod_{i=1}^n
  \Big[ \rho(\vec{r}_i) - \big\langle\rho(\vec{r}_i)\big\rangle \Big]
  \right\rangle
  \qquad \forall \; n \ge 2
\end{equation}
e.g.\
\begin{equation}
  \begin{aligned}
    H^{(2)}(\vec{r}_1, \vec{r}_2) &=
    \left\langle
    \left[ \rho(\vec{r}_1) - \rho^{(1)}(\vec{r}_1) \right]
    \left[ \rho(\vec{r}_2) - \rho^{(1)}(\vec{r}_2) \right]
    \right\rangle \\
    &=
    \big\langle \rho(\vec{r}_1) \rho(\vec{r}_2) \big\rangle -
    \rho^{(1)}(\vec{r}_1) \rho^{(1)}(\vec{r}_2) \\
    &=
    \rho^{(2)}(\vec{r}_1, \vec{r}_2) +
    \rho^{(1)}(\vec{r}_1) \delta(\vec{r}_1 - \vec{r}_2) -
    \rho^{(1)}(\vec{r}_1) \rho^{(1)}(\vec{r}_2) \\
    &=
    \rho^{(1)}(\vec{r}_1) \rho^{(1)}(\vec{r}_2) h^{(2)}(\vec{r}_1, \vec{r}_2)
    +
    \rho^{(1)}(\vec{r}_1) \delta(\vec{r}_1 - \vec{r}_2)
  \end{aligned}
\end{equation}

Higher moments of the instantaneous density are obtained through
\begin{equation}
  \left\langle \prod_{i=1}^n \rho(\vec{r}_1) \right\rangle
  =
  \sum_{m=1}^n {n \choose m}
  \rho^{(m)}(\vec{r}^m)
  \prod_{j=m+1}^n \delta(\vec{r}_j - \vec{r}_1)
\end{equation}

\subsection{?}

\todo{Finish this section}

\begin{equation}
  \chi_T \rho
  \left( \frac{\partial \rho^{(n)}}{\partial \rho} \right)_{V,T}
  =
  (n - \rho V) \rho^{(n)}(\vec{r}^n)
  + \int \rho^{(n+1)}(\vec{r}^{n+1}) \, d\vec{r}_{n+1}
\end{equation}
\begin{equation}
  \left( \frac{\partial g^{(n)}(\vec{r}^n)}{\partial \rho} \right)_{V,T}
  =
  \frac{1}{\rho^n}
  \left( \frac{\partial \rho^{(n)}(\vec{r}^n)}{\partial \rho} \right)_{V,T}
  - \frac{n}{\rho} g^{(n)}(\vec{r}^n)
\end{equation}
\begin{equation}
  \begin{split}
    \frac{\chi_T}{\rho^n}
    \left( \frac{\partial \rho^{(n)}}{\partial \rho} \right)_{V,T}
    &=
    \left(\frac{n}{\rho} - V\right) g^{(n)}(\vec{r}^n)
    + \int g^{(n+1)}(\vec{r}^{n+1}) \, d\vec{r}_{n+1}
    \\ &=
    \chi_T \left( \frac{\partial g^{(n)}}{\partial \rho} \right)_{V,T}
    + \chi_T \frac{n}{\rho} g^{(n)}
  \end{split}
\end{equation}
\begin{equation}
  \chi_T \left( \frac{\partial g^{(n)}}{\partial \rho} \right)_{V,T}
  =
  \left(
  \frac{n (1 - \chi_T)}{\rho}
  - V
  \right) g^{(n)}(\vec{r}^n)
  + \int g^{(n+1)}(\vec{r}^{n+1}) \, d\vec{r}_{n+1}
\end{equation}
\begin{equation}
  \begin{split}
    \chi_T \left( \frac{\partial g^{(2)}}{\partial \rho} \right)_{V,T}
    &=
    \left(
    \frac{2 (1 - \chi_T)}{\rho}
    - V
    \right) g^{(2)}(\vec{r}^2)
    + \int g^{(3)}(\vec{r}^3) \, d\vec{r}_3
    \\
    \left( \frac{\partial g^{(2)}}{\partial \rho} \right)_{V,T}
    &=
    \frac{2 (1 - \chi_T)}{\rho \chi_T}
    g^{(2)}(\vec{r}^2)
    +
    \int g^{(3)}(\vec{r}^3) - \rho g^{(2)}(\vec{r}^2) \, d\vec{r}_3
  \end{split}
\end{equation}

\subsection{Thermodynamic routes to the free energy}

Goal: obtain the free energy and we have everything.
Ordinarily this is achieved by obtaining an equation of state for the pressure $p = p(\rho)$, giving the free energy implicitly through the thermodynamic relation
\begin{equation}\label{eq:pressure-relation-1}
  p
  =
  - \left( \frac{\partial F}{\partial V} \right)_{N,T},
\end{equation}
although a state equation for any other thermodynamic observable would sufficiently determine $F$.

The first option for determining the free energy is through the compressibility, from the thermodynamic relation
\begin{equation}
  \frac{1}{\chi_T}
  %% =
  %% - V \left( \frac{\partial p}{\partial V} \right)_{N,T}
  =
  \left( \frac{\partial \beta p}{\partial \rho} \right)_{V,T}
  %% =
  %% V \left( \frac{\partial^2 F}{\partial V^2} \right)_{N,T}
  %% =
  %% \frac{\rho^2}{V}
  %% \left( \frac{\partial^2 F}{\partial \rho^2} \right)_{N,T}
\end{equation}
Integrating this relation over the density and making use of the isothermal compressibility identity for a uniform system \eqref{eq:compressibility-h2} gives
\begin{equation}\label{eq:compressibility-route}
  \beta p
  %% =
  %% \int_0^\rho \frac{1}{\rho' k_B T \kappa_T} d\rho'
  =
  \int_0^\rho \frac{1}{\chi_T} d\rho'
  %% =
  %% \int_0^\rho \frac{1}{1 + \rho \int h^{(2)}(\vec{r}) \, d\vec{r}} d\rho',
  =
  \int_0^\rho \lim_{\vec{k} \to 0} \frac{1}{S^{(2)}(\vec{k})} d\rho',
\end{equation}
i.e.\ the \emph{compressibility route} to the pressure.

Another option evaluates the pressure directly \eqref{eq:pressure-relation-1} from the partition function.
In terms of the canonical partition function this gives
\begin{equation}\label{eq:pressure-relation-2}
  \beta p
  =
  \left( \frac{\partial (\ln{Z_N})}{\partial V} \right)_{N,T}.
\end{equation}
We consider what happens during a volume change $V \to \alpha^d V$ emerging from the affine rescaling $\vec{r} \to \alpha \vec{r}$, so that the configurational integral \eqref{eq:canonical-partition} becomes
\begin{equation}\label{eq:inflated-canonical-partition}
  Z_N(\alpha^d V)
  =
  \int_{\alpha V^N} e^{-\beta U_N(\vec{r}^N)} d\vec{r}^N
  =
  \alpha^{dN}
  \int_{V^N} e^{-\beta U_N(\alpha \vec{r}^N)} d\vec{r}^N
\end{equation}
Using the identity
\begin{equation*}
  \frac{\partial f(xy)}{\partial y}
  =
  \frac{x}{y} \frac{\partial f(xy)}{\partial x},
\end{equation*}
we can write
\begin{equation}
  \frac{\partial (\ln{Z_N(\alpha^d V)})}{\partial V}
  =
  \frac{\alpha}{d V}
  \frac{\partial (\ln{Z_N(\alpha^d V)})}{\partial \alpha}.
\end{equation}
This trick allows the pressure relation \eqref{eq:pressure-relation-2} to be reexpressed as
\begin{equation*}
  \frac{\beta p}{\rho}
  =
  \frac{1}{d N}
  \left.
  \frac{\partial (\ln{Z_N(\alpha^d V)})}{\partial \alpha}
  \right|_{\alpha = 1}.
\end{equation*}
The derivative of \eqref{eq:inflated-canonical-partition} with respect to $\alpha$ can be calculated explicitly as
\begin{equation*}
  \begin{split}
    \frac{\partial Z_N(\alpha^d V)}{\partial \alpha}
    &=
    %% \frac{\partial}{\partial \alpha}
    %% \left(
    %% \int_{V^N} e^{-\beta U_N(\alpha^d \vec{r}^N)} d\vec{r}^N
    %% \right)
    %% \\ &=
    \frac{dN}{\alpha} Z_N
    +
    \alpha^{dN}
    \int_{V^N}
    \frac{\partial}{\partial \alpha}
    \left( e^{-\beta U_N(\alpha \vec{r}^N)} \right)
    d\vec{r}^N
    \\ &=
    \frac{dN}{\alpha} Z_N
    -
    \alpha^{dN}
    %\sum_{i,j}
    \int_{V^N}
    \frac{\partial \beta U_N(\alpha \vec{r}^N)}{\partial \alpha}
    e^{-\beta U_N}
    d\vec{r}^N,
  \end{split}
\end{equation*}
giving the final result
\begin{equation}\label{eq:virial-route-pressure}
  \begin{split}
    \frac{\beta p}{\rho}
    &=
    1
    -
    \frac{1}{\rho d Z_N}
    \int_{V^N}
    \left.
    \frac{\partial \beta U_N(\alpha \vec{r}^N)}{\partial \alpha}
    \right|_{\alpha = 1}
    e^{-\beta U_N}
    d\vec{r}^N
    \\ &=
    1
    -
    \frac{1}{\rho d}
    \left\langle
    \left.
    \frac{\partial \beta U_N(\alpha \vec{r}^N)}{\partial \alpha}
    \right|_{\alpha = 1}
    \right\rangle,
  \end{split}
\end{equation}
which is known as the \emph{virial route}%
\marginfootnote{This is so-named because historically it was derived through the virial theorem.
Despite the similar name, this approach has no relation to the virial series which will be introduced in section \ref{sec:virial-series}.}
to the pressure.
In the latter step we replaced the canonical average with the grand-canonical average \eqref{eq:grand-canonical-average} from ensemble equivalence in the thermodynamic limit.

There are other routes involving different observables (e.g.\ the potential energy or the chemical potential \cite{Santos2016}) to obtain the equation of state from the correlation functions; however, we will not discuss them as we will only use the virial route in the results chapters.
The degree of self-consistency between different routes can act as a proxy for the accuracy of an approximate theory.

\begin{tcolorbox}[title=Contact theorem for hard spheres]
For a single-component system interacting through a spherically symmetric pair potential $u(r)$, the virial route \eqref{eq:virial-route-pressure} yields
\begin{equation}\label{eq:virial-route-pressure-uniform}
  \frac{\beta p}{\rho}
  =
  1
  -
  \frac{\rho}{2 d}
  \int_V
  r g^{(2)}(r) \frac{d \beta u}{d r} \, d\vec{r},
\end{equation}
using the definition of the 2-particle distribution function \eqref{eq:n-particle-distribution} as the average over the remaining degrees of freedom.
In the case of hard spheres $u'(r)$ is not well-defined because the pair potential is singular, however the cavity function
\begin{equation*}\label{eq:cavity-function}
  y^{(2)}(r) = g^{(2)}(r) e^{\beta u(r)}
\end{equation*}
is continuous (see e.g.\ Refs.\ \cite{Hansen2013,Santos2016}) even in cases where the pair potential is not.
In terms of the cavity function, the virial route \eqref{eq:virial-route-pressure-uniform} becomes
\begin{equation*}\label{eq:virial-route-pressure-cavity}
  \frac{\beta p}{\rho}
  =
  1
  +
  \frac{\rho}{2 d}
  \int_V
  r y^{(2)}(r) \frac{d f}{dr}
  \, d\vec{r},
\end{equation*}
which for $d$-dimensional hard spheres of diameter $\sigma$ yields the \emph{contact theorem}:
\begin{equation}
  \frac{\beta p}{\rho}
  =
  1
  +
  \frac{\rho \omega_d \sigma^3}{2}
  y^{(2)}(\sigma).
  %% \\ &=
  %% 1
  %% +
  %% \frac{2 \pi \rho \sigma^3}{3}
  %% y^{(2)}(\sigma)
\end{equation}
\end{tcolorbox}

Derivatives of free energy reduce to $g^{(2)}$, emblematic of the conventional routes.
Absolute free energies require higher order correlation functions to appear.
\todo{Talk about this. Also: entropy route.}

The pair distribution function $g^{(2)}$ appears because the specified system interacts via a pair potential; we could reasonably expect the generalisation to an $n$-body interaction potential to give the equation of state in terms of the $n$-body distribution function $g^{(n)}$.

\subsection{Virial series}
\label{sec:virial-series}

Low density expansion of pressure in series with virial coefficients.
\todo{Finish this section}

\begin{equation}
  \frac{\beta p}{\rho} =
  1 + \sum_{i=2}^\infty B_i(T) \rho^{i-1}
\end{equation}
where $B_i$ are the virial coefficients.

Generalisation to a binary mixture \todo{Where does this come from? What do the diagrams look like? Are they simpler than the distribution function diagrams?}: \cite{Hansen-Goos2014}
\begin{equation}\label{eq:virial-series-binary}
  \Phi = \sum_{n=2}^\infty \sum_{j=0}^{n}
  \frac{1}{n-1} {n \choose j} B_n^{[j]} \rho_1^{n-j} \rho_2^j
\end{equation}
