%\restoregeometry

\chapter*{\vspace{-1em} Acknowledgements}
\addcontentsline{toc}{chapter}{Acknowledgements}

First and foremost, I want to thank my supervisor, Paddy, for being the reason that any of this happened at all, for being a crucial check on my natural impulse to be \emph{too} focused on details and for the fact that I now see beautiful (pink) icosahedra every time I close my eyes.
I thank him for giving me the freedom to develop my ideas, and for his remarkable degree of patience when that led to me studying six particles for four years.

It has now been several years since Paddy gave me the `humble' task of solving the glass transition once and for all.
Having failed spectacularly at this, I hope we can collectively agree that the real glass transition was the friends we made along the way.
In particular, all of the cave-dwellers (and countless mice) of G.39 who survived the water features, two fumigations and Paddy.
The forces of darkness may have stolen our toaster, but they never broke our spirit%
\footnote{After all, we had the science to do that.}.
I will treasure these memories, and your friendship, until the end of my days.
%Honorable mentions to the contriband toaster and the electric fence of Fowey.
%Together we braved the near-total sunlight deprivation, more mice than I care to remember and an incident with an electric fence.
%Special thanks go out to the electic fence which our trip to cornwall.

I owe a great debt to Francesco Turci for dedicating far more time than I deserve to discuss my work.
I honestly don't think I could have done this without his consistent, valuable input throughout.

I would like to express my gratitude to our collaborator Roland Roth, for founding the Bristol's White Bear institute, conjuring up the initial idea for my project with Paddy and providing guidance along the way.
Additionally, I am deeply grateful to our collaborators in chemistry for what turned out to be a fun and fruitful project: Flo Gregson, Rachel Miles and Jonathan Reid.
I especially thank Flo for doing the experiments which made this collaboration possible, and for providing valuable feedback on my initial attempts at describing them in chapter \ref{chapter:aerosols}.

Special thanks go out to Bob Evans, who showed interest in my work from the very beginning, for the many stimulating discussions about liquid state theory over the years.
I must also thank him for the constructive feedback he gave on an early draft of the paper underlying parts of chapters \ref{chapter:morphometric-framework} and \ref{chapter:morphometric-applications}.
I recall he described me as ``completely mad'' afterwards, which I \emph{think} was a compliment, so I must additionally thank him for inadvertently suggesting an `insanity defence' to rely upon in my viva.

I am very grateful to the organisers and lecturers at the Boulder School in Condensed Matter and Materials Physics 2017.
This school influenced my development as a statistical physicist in a myriad of ways.
Notably, I found myself referring back to the phenomenological lectures by Gilles Tarjus while writing chapter \ref{chapter:glass}.

Financially, I acknowledge the European Research Council under the FP7 / ERC Grant Agreement No.\ 617266 ``NANOPRS'' and Paddy for finding ways to support me after this ran out.

I have learned that research is an almost constant uphill struggle, with the occasional false summit.
My most heartfelt appreciation goes out to all my friends and colleagues who made the struggle worthwhile.
I would particularly like to thank Kate Oliver, Nick Wood and Max Meissner for keeping me sane in difficult times, and Kirsty Wynne for teaching me perspective and the meaning of a `false summit'.
Last, and certainly not least, I owe my deepest gratitude to all the close friends and family members who were there for me every step along the way.

%\newgeometry{margin=1in}
