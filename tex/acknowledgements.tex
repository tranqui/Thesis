%\restoregeometry

\chapter*{Acknowledgements}

It has now been 4 years since Paddy gave me the `humble' task of solving the glass transition once and for all.
Having failed spectacularly at this, I hope we can collectively agree that the real glass transition was the friends we made along the way.
In particular, all of the water-dwelling mammals of G.39 who survived toaster theft, near-total sunlight deprivation, icosahedral fever dreams, mass exodus into promised lands, countless water features, two fumigations, an incident with an electric fence, more mice than I care to remember and Paddy.
I will treasure these memories, and your friendship, until the end of my days.

I must thank Paddy for giving me the freedom to develop my ideas, and for being remarkably patient when that led to me studying 6 particles for 2 years.
I am grateful that he was overjoyed when I eventually doubled it.

Francesco, invaluable.
For giving me far more time than I deserve to discuss my 

My collaborator Roland Roth, for founding the Bristol's White Bear institute and conjuring up the initial idea for my project with Paddy.
Our collaborators in chemistry for a fun and fruitful project: Flo Gregson, Rachel Miles and Jonathan Reid.
I thank Flo Gregson in particular for doing the experiments which made this collaboration possible, and for providing valuable feedback on my initial attempts at describing them in chapter \ref{chapter:aerosols}.

Bob Evans, from the beginning showed interest in my work, and for giving constructive feedback on an early draft of paper which forms parts of chapters \ref{chapter:morphometric-framework} and \ref{chapter:morphometric-applications}.
I recall he described me as ``completely mad'' afterwards, which I \emph{think} was a compliment, so I must additionally thank him for inadverently suggesting an `insanity defence' to rely upon in my viva.

Financially, I acknowledge the European Research Council under the FP7 / ERC Grant Agreement No.\ 617266 ``NANOPRS'' and Paddy for continuing to find ways to support me after my 3.5 years ran out.

I have learned that research is an almost constant uphill struggle, with the occasional false summit.
I want to thank my comrades who made the struggle worthwhile.
In particular I want to single out Kate Oliver, Nick Wood and Max Meissner for keeping me sane in difficult times, and Kirsty Wynne for teaching me perspective and the meaning of a `false summit'.

Last, and certainly not least, I thank my close friends and family.
You know who you are.
My mothers.

%``disagreeing with me almost every step along the way, except when I was defeated''

%\newgeometry{margin=1in}
