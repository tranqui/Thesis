%\restoregeometry

\chapter*{Acknowledgements}

It has now been 4 years since Paddy gave me the `humble' task of solving the glass transition once and for all.
Having failed spectacularly at this, I hope we can collectively agree that the real glass transition was the friends we made along the way.
In particular, all of the water-dwelling mammals of G.39 who survived toaster theft, near-total sunlight deprivation, icosahedral fever dreams, mass exodus into promised lands, countless water features, two fumigations, an incident with an electric fence, more mice than I care to remember and Paddy.
I will treasure these memories, and your friendship, until the end of my days.

I have learned that research is an almost constant uphill struggle, with the occasional false summit.
Members of the group, past and present.
I want to thank Kate Oliver, Nick Wood and Max Meissner for keeping me sane in difficult times, and Kirsty Wynne for teaching me perspective and the meaning of a `false summit'.

Francesco, invaluable.

Paddy: freedom to develop the ideas.

Bob Evans, from the beginning.
Upon reading the published material which forms parts of chapters \ref{chapter:morphometric-framework} and \ref{chapter:morphometric-applications}, Bob described me as ``completely mad'', which I \emph{think} was a compliment, so I must additionally thank him for inadverently suggesting an `insanity defence' to rely upon in my viva.

Roland Roth, a founding member of the Bristol's White Bear institute: conjured up the initial idea with Paddy, which I have added to along the way.
Our collaborators in chemistry for a fun and fruitful project: Flo Gregson, Rachel Miles and Jonathan Reid.
I thank Flo Gregson in particular for doing the experiments which made this collaboration possible, and for providing valuable feedback on my initial attempts at describing them in chapter \ref{chapter:aerosols}.

Boulder was influential.
I've had the pleasure to encounter Gilles Tarjus multiple times throughout these years, who was always warm, supportive and optimistic about my work in stark contrast to my own views; this.

I have failed at this, but hopefully in an entertaining way: I have stumbled upon a scheme to predict things about a macroscopic system from fewer particles than a person (traditionally) has fingers.

Financially, I acknowledge the European Research Council under the FP7 / ERC Grant Agreement No.\ 617266 ``NANOPRS'' and Paddy for continuing to find ways to support me after my 3.5 years ran out.

Last, and certainly not least, I thank my close friends and family.
You know who you are.
My mothers.
I have tried to keep the introduction minimally mathematical for them.

%\newgeometry{margin=1in}
