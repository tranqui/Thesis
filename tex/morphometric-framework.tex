\documentclass[12pt]{report}
\usepackage{preamble}
\setcounter{chapter}{2}

\begin{document}
\chapter{Morphological framework for many-body correlations}
Mostly theoretical but with some numerical experiments which motivate/justify developments to the theory.
The main body of numerical work is left to the following chapter.

\section{Formalism for many-body correlations}

We write the $n$-particle distribution function $g^{(n)}$ as
\begin{equation}\label{eq:n-density-pdf}
  \textrm{Probability}\left( \textit{any } n \textrm{ particles in volume } d\vec{r}^n \right)
  \equiv
  \rho^n g^{(n)}(\vec{r}^n) \, d\vec{r}^n.
\end{equation}
In the main text, we expressed $g^{(n)}$ in terms of a generalised potential of mean force $\phi^{(n)}$: the reversible work required to insert $n$ particles into the liquid.
We decomposed $\phi^{(n)}$ into a local (potential energy) and solvent (free energy) component.
Although this quantity is quite intuitive and could be determined heuristically, here we give a short proof that this decomposition is formally exact and arises quite naturally from the definition of the distribution function.

In the grand-canonical ensemble the $n$-particle density $\rho^{(n)}(\vec{r}^n) \equiv \rho^n g^{(n)}$ is determined by integrating over the remaining degrees of freedom~\cite{Hansen2013}
\begin{equation}
  \rho^{(n)}(\vec{r}^n)
  = \frac{1}{\Xi} \sum_{N=n}^\infty \frac{z^N}{(N-n)!} \int e^{-\beta U_N} \, d\vec{r}^{(N-n)},
\end{equation}
where the activity is written in terms of the thermal de Broglie wavelength $\Lambda$ as $z = \exp{\beta\mu} / \Lambda^d$.
Changing the summation limits $N \rightarrow N+n$ we obtain
\begin{equation}\label{eq:n-density}
\begin{aligned}
  \rho^{(n)}(\vec{r}^n)
  &= \frac{z^n}{\Xi} \sum_{N=0}^\infty \frac{z^N}{N!} \int e^{-\beta U_{N+n}} \, d\vec{r}^{N} \\
 & = z^n e^{-\beta U_n} \left< e^{-\beta U_{n \leftrightarrow N}} \right>
\end{aligned}
\end{equation}
where in the latter step we decomposed the total potential $U_{N+n}$ into purely local and solvent terms, i.e.\ $U_{N+n} = U_n + U_N + U_{n \leftrightarrow N}$, where $U_\alpha$ for $\alpha \in \{n,N\}$ indicates the internal interactions between particles in component $\alpha$.
The ``interspecies'' interactions are contained within $U_{n \leftrightarrow N}$ which acts as an external field for the solvent.
The angled brackets indicate ensemble averaging over all arrangements of the solvent, i.e.\
\begin{equation}
  \left< \cdots \right> =
  \frac{1}{\Xi} \sum_{N=0}^\infty \frac{z^N}{N!} \int \left(\cdots\right) e^{-\beta U_N} \, d\vec{r}^N,
\end{equation}
with partition function of the unperturbed system $\Xi \equiv e^{-\beta \Omega_{hom}}$, where $\Omega_{hom} = -p V$ is the usual homogeneous grand potential.
Thus, \eqref{eq:n-density} becomes
\begin{equation}
  \rho^{(n)}(\vec{r}^n)
  = z^n e^{-\beta (U_n + \Omega - \Omega_{hom})}.
\end{equation}
where $\Omega$ is the grand potential of the solvent in the presence of the $n$-particle inhomogeneity.
Splitting the chemical potential into its ideal and excess parts so that $\beta\mu = \ln{\Lambda^d \rho} + \beta\mu^{ex}$ gives
\begin{equation}
  \rho^{(n)}(\vec{r}^n)
  = \rho^n e^{-\beta (U_n + \Omega - \Omega_{hom} - n\mu^{ex})}.
\end{equation}
The $n$-particle distribution functions are then determined from~\cite{Hansen2013}
\begin{equation}\label{eq:distribution-functions}
  g^{(n)}(\vec{r}^n)
  \equiv \frac{\rho^{(n)}(\vec{r}^n)}{\rho^n}
  = e^{-\beta(U_n + \Delta\Omega - n\mu^{ex})}
\end{equation}
where $\Delta\Omega = \Omega - \Omega_{hom}$ giving the generalised potential of mean force stated in the main text.

The above is essentially the generalisation of the \emph{potential distribution theorem} \cite{Widom1963,Widom1982} to many-particles. See Ref.\ \cite{Rowlinson2002} and references therein for a detailed review of this approach.

\section{Morphological form of the potential of mean force}
Justification of assumptions: additivity, continuity and motion invariance
\subsection{Limitations known from DFT literature}
\subsection{As a generalisation of scaled particle theory}
And the limitations this implies.

\section{Worked examples where morphometric form can be exact}
Under certain conditions.
\subsection{Low density limit in arbitrary dimensions from lowest order terms in the virial expansion of the pressure}
\subsection{Arbitrary densities at large lengthscales}
\subsection{Hard rods (dimension d = 1) at all densities}
\subsubsection{Exact result from DFT}
\subsubsection{Morphometric result}
Explore where additivity, continuity and motion invariance apply.
\subsubsection{Implications for higher dimensions}

\section{Derivation of thermodynamic coefficients for hard spheres in $d = 3$}

\subsubsection{Scaled particle route}

We describe here in more detail the derivation of the first set of morphometric coefficients, equivalent to those given in \cite{Hansen-Goos2006} through fundamental measure theory (FMT).
The derivation sketched below avoids use of FMT, instead favouring a geometric formulation equivalent to the scaled particle approach of Reiss \emph{et al}.\ \cite{Reiss1959,Reiss1960}.
The standard scaled particle approach considers an expansion of the grand potential surrounding a spherical solute in powers of radii; here, we modify the \emph{ansatz} to use morphological measures instead, so that the resulting theory is more naturally extended to geometries of arbitrary shapes.
Additionally, we impose the Carnahan-Starling equation of state as an \emph{input} whereas the Percus-Yevick equation of state is an \textit{output} of standard scaled particle approaches.

Following the protocol of scaled particle theories, we consider the insertion of a hard ball of radius $R-\frac{\sigma}{2}$ into the liquid at the origin.
This choice of radius ensures that contact with the \emph{center} of solvent particles occurs at distance $R$ from the point of insertion i.e.\ $\rho(r) = 0$ for $r < R$.
Writing the change in the grand potential due to the insertion of the ball in its morphometric form (from \eqref{eq:surface-tension} and \eqref{eq:morphometric-surface-tension} of the main text), we have the \emph{ansatz}
\begin{equation}\label{eq:morph-ball-solvation}
  \Delta \Omega(R) =
  \frac{4\pi R^3}{3} p +
  4\pi R^2 \, \gamma_\infty +
  4\pi R \, \kappa +
  4 \pi \, \overline{\kappa}.
\end{equation}
If an equation of state for the pressure is taken as input, only three equations are needed to set the remaining coefficients of surface tension $\gamma_\infty, \kappa$ and $\overline{\kappa}$.

Restating the expressions for the insertion of a hard point and a new particle from the main text as
\begin{align}
  \label{eq:spt-point} \Delta\Omega(R=0) &= -k_B T \ln{(1- \eta)}, \\
  \label{eq:spt-mu} \Delta\Omega\left(R=\frac{\sigma}{2}\right) &= \mu^{ex},
\end{align}
we need one more equation to set the thermodynamic coefficients for the theory.
Following Ref.\ \cite{Bryk2003} we take the normal derivative of $\Omega$ with respect to $R$, and noting that $\Delta\Omega(R) = \Omega(R) - \Omega_{hom}$ gives
\begin{equation}\label{eq:spt-derivative}
  \left( \frac{\partial \Delta \Omega}{\partial R} \right)_{\mu,V,T} =
  \left( \frac{\partial \Omega}{\partial R} \right)_{\mu,V,T} =
  \int
  \frac{\delta \Omega[\rho_0(\vec{r})]}{\delta \rho}
  \left( \frac{\partial \rho_0(\vec{r})}{\partial R} \right)_{\mu,V,T}
  \, d\vec{r} +
  \int
  \rho_0(\vec{r})
  \left( \frac{\partial \phi_{ext}(\vec{r})}{\partial R} \right)_{\mu,V,T}
  \, d\vec{r},
\end{equation}
where $\rho_0$ is the equilibrium density profile and $\phi_{ext}$ is the external potential (i.e.\ the potential of the ball).
In equilibrium $\Omega$ is minimised so
\begin{equation}
  \left.
  \frac{\delta \Omega[\rho(\vec{r}); \phi_{ext}]}{\delta \rho}
  \right|_{\rho(\vec{r})=\rho_0(\vec{r})} = 0,
\end{equation}
so the first integral in \eqref{eq:spt-derivative} vanishes.
As the ball is hard, the external potential and its derivative are zero everywhere except at the surface $R$ where both $\rho_0$ and $\phi_{ext}$ are discontinuous.
We consider its Boltzmann weight, i.e.\
\begin{equation}
  e^{-\beta\phi_{ext}(\vec{r})} = \Theta(|\vec{r}| - R).
\end{equation}
Taking the derivative of both sides gives
\begin{equation}
  \beta\left( \frac{\partial\phi_{ext}(\vec{r})}{\partial R} \right)_{\mu,V,T} =
  \delta(|\vec{r}| - R) e^{\beta\phi_{ext}(\vec{r})}
\end{equation}
Inserting this expression into \eqref{eq:spt-derivative} and using the fact that $\rho(\vec{r}) e^{\beta\phi_{ext}(\vec{r})}$ is continuous (c.f.\ Ref.\ \cite{Hansen2013}) gives the contact theorem
\begin{equation}
  \beta \left( \frac{\partial \Omega}{\partial R} \right)_{\mu,V,T} =
  4\pi R^2 \rho(R).
\end{equation}
When $R = \sigma$ the inserted ball is equivalent to the hard sphere particles themselves, so $\Omega = \mu^{ex}$ and the contact density is $\rho(\sigma) = \rho \, g^{(2)}(\sigma)$ giving
\begin{equation}\label{eq:spt-contact-density}
  \left. \beta \left( \frac{\partial \Delta \Omega}{\partial R} \right)_{\mu,V,T}
  \right|_{R = \sigma}
  =
  \left. \beta \left( \frac{\partial \Omega}{\partial R} \right)_{\mu,V,T}
  \right|_{R = \sigma}
  =
  %\beta \frac{\partial \mu^{ex}}{\partial \sigma} =
  4\pi \sigma^2 \rho \, g^{(2)}(\sigma),
\end{equation}
or written in morphometric form using \eqref{eq:morph-ball-solvation} we have
\begin{equation}
  4\pi \sigma^2 \, p +
  8\pi \sigma \, \gamma_\infty +
  4\pi \, \kappa =
  \frac{4\pi \sigma^2 \rho}{\beta} \, g^{(2)}(\sigma).
\end{equation}
Applying the virial theorem (equation \eqref{eq:contact-g} in the main text) to the right hand side gives the final expression:
\begin{equation}\label{eq:spt-virial}
  4\pi \sigma^2 \, p +
  8\pi \sigma \, \gamma_\infty +
  4\pi \, \kappa =
  \frac{6}{\beta\sigma} \left( \frac{\beta p}{\rho} - 1 \right).
\end{equation}
Together \eqref{eq:spt-point}, \eqref{eq:spt-mu} and \eqref{eq:spt-virial} form a complete system of equations which we solve to obtain the coefficients
\begin{subequations}
  \begin{align}
    \frac{\beta \gamma_\infty^{WBII}}{R \rho} &=
    \left(\frac{\pi}{6\eta^2} - \frac{5\pi}{18\eta}\right) p -
    \frac{\mu^{ex}[p]}{3\eta} -
    \frac{\ln{(1-\eta)}}{3\eta} -
    \frac{1}{\eta}
    \label{eq:spt-gamma}
    \\
    \frac{\beta \kappa^{WBII}}{R^2\rho} &=
    \left( \frac{4\pi}{9\eta} - \frac{\pi}{2\eta^2} \right) p +
    \frac{4\mu^{ex}[p]}{3\eta} + \frac{4\ln{(1-\eta)}}{3\eta} + \frac{3}{\eta}
    \\
    \frac{\beta \overline{\kappa}^{WBII}}{R^3\rho} &=
    \left( \frac{\pi}{3\eta^2} - \frac{2\pi}{9\eta} \right) p -
    \frac{\mu^{ex}[p]}{\eta} - \frac{4\ln{(1-\eta)}}{3\eta} - \frac{2}{\eta}.
  \end{align}
\end{subequations}
Inserting the Carnahan-Starling parameters \eqref{eq:cs-pressure} and \eqref{eq:cs-mu} gives the coefficients explicitly as
\begin{subequations}
  \begin{align}
    \frac{\beta p^{WBII}}{\rho} &=
    \frac{1 + \eta + \eta^2 - \eta^3}{(1-\eta)^3}
    \\
    \frac{\beta \gamma_\infty^{WBII}}{R\rho} &=
    -\frac{1 + 2\eta + 8\eta^2 - 5\eta^3}{3(1-\eta)^3}
    - \frac{\ln{(1-\eta)}}{3\eta}
    \\
    \frac{\beta \kappa^{WBII}}{R^2\rho} &=
    \frac{4 - 10\eta + 20\eta^2 - 8\eta^3}{3(1-\eta)^3} + \frac{4 \ln{(1-\eta)}}{3\eta}
    \\
    \frac{\beta \overline{\kappa}^{WBII}}{R^3\rho} &=
    - \frac{4 - 11\eta + 13\eta^2 - 4\eta^3}{3(1-\eta)^3} - \frac{4 \ln{(1-\eta)}}{3\eta},
  \end{align}
\end{subequations}
which are \emph{identical} to the coefficients derived from the WBII free energy functional in Ref.\ \cite{Hansen-Goos2006}.
Remarkably, we have obtained these coefficients through a route completely different from their original derivation.
In Ref.\ \cite{Hansen-Goos2006} the coefficients were determined within FMT by taking the limit of a binary mixture where one component is infinitely dilute.
Here we completely avoided FMT, in favour of geometrical arguments similar to standard scaled particle approaches.
This suggests that the above scaled particle argument is somehow built into the structure of the WBII functional; we note that this is a nonobvious fact which cannot be determined from the form of the functional alone, nor is it obvious how it emerges from its original derivation.

Finally, note that (as described in the main text) the resulting $g^{(2)}$ performs poorly in the supercooled regime as compared with the ``exact'' result from virial theorem i.e.\ Eq.\ \eqref{eq:contact-g} in the main text.
In Fig.\ \ref{fig:contact-g} we plot the contact value with this set of coefficients, finding that it is reasonably accurate until around the freezing density where contact correlations spuriously decay.
The next section will detail how to modify the derivation to produce coefficients which describe more accurate correlation functions at high densities.

\subsubsection{Virial route}

In this section we give the detailed steps used in the derivation of the new set of morphometric coefficients, following the direction laid out in the main text.
In section \ref{SI:two-particles} above the morphological form of $g^{(2)}$ was computed explicitly; this derivation imposes the contact density using $\rho(\sigma) = \rho \, g^{(2)}(\sigma)$ with this explicit form.

From \eqref{eq:g2-explicit} the potential of mean force for neighbouring particles reduces to $\phi^{(2)}(r) = \Delta\Omega(r) - 2\mu^{ex}$ if they do not overlap, giving
\begin{equation}
  \begin{split}
  \phi^{(2)}(r) &\equiv - k_B T \ln g^{(2)}(r) \\
  &= pV(r) + \gamma_\infty A(r) + \kappa C(r) + \overline{\kappa} X(r) - 2\mu^{ex}
  \quad \forall \; \sigma \le r \le 2\sigma.
  \end{split}
\end{equation}
Inserting the morphological quantities at contact from \eqref{eq:g2-explicit-morph} gives
\begin{equation}
  \phi^{(2)}(\sigma) =
  \frac{9\pi \sigma^3}{4} p +
  6\pi\sigma^2 \, \gamma_\infty +
  \left( 6\pi\sigma - \frac{\pi^2\sigma}{2\sqrt{3}} \right) \, \kappa +
  4\pi \, \overline{\kappa}
  - 2\mu^{ex}[p].
\end{equation}
Equating this with $-k_B T \ln g^{(2)}(\sigma)$ and using the virial theorem (Eq.\ \eqref{eq:contact-g} in the main text) gives the final expression
\begin{equation}\label{eq:v-virial}
  \frac{9\pi \sigma^3}{4} p +
  6\pi\sigma^2 \, \gamma_\infty +
  \left( 6\pi\sigma - \frac{\pi^2\sigma}{2\sqrt{3}} \right) \, \kappa +
  4\pi \, \overline{\kappa} = 
  2\mu^{ex}[p] - \beta^{-1} \ln{\frac{3}{2\pi \rho \sigma^3} \left( \frac{\beta p}{\rho} - 1 \right)}.
\end{equation}
We will use this last expression instead of the contact theorem \eqref{eq:spt-virial} in order to obtain new coefficients.
Together \eqref{eq:spt-point}, \eqref{eq:spt-mu} and \eqref{eq:v-virial} solve to give coefficients:
\begin{subequations}
  \begin{align}
    \frac{\beta \gamma_\infty^{V}}{R\rho} &=
    \frac{ (18\pi - 7\sqrt{3} \pi^2) p + 6\sqrt{3}\pi \mu^{ex}[p]
      - 6(12 - \sqrt{3}\pi) \ln{(1-\eta)}
      - 72\ln{\left( \frac{\pi p - 6\eta}{24\eta^2} \right)} }
    {54(\sqrt{3}\pi - 4) \eta}
    \label{eq:virial-gamma}
    \\
    \frac{\beta \kappa^{V}}{R^2\rho} &=
    \frac{ 5\pi p - 12 \mu^{ex}[p] + 24 \ln{(1-\eta)}
      + 36\ln{\left( \frac{\pi p - 6\eta}{24\eta^2} \right)} }
    {9(\sqrt{3}\pi - 4) \eta}
    \\
    \frac{\beta \overline{\kappa}^{V}}{R^3\rho} &=
    - \frac{
      (18\pi - 2\sqrt{3} \pi^2) p - (36 - 3\sqrt{3}\pi) \mu^{ex}[p] + 12\sqrt{3}\pi \ln{(1-\eta)}
      + 72\ln{\left( \frac{\pi p - 6\eta}{24\eta^2} \right)} }
    {27(\sqrt{3}\pi - 4) \eta},
  \end{align}
\end{subequations}
which upon insertion of the Carnahan-Starling parameters \eqref{eq:cs-pressure} and \eqref{eq:cs-mu} gives the coefficients explicitly as
\begin{subequations}
  \begin{align}
    \frac{\beta p^{V}}{\rho} &=
    \frac{1 + \eta + \eta^2 - \eta^3}{(1-\eta)^3} \\
    \frac{\beta \gamma_\infty^{V}}{R\rho} &=
    \frac{ 18\eta \frac{1 + \eta + \eta^2 - \eta^3}{(1-\eta)^3}
      + \sqrt{3}\pi\eta \frac{1 - 16\eta - 4\eta^2 + 7\eta^3}{(1-\eta)^3}
      - (12 - \sqrt{3}\pi) \ln{(1-\eta)}
      - 12\ln{\left( \frac{2 - \eta}{2(1-\eta)^3} \right)} }
    {9(\sqrt{3}\pi - 4) \eta} \\
    \frac{\beta \kappa^{V}}{R^2\rho} &= -
    \frac{ 2\eta \frac{11 - 23\eta + \eta^2 + 5\eta^3}{(1-\eta)^3}
      - 8 \ln{(1-\eta)}
      - 12\ln{\left( \frac{2 - \eta}{2(1-\eta)^3} \right)} }
    {3(\sqrt{3}\pi - 4) \eta} \\
    \frac{\beta \overline{\kappa}^{V}}{R^3\rho} &=
    \frac{ 12\eta \frac{5 - 12\eta + 3\eta^3}{(1-\eta)^3}
      - \sqrt{3}\pi\eta \frac{4 - 13\eta - \eta^2 + 4\eta^3}{(1-\eta)^3}
      - 4\sqrt{3}\pi \ln{(1-\eta)}
      - 24\ln{\left( \frac{2 - \eta}{2(1-\eta)^3} \right)} }
    {9(\sqrt{3}\pi - 4) \eta}.
  \end{align}
\end{subequations}
Unlike the WBII coefficients above these are entirely new, and produce significantly more accurate correlation functions at high densities as described in the main text.
The pair correlation produced by these coefficients (black line in Fig.\ \ref{fig:contact-g}) is self-consistent with CS at contact by construction, but as an additional bonus we find that the new coefficients provide a theory that outperforms the older WBII approach across the whole range of distances typical of neighbouring particles (inset of Fig.\ \ref{fig:contact-g}).
The latter observation enables us to accurately model complex many-particle local structures.

However, it should be noted that the planar surface tension $\gamma_\infty^{V}$ is considerably less accurate than $\gamma_\infty^{WBII}$ as compared with molecular dynamics studies in \cite{Davidchack2016}.
For this reason, WBII coefficients may give more accurate grand potentials (and thus correlations) for large solutes where the surface becomes approximately planar.

\section{Accuracy of predicted distribution functions in $d = 3$}
\subsection{Comparison with molecular dynamics simulations}
\subsection{Comparison with Kirkwood superposition approximation}

\end{document}
