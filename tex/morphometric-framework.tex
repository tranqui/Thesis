\documentclass[12pt]{report}
\usepackage{preamble}
\setcounter{chapter}{2}

\begin{document}
\chapter{Morphological framework for many-body correlations}
Mostly theoretical but with some numerical experiments which motivate/justify developments to the theory.
The main body of numerical work is left to the following chapter.

\section{Formalism for many-body correlations}
A generalised potential distribution theorem and the potential of mean force.

\section{Morphological form of the potential of mean force}
Justification of assumptions: additivity, continuity and motion invariance
\subsection{Limitations known from DFT literature}
\subsection{As a generalisation of scaled particle theory}
And the limitations this implies.

\section{Worked examples where morphometric form can be exact}
Under certain conditions.
\subsection{Low density limit in arbitrary dimensions from lowest order terms in the virial expansion of the pressure}
\subsection{Arbitrary densities at large lengthscales}
\subsection{Hard rods (dimension d = 1) at all densities}
\subsubsection{Exact result from DFT}
\subsubsection{Morphometric result}
Explore where additivity, continuity and motion invariance apply.
\subsubsection{Implications for higher dimensions}

\section{Derivation of thermodynamic coefficients for hard spheres in $d = 3$}

\section{Accuracy of predicted distribution functions in $d = 3$}
\subsection{Comparison with molecular dynamics simulations}
\subsection{Comparison with Kirkwood superposition approximation}

\end{document}
