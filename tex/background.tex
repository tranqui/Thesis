%TC: macro \marginfootnote [other]
%TC: envir SCfigure [] other
%TC: macrocount beginSCfigure [figure]
\documentclass[11pt,twoside]{report}
\usepackage{preamble}
\setcounter{chapter}{1}
\graphicspath{{../img/}}
\def\includebibliography{}

\externaldocument{introduction}
\externaldocument{morphometric-framework}
\externaldocument{morphometric-applications}
\externaldocument{resummation}
\externaldocument{aerosols}

\begin{document}
\chapter{Geometry and the liquid state}
\epigraph{It tells me that the Creator used the wrong kind of circles.}{Terry Pratchett, \emph{Pyramids} (1989).}
%\epigraph{It was not certain what significance the ceremony held... but the formality was no less sacred for it being unintelligible.}{Mervyn Peake, \emph{Titus Groan}, (1946).}
\label{chapter:background}

In this chapter we provide a self-contained account of the foundational frameworks underlying the two themes of this thesis: \emph{integral geometry} and \emph{liquid state theory}.
I expect the reader to have a background in statistical physics, so my account of liquid state theory is not intended to be exhaustive; for more in-depth treatments see the references herein.
By contrast, I do \emph{not} expect much familiarity with integral geometry.
Understanding the underlying mathematical detail of this field is not essential to follow the rest of the thesis, so I will focus on the key concepts and notation rather than detailed derivation.

This chapter was assembled from my notes on liquid state theory over the previous few years, which I wanted to organise in one place mostly for my own benefit.
As such, this chapter is a little long so I anticipate the expert reader will skim over it; to facilitate this I have placed the important results in boxes as a guide to the most relevant parts.

As many-body correlation functions are a central theme of the results chapters, I have emphasised correlation functions in section \ref{sec:liquid-structure} on liquid structure to the point where I have somewhat belaboured giving the explicit forms and normalisations of the various correlation functions.
Even though we will only use one particular hierarchy of correlation functions in the results chapters, I personally found it helpful to have these formulas in one place.
I have found myself frequently revisiting the transformations between the various hierarchies of correlation functions, so I include them in anticipation that someone repeating or extending this work may profit from having a kind of ``cheat sheet''.

\section{Integral geometry}
The mathematical formalism which provides elegant and unified description of sizes, underlying the morphological approach taken in the thesis.
Ideas from this branch of mathematics were crucial to the development of fundamental measure theory, so it makes sense to place this before the section
on liquid state theory.
Integral geometry is generally unfamiliar to people from a physics background, so I will attempt to describe this area with additional care.

\subsection{Motivation}

\subsection{Generalised functions acting on sets}
\subsubsection{Set arithmetic}
Minkowski addition and its inverse
\subsubsection{Distances between sets}
The Hausdorff metric.
\subsubsection{Valuations on sets}
Additivity criterion and its significance

\subsection{Important theorems for continuous invariant valuations}
\subsubsection{Principal kinematic formula}
\subsubsection{Steiner formula for parallel surfaces}
\subsubsection{Hadwiger’s characterisation theorem}

\section{Liquid state theory}

Here we talk in general terms about descriptions of the liquid state.
Broadly speaking, in its historical development approaches can be placed inside one of two categories.
Namely, theories involving
\begin{enumerate}
  \item Local geometric approximations capturing the short range interactions (free volume theory/cell theory, scaled particle theory), and
  \item Integral equations (Ornstein-Zernike closures, density functional theory) which properly treat the long range correlations.
\end{enumerate}
These two approaches are not mutually exclusive, and hybrid theories can improve on.
For instance, fundamental measure theory (FMT) involves the synthesis of integral geometry with the formalism of density functional theory which involves minimising a functional (i.e.\ an integral equation).

Classical theory of phase transitions (Landau).
Van der waals theory.
How does the transition occur?
Metastability leads into kinetics.

Kinetics vs thermodynamics.
Thermodynamic driving force vs activation barrier.

Relaxation behaviour controlled by activation barrier.
A thermal fluctuation which takes the system over the barrier%
\marginfootnote{These fluctuations are conventionally called \emph{instantons} as they spontaneously appear and vanish just like virtual particles in fundamental physics.
  The name for this relatively straightforward phenomenon is thus a reference to a much more counterintuitive and bizarre phenomenon, because physicists are good at making helpful analogies.}
occurs with rate $\exp{-\beta \Delta U}$ \cite{Langer}.

Liquids:
Free volume theory.
Early curvature corrections.
Free volume theory depends on the free volume (duh).
This is an example of an integral geometric theory.
Morphological 

Technical - handwavey:
Motion invariance: translation and rotation invariance.
Additivity: the energy is extensive (even for small system sizes)
Continuity: the function is well behaved

A summary of results in the modern formalism as given by standard texts (notably references \cite{Hansen2013} and \cite{Santos2016}) that will be used throughout the thesis.

\subsection{Interaction potentials}
\subsection{Simple liquids}
\subsection{Hard sphere model system}

\subsubsection{Grand canonical ensemble}
\paragraph{Averages}
The partition function and free energy
\paragraph{Correlations}
Particle densities $\rho^{(n)}$ and distribution functions $g^{(n)}$.

\subsection{Homogeneous liquid state theory}

\subsubsection{Virial expansion of the equation of state}

\paragraph{Distribution function form}
Pressure in terms of pair distribution function $g(r)$.

For pair potentials:
\begin{equation}
  \frac{\beta P}{\rho} =
  1 - \frac{2 \pi \beta \rho}{3}
  \int_0^\infty v'(r) g(r) r^3 \, dr
\end{equation}
Generalisation to many-body potentials requires higher order distribution functions.
Difficulty of discontinuities overcome by introducing cavity function
\begin{equation}
  y(r) = e^{\beta v(r)} g(r)
\end{equation}
which is continuous \todo{Find out why this is continuous} leading to
\begin{equation}
  \frac{\beta P}{\rho} =
  1 + \frac{2 \pi \beta \rho}{3}
  \int_0^\infty e'(r) g(r) r^3 \, dr,
\end{equation}
where Boltzmann factor of pair potential is
\begin{equation}
  e(r) = e^{\beta v(r)}.
\end{equation}

\paragraph{Diagrammatic form}
Low density expansion of pressure in series with virial coefficients.

\begin{equation}
  \frac{\beta P}{\rho} =
  1 + \sum_{i=2}^\infty B_i(T) \rho^{i-1}
\end{equation}
where $B_i$ are the virial coefficients.

Generalisation to a binary mixture \todo{Where does this come from? What do the diagrams look like? Are they simpler than the distribution function diagrams?}: \cite{Hansen-Goos2014}
\begin{equation}\label{eq:virial-series-binary}
  \Phi = \sum_{n=2}^\infty \sum_{j=0}^{n}
  \frac{1}{n-1} {n \choose j} B_n^{[j]} \rho_1^{n-j} \rho_2^j
\end{equation}

\paragraph{Empirical Carnahan-Starling equation of state for hard spheres}

The excess free energy is determined from the equation of state by
\begin{equation}
  \frac{\beta F^{ex}}{N}
  = \int_0^\eta \left( \frac{\beta p}{\rho} - 1 \right) \, \frac{d\eta'}{\eta'},
\end{equation}
giving the excess chemical potential from the thermodynamic relation
\begin{equation}\label{eq:chemical-potential}
  \beta \mu^{ex}[p]
  = \beta \left( \frac{\partial F^{ex}}{\partial N} \right)_{V,T}
  = \left( \frac{\beta p}{\rho} - 1 \right)
  + \int_0^\eta \left( \frac{\beta p}{\rho} - 1 \right) \, \frac{d\eta'}{\eta'}.
\end{equation}

The Carnahan-Starling equation of state approximates the pressure for hard spheres as \cite{Carnahan1969}
\begin{equation}\label{eq:cs-pressure}
  \frac{\beta p_{cs}}{\rho} = \frac{1 + \eta + \eta^2 - \eta^3}{(1-\eta)^3},
\end{equation}
which gives the excess chemical potential using \eqref{eq:chemical-potential} as
\begin{equation}\label{eq:cs-mu}
  \beta \mu_{cs}^{ex} = \frac{8\eta - 9\eta^2 + 3\eta^3}{(1-\eta)^3}.
\end{equation}

\subsubsection{Free energy from distribution functions}
\paragraph{Distribution function theories}
\paragraph{Kirkwood superposition approximation}
\paragraph{Distribution functions from direct correlation functions: Ornstein-Zernike equation}

\subsubsection{Beyond hard spheres: perturbation theory and the mean field approximation}

\subsection{Inhomogeneous liquid state theory}

\subsubsection{Solvation physics}

\section{Functional calculus}

The chain rule of functional calculus is
\todo{Insert citation to some calculus of variations reference, and Hansen2013/Bob's reviews for application to liquids.}
\begin{equation*}
  \frac{\delta f}{\delta u(\vec{r})} =
  \int
  \frac{\delta f}{\delta v(\vec{r}')}
  \frac{\delta v(\vec{r}')}{\delta u(\vec{r})}
  \, d\vec{r}'
\end{equation*}
and inverse derivatives are found via
\begin{equation*}
  \int
  \frac{\delta u(\vec{r})}{\delta v(\vec{r}')}
  \frac{\delta v(\vec{r}')}{\delta u(\vec{r})}
  \, d\vec{r} =
  \delta(\vec{r} - \vec{r}')
\end{equation*}

\section{Correlation functions}

\begin{equation}
  H^{(n)}(\vec{r}^n) =
  \left\langle
  \prod_{i=1}^n
  \Big[ \rho(\vec{r}_i) - \rho^{(1)}(\vec{r}_i) \Big]
  \right\rangle
  \qquad \forall \; n \ge 2
\end{equation}
\begin{equation}
  \begin{aligned}
    H^{(2)}(\vec{r}_1, \vec{r}_2) &=
    \left\langle
    \left[ \rho(\vec{r}_1) - \rho^{(1)}(\vec{r}_1) \right]
    \left[ \rho(\vec{r}_2) - \rho^{(1)}(\vec{r}_2) \right]
    \right\rangle \\
    &=
    \big\langle \rho(\vec{r}_1) \rho(\vec{r}_2) \big\rangle -
    \rho^{(1)}(\vec{r}_1) \rho^{(1)}(\vec{r}_2) \\
    &=
    \rho^{(2)}(\vec{r}_1, \vec{r}_2) +
    \rho^{(1)}(\vec{r}_1) \delta(\vec{r}_1 - \vec{r}_2) -
    \rho^{(1)}(\vec{r}_1) \rho^{(1)}(\vec{r}_2) \\
    &=
    \rho^{(1)}(\vec{r}_1) \rho^{(1)}(\vec{r}_2) h^{(2)}(\vec{r}_1, \vec{r}_2)
    +
    \rho^{(1)}(\vec{r}_1) \delta(\vec{r}_1 - \vec{r}_2)
  \end{aligned}
\end{equation}
where
\begin{equation}
  h^{(n)}(\vec{r}^n) \equiv g^{(n)}(\vec{r}^n) - 1
\end{equation}

Density-density correlation functions
\begin{equation}
  H^{(n)}(\vec{r}^n) =
  \left\langle
  \prod_{i=1}^n
  \Big[ \rho(\vec{r}_i) - \big\langle\rho(\vec{r}_i)\big\rangle \Big]
  \right\rangle
  \qquad \forall \; n \ge 2
\end{equation}
They are obtained from the total grand potential by repeat functional differentiation, as in
\begin{equation}
  \begin{aligned}
  H^{(n)}(\vec{r}^n) &=
  - \frac{\delta^n \beta \Omega}{\delta \beta\psi(\vec{r}_1) \delta \beta\psi(\vec{r}_2) \cdots \delta \beta\psi(\vec{r}_n)} \\
  &=
  \frac{\delta^{n-1} \rho(\vec{r}_1)}{\delta \beta\psi(\vec{r}_2) \delta \beta\psi(\vec{r}_3) \cdots \delta \beta\psi(\vec{r}_n)}.
  \end{aligned}
\end{equation}
I.e.\ the grand potential is the generating functional for the density-density correlation functions.

\section{Density functional theory}

\subsection{Functional form of thermodynamic potentials}

From the Legendre transform of $\beta \Omega$
\begin{equation}
  \begin{aligned}
    \Omega[\rho(\vec{r})] &=
    F[\rho(\vec{r})] -
    \int \rho(\vec{r}) \psi(\vec{r}) \, d\vec{r} \\
    &=
    F_{id}[\rho(\vec{r})] +
    F_{ex}[\rho(\vec{r})] -
    \int \rho(\vec{r}) \psi(\vec{r}) \, d\vec{r}
  \end{aligned}
\end{equation}
after defining intrinsic chemical potential
\begin{equation}\label{eq:intrinsic-chemical-potential}
  \psi(\vec{r}) = \mu - V_{ext}(\vec{r})
\end{equation}

For the ideal gas
\begin{align*}
  \Xi_{id} &=
  \sum_{N=0}^\infty
  \frac{(e^{\beta\mu} Z_1)^N}{N!}
  = \exp{\left( Z_1 e^{\beta \mu} \right)} \\
  Z_1 &= \int e^{-\beta V_{ext}(\vec{r})} \, d\vec{r} \\
  \beta\Omega_{id}[V_{ext}] &=
  - \ln{\Xi_{id}}
  =
  - e^{\beta\mu} \int e^{-\beta V_{ext}(\vec{r})} \, d\vec{r} \\
  \rho^{(1)}(\vec{r}) &=
  \frac{\langle N \rangle e^{-\beta V_{ext}(\vec{r})}}
       {\int e^{-\beta V_{ext}(\vec{r}')} \, d\vec{r}'} =
  \frac{\langle N \rangle}{Z_1} e^{-\beta V_{ext}(\vec{r})}
  \\
  \beta V_{ext}(\vec{r})
  &=
  -\ln
  \left(
  \frac{\rho^{(1)}(\vec{r})}{\langle N \rangle}
  \int e^{-\beta V_{ext}(\vec{r}')} \, d\vec{r}'
  \right)
  \\
  e^{\beta \mu} &= \lambda^d \rho
  \\
  F[\rho]
  &=
  - \lambda^d \rho
  \int e^{-\beta V_{ext}(\vec{r})} \, d\vec{r} -
  \int
  \rho(\vec{r})
  (V_{ext}(\vec{r}) - \mu)
  \, d\vec{r}
  \\
  &=
  - \lambda^d \rho
  \int e^{-\beta V_{ext}(\vec{r})} \, d\vec{r} -
  \int
  \rho(\vec{r})
  V_{ext}(\vec{r})
  \, d\vec{r} -
  \int
  \rho(\vec{r}) \mu
  \, d\vec{r}
  \\
  &=
  - \lambda^d \rho
  \int e^{-\beta V_{ext}(\vec{r})} \, d\vec{r} -
  \int
  \rho(\vec{r})
  \ln {\frac{Z_1}{\langle N \rangle} \rho(\vec{r})}
  \, d\vec{r} +
  \int
  \rho(\vec{r}) \mu
  \, d\vec{r}
  \\
  &=
  - \lambda^d \rho
  \int
  \frac{\rho^{(1)}(\vec{r})}{\langle N \rangle}
  \left( \int e^{-\beta V_{ext}(\vec{r}')} \, d\vec{r}' \right)
  \, d\vec{r} -
  \int
  \rho(\vec{r})
  \ln{\left(
  \frac{\rho^{(1)}(\vec{r})}{\langle N \rangle}
  \int e^{-\beta V_{ext}(\vec{r}')} \, d\vec{r}'
  \right)}
  (V_{ext}(\vec{r}) - \mu)
  \, d\vec{r}
  \\
  \beta\Omega_{id}[\rho] &=
  F[\rho] + \int \rho(\vec{r}) (V_{ext}(\vec{r}) - \mu) \\
  &=
  F[\rho] + \int \rho(\vec{r}) (V_{ext}(\vec{r}) - \mu) \\
\end{align*}
Following \cite{Ashcroft1996}, we write the single-particle density as
\begin{equation*}
  \rho^{(1)}(\vec{r}) =
  \frac{1}{\Xi}
  \Tr{ \left(
    e^{-\beta U_N(\vec{r}^N)} \sum_{i=1}^N \delta(\vec{r} - \vec{r}')
    \right)
  }
\end{equation*}
which in the absence of particle interactions reduces to
\begin{equation*}
  \rho^{(1)}(\vec{r}) =
  \frac{\langle N \rangle e^{-\beta V_{ext}(\vec{r})}}
       {\int e^{-\beta V_{ext}(\vec{r}')} \, d\vec{r}'}
\end{equation*}
We note that the energetic contribution to the free energy arising solely from the external potential is
\begin{equation*}
  \int \rho(\vec{r}') V_{ext}(\vec{r}') \, d\vec{r}'
\end{equation*}
The intrinsic free energy of the ideal gas is then obtained by subtracting the contributions from the external potential, as in
\begin{equation*}
  \begin{aligned}
  F_{id}[\rho] &=
  F[\rho] -
  \int \rho(\vec{r}') V_{ext}(\vec{r}') \, d\vec{r}' \\
  &=
  F[\rho] -
  \int \rho(\vec{r}') V_{ext}(\vec{r}') \, d\vec{r}' \\
  \end{aligned}
\end{equation*}

Derivatives of the ideal gas free energy.
The ideal gas free energy is given as
\begin{equation*}
  \beta F_{id}[\rho(\vec{r})] =
  \int \rho(\vec{r}) (\ln{\lambda^d \rho(\vec{r})} - 1) \, d\vec{r},
\end{equation*}
so the first functional derivative is
\begin{equation*}
  \frac{\delta \beta F_{id}}{\delta \rho(\vec{r})} =
  \ln{\lambda^d \rho(\vec{r})}.
\end{equation*}
To obtain the higher order functional derivatives it is helpful to write this as an integral with a delta function
\begin{equation*}
  \frac{\delta \beta F_{id}}{\delta \rho(\vec{r})} =
  \int \delta{(\vec{r}' - \vec{r})}
  \ln{\lambda^d \rho(\vec{r}')} \, d\vec{r}',
\end{equation*}
so we can obtain the second derivative as
\begin{equation*}
  \frac{\delta^2 \beta F_{id}}{\delta \rho(\vec{r}) \delta \rho(\vec{r}')} =
  \frac{\delta(\vec{r}'-\vec{r})}{\rho(\vec{r})}.
\end{equation*}
Iterating this procedure gives us the $n$th functional derivative as
\begin{equation}
  \begin{aligned}
    \frac{\delta^n \beta F_{id}}{\delta \rho(\vec{r}_1) \delta \rho(\vec{r}_2) \cdots \delta \rho(\vec{r}_n)} &=
    \frac{\partial^{n-1} (\ln{\lambda^d \rho(\vec{r})})}{\partial \rho(\vec{r})^{n-1}}
    \prod_{i=2}^n \delta(\vec{r}_i - \vec{r}_1) \\
    &=
    (-1)^{n-1}
    \frac{(n-2)!}{\rho(\vec{r})^{n-1}}
    \prod_{i=2}^n \delta(\vec{r}_i - \vec{r}_1),
  \end{aligned}
\end{equation}
where the last line is valid for all $n \ge 2$.

\subsection{Correlation functions}

Excess free energy is the generating functional for direct correlations
\begin{equation}\label{eq:direct-correlations}
  c^{(n)}(\vec{r}^n) =
  - \frac{\delta^n \beta F_{ex}}{\delta \rho(\vec{r}_1)\delta \rho(\vec{r}_2) \cdots \delta \rho(\vec{r}_n)}
\end{equation}

We have to compute a derivative like
\begin{equation}\label{eq:intrinsic-chemical-potential-derivative-1}
  \frac{\delta}{\delta\rho(\vec{r})}
  \left(
  \int \psi(\vec{r}') \rho(\vec{r}') \, d\vec{r}'
  \right)
  =
  \psi(\vec{r}) +
  \int
  \rho(\vec{r}') \frac{\delta\psi(\vec{r}')}{\delta\rho(\vec{r})}
  \, d\vec{r}'.
\end{equation}
The functional derivative on the right-hand side of \eqref{eq:intrinsic-chemical-potential-derivative-1} is a little odd.
In general the external potential $V_{ext}(\vec{r})$%
\marginfootnote{And thus $\psi(\vec{r})$ through \eqref{eq:intrinsic-chemical-potential}}
is fixed so we consider the density profile as relaxing in response to perturbations from the potential i.e.\ terms like \[ \frac{\delta \rho(\vec{r})}{\delta \psi(\vec{r}')}. \]
However, here the \emph{inverse} derivative appears.
This must satisfy the inversion formula
\begin{equation}\label{eq:intrinsic-chemical-potential-inverse-derivative}
  \int
  \frac{\delta \rho(\vec{r}_1)}{\delta \psi(\vec{r}')}
  \frac{\delta \psi(\vec{r}')}{\delta \rho(\vec{r}_2)}
  \, d\vec{r}' =
  \delta(\vec{r}_1 - \vec{r}_2),
\end{equation}
which will be very important for obtaining Ornstein-Zernike relations later.
Considering the external field as the control parameter, and noting the definition of $\psi(\vec{r})$ in \eqref{eq:intrinsic-chemical-potential} we take
\begin{equation*}
  \frac{\delta\psi(\vec{r}')}{\delta\rho(\vec{r})} = 0.
\end{equation*}
This expression satisfies \eqref{eq:intrinsic-chemical-potential-inverse-derivative}, being zero in general except where $\psi(\vec{r}')$ is used as an intermediate function in the (functional) chain rule expression.
With this expression \eqref{eq:intrinsic-chemical-potential-derivative-2} becomes
\begin{equation}\label{eq:intrinsic-chemical-potential-derivative-2}
  \frac{\delta}{\delta\rho(\vec{r})}
  \left(
  \int \psi(\vec{r}') \rho(\vec{r}') \, d\vec{r}'
  \right)
  =
  \psi(\vec{r}).
\end{equation}

In equilibrium
\begin{equation}
  \left.
  \frac{\delta \Omega[\rho]}{\delta\rho(\vec{r})}
  \right|_{\rho(\vec{r}) = \rho^{(1)}(\vec{r})}
  = 0
\end{equation}
and consequently all higher-order derivatives must be zero, i.e.\
\begin{equation}
  \frac{\delta^n \Omega[\rho]}{\delta\rho(\vec{r}_1) \delta\rho(\vec{r}_2) \cdots \delta\rho(\vec{r}_n)} = 0 \qquad \forall \; n \ge 1
\end{equation}
therefore we have
\begin{equation}
  \frac{\delta^n F[\rho]}{\delta \rho(\vec{r}_1) \delta \rho(\vec{r}_2) \cdots \delta \rho(\vec{r}_n)} -
  \frac{\delta}{\delta\rho(\vec{r})}
  \left(
  \int \psi(\vec{r}') \rho(\vec{r}') \, d\vec{r}'
  \right)
  = 0
\end{equation}
or using \eqref{eq:direct-correlations} this becomes
\begin{equation*}
  c^{(n)}(\vec{r}^n) =
  \frac{\delta^n \beta F_{id}[\rho]}{\delta \rho(\vec{r}_1) \delta \rho(\vec{r}_2) \cdots \delta \rho(\vec{r}_n)} -
  \frac{\delta^n}{\delta \rho(\vec{r}_1) \delta \rho(\vec{r}_2) \cdots \delta \rho(\vec{r}_n)}
  \left(
  \int \beta\psi(\vec{r}') \rho(\vec{r}') \, d\vec{r}'
  \right)
\end{equation*}
At the one-body level we have:
\begin{equation}\label{eq:c1}
  \begin{aligned}
    c^{(1)}(\vec{r}) &=
    \frac{\delta \beta F_{id}}{\delta \rho(\vec{r})} -
    \frac{\delta}{\delta \rho(\vec{r})^{(1)}}
    \left(
    \int \beta\psi(\vec{r}') \rho(\vec{r}') \, d\vec{r}'
    \right) \\
    &=
    \ln{\lambda^d \rho(\vec{r})} -
    \beta\psi(\vec{r})
  \end{aligned}
\end{equation}
Which gives the equilibrium density as
\begin{equation}
  \rho^{(1)}(\vec{r}) = \lambda^{-d} \exp{\left(\beta\psi(\vec{r}) + c^{(1)}(\vec{r})\right)}
\end{equation}
We obtain the two-body correlations by functionally differentiating \eqref{eq:c1}, as in
\begin{equation}\label{eq:c2}
  \begin{aligned}
    c^{(2)}(\vec{r}_1, \vec{r}_2) &=
    \frac{\delta c^{(1)}(\vec{r}_1)}{\delta \rho^{(1)}(\vec{r}_2)} \\
    &=
    \frac{\delta(\vec{r}_2 - \vec{r}_1)}{\rho^{(1)}(\vec{r}_1)} -
    \frac{\delta\beta\psi(\vec{r}_1)}{\delta \rho(\vec{r}_2)}
  \end{aligned}
\end{equation}
From \eqref{eq:intrinsic-chemical-potential-inverse-derivative} we get
\begin{equation*}
  \begin{aligned}
    \delta(\vec{r}_1 - \vec{r}_2) &=
    \int
    \frac{\delta \rho(\vec{r}_1)}{\delta \psi(\vec{r}')}
    \frac{\delta \psi(\vec{r}')}{\delta \rho(\vec{r}_2)}
    \, d\vec{r}' \\
    &=
    \int
    H^{(2)}(\vec{r}_1, \vec{r}')
    \left(
    \frac{\delta(\vec{r}' - \vec{r}_2)}{\rho^{(1)}(\vec{r}')} -
    c^{(2)}(\vec{r}', \vec{r}_2)
    \right)
    \, d\vec{r}' \\
    &=
    \rho^{(1)}(\vec{r}_1)
    \left(
    h^{(2)}(\vec{r}_1, \vec{r}_2) -
    c^{(2)}(\vec{r}_1, \vec{r}_2)
    \right) +
    \delta(\vec{r}_1 - \vec{r}_2) - \\
    &\qquad
    \rho^{(1)}(\vec{r}_1)
    \int
    \rho^{(1)}(\vec{r}')
    h^{(2)}(\vec{r}_1, \vec{r}')
    c^{(2)}(\vec{r}', \vec{r}_2)
    \, d\vec{r}'
  \end{aligned}
\end{equation*}
which rearranges to give the Ornstein-Zernike equation
\begin{equation}
  h^{(2)}(\vec{r}_1, \vec{r}_2) =
  c^{(2)}(\vec{r}_1, \vec{r}_2) +
  \int
  \rho^{(1)}(\vec{r}')
  h^{(2)}(\vec{r}_1, \vec{r}')
  c^{(2)}(\vec{r}', \vec{r}_2)
  \, d\vec{r}'.
\end{equation}
This is a classic result in liquid state theory (cf.\ Refs.\ \cite{Ornstein1914,Hansen2010,Evans1979}) though normally it is expressed for the uniform liquid $\rho^{(1)}(\vec{r}) = \rho$ for spherically symmetric pair potentials thus
\begin{equation}
  h^{(2)}(r) =
  c^{(2)}(r) +
  \rho
  \int
  h^{(2)}(r)
  c^{(2)}(|\vec{r}' - \vec{r}|)
  \, d\vec{r}',
\end{equation}
where $r = |\vec{r}_2 - \vec{r}_1|$.
The integral is a convolution, so it simplifies under Fourier transform to
\begin{equation}
  \tilde{h}(\vec{k}) =
  \tilde{c}(\vec{k}) +
  \rho \tilde{h}(\vec{k}) \tilde{c}(\vec{k})
\end{equation}
or rearranging for
\begin{equation}
  \tilde{h}(\vec{k}) =
  \frac{\tilde{c}(\vec{k})}{1 - \rho \tilde{c}(\vec{k})}.
\end{equation}
Note that the static structure factor is defined as the Fourier transform of the two-body distribution function, i.e.\
\begin{equation}
  \begin{aligned}
    S(\vec{k}) \equiv \tilde{g}^{(2)}(\vec{k}) &=
    \delta(\vec{k}) + \tilde{h}(\vec{k}) \\
    &=
    \delta(\vec{k}) +
    \frac{\tilde{c}(\vec{k})}{1 - \rho \tilde{c}(\vec{k})}.
  \end{aligned}
\end{equation}

\subsection{Generalised Ornstein-Zernike equations}

This section follows \cite{Barrat1988}.

Thus for higher $n$ we have
\begin{equation}
  \begin{aligned}
    c^{(n)}(\vec{r}^n) &=
    \frac{\delta^{n-1} c^{(1)}(\vec{r}_1)}{\delta \rho(\vec{r}_2) \delta \rho(\vec{r}_3) \cdots \delta \rho(\vec{r}_n)} \\
    &=
    (-1)^n
    \frac{(n-2)!}{\rho(\vec{r}_1)^{n-1}}
    \prod_{i=2}^n \delta(\vec{r}_i - \vec{r}_1) -
    \frac{\delta^{n-1} \beta\psi(\vec{r}_1)}{\delta \rho(\vec{r}_2) \delta \rho(\vec{r}_3) \cdots \delta \rho(\vec{r}_n)}
  \end{aligned}
\end{equation}
or
\begin{equation}
  \frac{\delta^{n-1} \beta\psi(\vec{r}_1)}{\delta \rho(\vec{r}_2) \delta \rho(\vec{r}_3) \cdots \delta \rho(\vec{r}_n)} =
  (-1)^n
  \frac{(n-2)!}{\rho(\vec{r}_1)^{n-1}}
  \prod_{i=2}^n \delta(\vec{r}_i - \vec{r}_1) -
  c^{(n)}(\vec{r}^n)
\end{equation}

\begin{equation*}
  H^{(n)}(\vec{r}^n) =
  \frac{\delta^{n-1} \rho(\vec{r}_1)}{\delta \beta\psi(\vec{r}_2) \delta \beta\psi(\vec{r}_3) \cdots \delta \beta\psi(\vec{r}_n)}.
\end{equation*}

Defining
\begin{equation*}
  K^{(n)}(\vec{r}^n) =
  \frac{\delta^{n-1} \beta\psi(\vec{r}_1)}{\delta \rho(\vec{r}_2) \delta \rho(\vec{r}_3) \cdots \delta \rho(\vec{r}_n)}
\end{equation*}
we have
\begin{equation*}
  \frac{\delta K^{(n)}(\vec{r}^n)}{\delta \rho(\vec{r}_{n+1})} =
  K^{(n+1)}(\vec{r}^{n+1}).
\end{equation*}
and
\begin{equation*}
  \begin{aligned}
    \frac{\delta H^{(n)}(\vec{r}^n)}{\delta \rho(\vec{r}_{n+1})}
    &=
    \int
    \frac{\delta H^{(n)}(\vec{r}^n)}{\delta \psi(\vec{r}')}
    \frac{\delta \psi(\vec{r}')}{\delta \psi(\vec{r}_{n+1})}
    \, d\vec{r}' \\
    &=
    \int
    H^{(n+1)}(\vec{r}^n, \vec{r}')
    K^{(2)}(\vec{r}', \vec{r}_{n+1})
    \, d\vec{r}' \\
    &=
    H^{(n+1)} \otimes K^{(2)}(\vec{r}^{n+1}).
  \end{aligned}
\end{equation*}
In this form the Ornstein-Zernike equation can be written.
\begin{equation*}
  \begin{aligned}
    \delta(\vec{r}_1 - \vec{r}_2) &=
    \int
    \frac{\delta \rho(\vec{r}_1)}{\delta \psi(\vec{r}')}
    \frac{\delta \psi(\vec{r}')}{\delta \rho(\vec{r}_2)}
    \, d\vec{r}' \\
    &=
    \int
    H^{(2)}(\vec{r}_1, \vec{r}') K^{(2)}(\vec{r}', \vec{r}_2)
    \, d\vec{r}' \\
    &=
    H^{(2)} \otimes K^{(2)} (\vec{r}^2)
  \end{aligned}
\end{equation*}
Taking functional derivatives of this expression gives us a hierarchy of generalised Ornstein-Zernike equations.
For example, the next equation in the hierarchy is
\begin{equation*}
  H^{(2)} \otimes K^{(3)} (\vec{r}^3) +
  H^{(3)} \otimes K^{(2)} \otimes K^{(2)} (\vec{r}^3) = 0
\end{equation*}
The next functional derivative
\begin{equation}
  \begin{aligned}
  H^{(2)} \otimes K^{(4)} (\vec{r}^4) & \\
  + \; 2 H^{(3)} \otimes K^{(3)} \otimes K^{(2)} (\vec{r}^4) & \\
  + \; H^{(4)} \otimes K^{(2)} \otimes K^{(2)} \otimes K^{(2)} (\vec{r}^4)
  &= 0
  \end{aligned}
\end{equation}
And the next one%
\todo{Can we find a general formula? I notice some constraints on the indices: the sum of the indices $\{m\}$ in the $K^{(m)}$ terms must add up to $n-1$ so that the right number of independent variables are returned (the extra one is provided by the $H^{(l)}$ function giving the $\vec{r}^n$ total.}
\begin{equation}
  \begin{aligned}
    H^{(2)} \otimes K^{(5)} (\vec{r}^5) & \\
    + \; 3 H^{(3)} \otimes K^{(3)} \otimes K^{(3)} (\vec{r}^5) & \\
    + \; 2 H^{(3)} \otimes K^{(4)} \otimes K^{(2)} (\vec{r}^5) & \\
    + \; 5 H^{(4)} \otimes K^{(3)} \otimes K^{(2)} \otimes K^{(2)} (\vec{r}^5) & \\
    + \; H^{(5)} \otimes K^{(2)} \otimes K^{(2)} \otimes K^{(2)} \otimes K^{(2)} (\vec{r}^5)
    &= 0
  \end{aligned}
\end{equation}

\subsubsection{Density functional theory (DFT)}

Classic texts are \cite{Evans1979,Evans1992}, and a more recent review \cite{Roth2010}.
Also mention \cite{Lutsko2010} for more of a focus on crystallisation.
This exposition follows \cite{Roth2010} mainly.

Density functional theory traces back to 

Rigorously prove that \cite{Evans1979,Evans1992}
\begin{equation}
  \Omega[\{\rho_i\}] =
  \mathcal{F}[\{\rho_i\}]
  + \sum_{i=1}^\nu \int d^d \vec{r} \rho_i(\vec{r}) (\phi(\vec{r}) - \mu_i)
\end{equation}

\begin{equation}
  \left.
  \frac{\delta \Omega[\{\rho_i\}]}{\delta \rho_i(\vec{r})}
  \right|_{\{\rho_i(\vec{r}) = \rho_i^0(\vec{r})\}}
  = 0.
\end{equation}

Split into ideal and excess parts
\begin{equation}
  \mathcal{F}[\{\rho_i\}] =
  \mathcal{F}_{id}[\{\rho_i\}] + \mathcal{F}_{ex}[\{\rho_i\}]
\end{equation}
where the ideal part is
\begin{equation}
  \beta \mathcal{F}_{id}[\{\rho_i\}] =
  \sum_{i=1}^\nu \int d^d \vec{r} \rho_i(\vec{r})
  (\ln{\Lambda_i^d \rho_i(\vec{r})} - 1)
\end{equation}

\begin{itemize}
\item Contrast mechanical problem (e.g.\ simulations) with inverse problem
\item Summarise successes
\end{itemize}

\subsubsection{Heterogeneous approaches to the homogeneous liquid}
\paragraph{Potential distribution theorem}
\paragraph{Scaled particle theory}

\section{Density functional theory}

\subsection{Functional form of the thermodynamic potentials}

\todo{Insert references to classic texts i.e.\ \cite{Evans1979,Evans1992}}

For inhomogeneous systems the Legendre transform of $\Omega$ \eqref{eq:grand-potential-legendre-transform} generalises to
\begin{equation}
  \Omega = F - \int \rho^{(1)}(\vec{r}) \mu \, d\vec{r}.
\end{equation}
Subtracting external potential contributions from the Helmholtz free energy defines an \emph{intrinsic free energy} containing contributions arising solely from the internal interactions, i.e.\
\begin{equation}\label{eq:intrinsic-free-energy}
  \mathcal{F}
  =
  F - \int \rho^{(1)}(\vec{r}) \phi_\mathrm{ext}(\vec{r}) \, d\vec{r}
\end{equation}
so that the grand potential becomes
\begin{equation}\label{eq:dft-grand-potential}
  \begin{split}
    \Omega
    &=
    \mathcal{F}
    - \int \rho^{(1)}(\vec{r}) (\mu - \phi_\mathrm{ext}(\vec{r})) \, d\vec{r}
    \\ &=
    \mathcal{F}
    - \int \rho^{(1)}(\vec{r}) \psi(\vec{r}) \, d\vec{r}
  \end{split}
\end{equation}
where we defined the \emph{intrinsic chemical potential} $\psi(\vec{r}) = \mu - \phi_\mathrm{ext}(\vec{r})$ in the final step.

Furthermore, the intrinsic free energy can be decomposed into an \emph{ideal} and \emph{excess} part as in
\begin{equation}\label{eq:F-decomposition}
  \mathcal{F}
  =
  \mathcal{F}^\mathrm{id} +
  \mathcal{F}^\mathrm{ex}.
\end{equation}
The excess component emerges as from the interactions between particles and in general it is intractably hard to determine this exactly except in special limits (e.g.\ in the one-dimensional limit).
As such, approximate forms for $\mathcal{F}^\mathrm{ex}$ must be used in general which contrains the success of applications of DFT to the accuracy of this contribution.
By contrast, the ideal component can be computed explicitly.
The partition function for the ideal gas is easily calculated giving
\begin{equation}\label{eq:ideal-grand-canonical-partition}
  \Xi^\mathrm{id}
  =
  \sum_{N=0}^\infty
  \frac{(e^{\beta\mu} Z_1)^N}{N!}
  =
  \exp{\left( \frac{Z_1 e^{\beta \mu}}{\Lambda^d} \right)}.
\end{equation}
Then, following Ref.\ \cite{Ashcroft1996}, we write the equilibrium single-particle density as
\begin{equation*}
  \rho^{(1)}(\vec{r})
  =
  \left\langle
  \sum_{i=1}^N \delta(\vec{r} - \vec{r}_i)
  e^{-\beta \phi_\mathrm{ext}(\vec{r})}
  \right\rangle
\end{equation*}
which in the absence of particle interactions reduces to
\begin{equation}\label{eq:ideal-density}
  \rho^{(1)}(\vec{r})
  =
  \frac{
    \langle N \rangle e^{-\beta \phi_\mathrm{ext}(\vec{r})}
  }{
    \int e^{-\beta \phi_\mathrm{ext}(\vec{r}')} \, d\vec{r}'
  }
  =
  \frac{e^{-\beta (\mu - \phi_\mathrm{ext}(\vec{r}))}}{\Lambda^d}.
\end{equation}
We can express the grand potential for the non-interacting system as a functional of the external potential from the partition function \eqref{eq:ideal-grand-canonical-partition} as
\begin{equation*}
  \beta\Omega^\mathrm{id}
  =
  - \ln{\Xi^\mathrm{id}}
  =
  - \int \frac{e^{-\beta (\phi_\mathrm{ext}(\vec{r}) - \mu)}}{\Lambda^d} d\vec{r}
\end{equation*}
or in its dual form as a functional of density \eqref{eq:dft-grand-potential} as
\begin{equation*}
  \beta\Omega^\mathrm{id}
  =
  \beta \mathcal{F}^\mathrm{id}
  - \int \rho^{(1)}(\vec{r}) \beta \psi(\vec{r}) \, d\vec{r}.
\end{equation*}
Equating these two forms and rearranging we find the ideal part of the Helmholtz free energy as
\begin{align}\
  \beta \mathcal{F}^\mathrm{id}
  &=
  \int
  \left(
  \rho^{(1)}(\vec{r}) \beta (\phi_\mathrm{ext}(\vec{r}) - \mu)
  - \frac{e^{-\beta (\phi_\mathrm{ext}(\vec{r}) - \mu)}}{\Lambda^d}
  \right)
  \, d\vec{r}
  \nonumber \\ &=
  \int
  \rho^{(1)}(\vec{r})
  \left(
  \ln{(\Lambda^d \rho^{(1)}(\vec{r}))} - 1
  \right)
  \, d\vec{r}
  \label{eq:ideal-free-energy-functional}
\end{align}
using the ideal density \eqref{eq:ideal-density} in the final step.
The inhomogeneous ideal gas free energy density is thus identical to the homogeneous case \eqref{eq:ideal-free-energy-density} after replacing the global density with a local one.

\subsection{Thermodynamic potentials as generating functionals}

The fundamental thermodynamic relation describing an infinitesimal change in the Helmholtz free energy, i.e.\
\begin{equation*}
  dF = -S dT - p dV + \mu dN,
\end{equation*}
generalises to an inhomogeneous system as
\begin{equation*}
  \begin{split}
    \delta F
    =
    - S \delta T
    + \int \rho^{(1)}(\vec{r}) \delta \phi_\mathrm{ext}(\vec{r}) \, d\vec{r}
    + \int \mu \delta \rho^{(1)}(\vec{r}) \, d\vec{r}
  \end{split}
\end{equation*}
The change in the intrinsic free energy is then
\begin{equation}\label{eq:infinitesimal-free-energy}
  \begin{split}
    \delta \mathcal{F}
    &=
    \delta F
    - \int \delta \rho^{(1)}(\vec{r}) \phi_\mathrm{ext}(\vec{r}) \, d\vec{r}
    - \int \rho^{(1)}(\vec{r}) \delta \phi_\mathrm{ext}(\vec{r}) \, d\vec{r}
    \\ &=
    - S \delta T
    + \int \delta \rho^{(1)}(\vec{r}) \psi(\vec{r}) \, d\vec{r}.
  \end{split}
\end{equation}
By similar steps, or using the Legendre transform of the grand potential \eqref{eq:grand-potential-legendre-transform}, it follows that
\begin{equation}\label{eq:infinitesimal-grand-potential}
  \delta \Omega
  =
  - S \delta T
  - \int \rho^{(1)}(\vec{r}) \delta \psi(\vec{r}) \, d\vec{r}
\end{equation}
Hence, from functional differentiation of \eqref{eq:infinitesimal-free-energy} and \eqref{eq:infinitesimal-grand-potential} it follows that
\begin{align}
  \label{eq:psi-generator}
  \frac{\delta \mathcal{F}}{\delta \rho^{(1)}(\vec{r})}
  &=
  \psi(\vec{r})
  \\
  \label{eq:rho-generator}
  \frac{\delta \Omega}{\delta \psi(\vec{r})}
  &=
  - \rho^{(1)}(\vec{r})
\end{align}
i.e.\ the intrinsic free energy and grand potentials act as \emph{generating functionals} for the intrinsic chemical potential and density respectively.

Repeated functional differentiation of the thermodynamic potentials produces a whole hierarchy of correlation functions.
The hierarchy obtained from the grand potential is the density-density correlations which we already introduced in \eqref{eq:density-density-correlations};
these are generated by the grand potential as \cite{Hansen2013}
\begin{equation}\label{eq:density-density-generator}
  H^{(n)}(\vec{r}^n)
  =
  - \frac{
    \delta^n \beta \Omega
  }{
    \delta \beta\psi(\vec{r}_1) \cdots \delta \beta\psi(\vec{r}_n)
  }
  =
  \frac{
    \delta^{n-1} \rho^{(1)}(\vec{r}_1)
  }{
    \delta \beta\psi(\vec{r}_2) \cdots \delta \beta\psi(\vec{r}_n)
  }.
\end{equation}
The intrinsic free energy also generates a new hierarchy of correlation functions, however the contribution from the ideal part is not especially interesting.
We thus define the \emph{direct correlation functions} emerging from the excess part as
\begin{equation}\label{eq:direct-correlations}
  c^{(n)}(\vec{r}^n)
  =
  - \frac{
    \delta^n \beta \mathcal{F}^\mathrm{ex}
  }{
    \delta \rho^{(1)}(\vec{r}_1) \cdots \delta \rho^{(1)}(\vec{r}_n)
  }.
\end{equation}
These correlation functions form the basis of integral equation theories which we will outline in the next section.

\subsection{Integral equation theories}
\label{sec:oz-equation}

In this section we derive the Ornstein-Zernike equation which forms the basis of many theories of the liquid state.
This is an integral equation which connects the density-density and direct correlation functions through the chain rule of functional calculus, i.e.\
\begin{equation}\label{eq:oz-chain-rule}
  \delta(\vec{r}_1 - \vec{r}_2) =
  \int
  \frac{\delta \rho^{(1)}(\vec{r}_1)}{\delta \beta\psi(\vec{r}')}
  \frac{\delta \beta\psi(\vec{r}')}{\delta \rho^{(1)}(\vec{r}_2)}
  \, d\vec{r}'.
\end{equation}
The first term appearing in the integrand is simply the pair density-density correlation function $H^{(2)}$ via \eqref{eq:density-density-generator},
so we will require an explicit expression for the second term to proceed.

The first functional derivative of the ideal intrinsic free energy \eqref{eq:ideal-free-energy-functional} yields
\begin{equation*}
  \frac{
    \delta \beta \mathcal{F}^\mathrm{id}
  }{
    \delta \rho^{(1)}(\vec{r})
  }
  =
  \ln{(\Lambda^d \rho^{(1)}(\vec{r}))}.
\end{equation*}
To obtain the higher order functional derivatives it is helpful to write this as an integral with a delta function
\begin{equation*}
  \frac{
    \delta \beta \mathcal{F}^\mathrm{id}
  }{
    \delta \rho^{(1)}(\vec{r})
  }
  =
  \int \delta{(\vec{r}' - \vec{r})}
  \ln{(\Lambda^d \rho^{(1)}(\vec{r}'))} \, d\vec{r}',
\end{equation*}
so we can obtain the second derivative as
\begin{equation*}
  \frac{
    \delta^2 \beta \mathcal{F}^\mathrm{id}
  }{
    \delta \rho^{(1)}(\vec{r}) \delta \rho^{(1)}(\vec{r}')
  }
  =
  \frac{\delta(\vec{r}'-\vec{r})}{\rho^{(1)}(\vec{r})}.
\end{equation*}
%% Iterating this procedure gives us the $n$th functional derivative as
%% \begin{equation}
%%   \begin{split}
%%     \frac{
%%       \delta^n \beta \mathcal{F}^\mathrm{id}
%%     }{
%%       \delta \rho^{(1)}(\vec{r}_1) \cdots \delta \rho^{(1)}(\vec{r}_n)
%%     }
%%     &=
%%     \frac{
%%       \partial^{n-1} (\ln{(\Lambda^d \rho^{(1)}(\vec{r}))})
%%     }{
%%       \partial \rho^{(1)}(\vec{r})^{n-1}
%%     }
%%     \prod_{i=2}^n \delta(\vec{r}_i - \vec{r}_1)
%%     \\ &=
%%     (-1)^{n-1}
%%     \frac{(n-2)!}{\rho^{(1)}(\vec{r})^{n-1}}
%%     \prod_{i=2}^n \delta(\vec{r}_i - \vec{r}_1),
%%   \end{split}
%% \end{equation}
%% where the last line is valid for all $n \ge 2$.
From the free energy as a generating functional \eqref{eq:psi-generator} and the decomposition of the free energy into ideal and excess parts \eqref{eq:F-decomposition} it follows that
\begin{equation*}
  \begin{split}
    \beta \psi(\vec{r})
    =
    \frac{\delta \beta \mathcal{F}}{\delta \rho^{(1)}(\vec{r})}
    &=
    \frac{\delta \beta \mathcal{F}^\mathrm{id}}{\delta \rho^{(1)}(\vec{r})}
    + \frac{\delta \beta \mathcal{F}^\mathrm{ex}}{\delta \rho^{(1)}(\vec{r})}
    \\
    &=
    \ln{(\Lambda^d \rho^{(1)}(\vec{r}))} - c^{(1)}(\vec{r}),
  \end{split}
\end{equation*}
using the definition of the direct correlation function \eqref{eq:direct-correlations} in the latter step.
Further functional differentiation yields
\begin{equation*}\label{eq:intrinsic-chemical-potential-inverse-derivative}
  \frac{\delta \beta \psi(\vec{r})}{\delta \rho^{(1)}(\vec{r}')}
  =
  \frac{\delta(\vec{r} - \vec{r}')}{\rho^{(1)}(\vec{r}')}
  - c^{(2)}(\vec{r}, \vec{r}').
\end{equation*}
Inserting this expression into \eqref{eq:oz-chain-rule} gives
\begin{equation*}
  \begin{split}
    \delta(\vec{r}_1 - \vec{r}_2)
    &=
    \int
    H^{(2)}(\vec{r}_1, \vec{r}')
    \left(
    \frac{\delta(\vec{r}' - \vec{r}_2)}{\rho^{(1)}(\vec{r}')} -
    c^{(2)}(\vec{r}', \vec{r}_2)
    \right)
    \, d\vec{r}'
    %% \\ &=
    %% \rho^{(1)}(\vec{r}_1)
    %% \left(
    %% h^{(2)}(\vec{r}_1, \vec{r}_2) -
    %% c^{(2)}(\vec{r}_1, \vec{r}_2)
    %% \right) +
    %% \delta(\vec{r}_1 - \vec{r}_2) -
    %% \\ &\qquad
    %% \rho^{(1)}(\vec{r}_1)
    %% \int
    %% \rho^{(1)}(\vec{r}')
    %% h^{(2)}(\vec{r}_1, \vec{r}')
    %% c^{(2)}(\vec{r}', \vec{r}_2)
    %% \, d\vec{r}'
  \end{split}
\end{equation*}
which rearranges to give the Ornstein-Zernike equation
\begin{equation}
  h^{(2)}(\vec{r}_1, \vec{r}_2) =
  c^{(2)}(\vec{r}_1, \vec{r}_2) +
  \int
  \rho^{(1)}(\vec{r}')
  h^{(2)}(\vec{r}_1, \vec{r}')
  c^{(2)}(\vec{r}', \vec{r}_2)
  \, d\vec{r}',
\end{equation}
which is a classic result of liquid state theory (cf.\ Refs.\ \cite{OrnsteinPAS1914,Hansen2013,EvansAP1979}).

For a uniform liquid interacting through a spherically symmetric pair potential the Ornstein-Zernike equation becomes
\begin{equation}
  \begin{split}
    h^{(2)}(r)
    &=
    c^{(2)}(r) +
    \rho
    \int
    h^{(2)}(r)
    c^{(2)}(|\vec{r}' - \vec{r}|)
    \, d\vec{r}'
    \\ &=
    c^{(2)}(r) + \rho \, (h^{(2)} * c^{(2)})(r),
  \end{split}
\end{equation}
where $r = |\vec{r}_2 - \vec{r}_1|$ and $(f*g)(\vec{r})$ denotes a convolution between functions $f$ and $g$.
In Fourier space the convolution becomes a product so we obtain
\begin{equation*}
  \tilde{h}^{(2)}(\vec{k})
  =
  \tilde{c}^{(2)}(\vec{k}) +
  \rho \, \tilde{h}^{(2)}(\vec{k}) \tilde{c}^{(2)}(\vec{k})
\end{equation*}
where the tildes over a function denotes its Fourier transform.
This rearranges to give
\begin{equation*}
  \tilde{h}^{(2)}(\vec{k})
  =
  \frac{\tilde{c}^{(2)}(\vec{k})}{1 - \rho \tilde{c}^{(2)}(\vec{k})}
\end{equation*}
which gives the static structure factor \eqref{eq:static-structure-factor} as
\begin{equation*}
  S^{(2)}(\vec{k})
  =
  \rho \delta(\vec{k}) +
  \frac{1}{1 - \rho \tilde{c}^{(2)}(\vec{k})}.
\end{equation*}
If the pair direct correlation function is known it is thus straightforward to obtain the equation of state through the compressibility route \eqref{eq:compressibility-route}
\begin{equation*}
  \frac{\beta p}{\rho}
  =
  1 - \frac{1}{\rho} \int_0^\rho \tilde{c}^{(2)}(0) \rho' d\rho'.
\end{equation*}

The main task of integral equation approaches is to find approximate closures for $c^{(2)}$ and then solve the Ornstein-Zernike equation.
The process of determining the direct correlation functions is equivalent (at least formally) to finding its generating function $\mathcal{F}^\mathrm{ex}$.

\subsection{Generalised Ornstein-Zernike equations}

This section follows \cite{Barrat1988}.\todo{Check this reference and rewrite accordingly}

Thus for higher $n$ we have
\begin{equation}
  \begin{split}
    c^{(n)}(\vec{r}^n)
    &=
    \frac{
      \delta^{n-1} c^{(1)}(\vec{r}_1)
    }{
      \delta \rho^{(1)}(\vec{r}_2) \cdots \delta \rho^{(1)}(\vec{r}_n)
    }
    \\ &=
    (-1)^n
    \frac{(n-2)!}{\rho^{(1)}(\vec{r}_1)^{n-1}}
    \prod_{i=2}^n \delta(\vec{r}_i - \vec{r}_1)
    - \frac{
      \delta^{n-1} \beta\psi(\vec{r}_1)
    }{
      \delta \rho^{(1)}(\vec{r}_2) \cdots \delta \rho^{(1)}(\vec{r}_n)
    }
  \end{split}
\end{equation}
or
\begin{equation}
  \frac{\delta^{n-1} \beta\psi(\vec{r}_1)}{\delta \rho^{(1)}(\vec{r}_2) \cdots \delta \rho^{(1)}(\vec{r}_n)}
  =
  (-1)^n
  \frac{(n-2)!}{\rho^{(1)}(\vec{r}_1)^{n-1}}
  \prod_{i=2}^n \delta(\vec{r}_i - \vec{r}_1)
  - c^{(n)}(\vec{r}^n)
\end{equation}

\begin{equation*}
  H^{(n)}(\vec{r}^n)
  =
  \frac{
    \delta^{n-1} \rho^{(1)}(\vec{r}_1)
  }{
    \delta \beta\psi(\vec{r}_2) \cdots \delta \beta\psi(\vec{r}_n)
  }.
\end{equation*}

Defining
\begin{equation*}
  K^{(n)}(\vec{r}^n)
  =
  \frac{
    \delta^{n-1} \beta\psi(\vec{r}_1)
  }{
    \delta \rho^{(1)}(\vec{r}_2) \cdots \delta \rho^{(1)}(\vec{r}_n)
  }
\end{equation*}
we have
\begin{equation*}
  \frac{
    \delta K^{(n)}(\vec{r}^n)
  }{
    \delta \rho^{(1)}(\vec{r}_{n+1})
  }
  =
  K^{(n+1)}(\vec{r}^{n+1}).
\end{equation*}
and
\begin{equation*}
  \begin{split}
    \frac{\delta H^{(n)}(\vec{r}^n)}{\delta \rho^{(1)}(\vec{r}_{n+1})}
    &=
    \int
    \frac{\delta H^{(n)}(\vec{r}^n)}{\delta \psi(\vec{r}')}
    \frac{\delta \psi(\vec{r}')}{\delta \rho^{(1)}(\vec{r}_{n+1})}
    \, d\vec{r}' \\
    &=
    \int
    H^{(n+1)}(\vec{r}^n, \vec{r}')
    K^{(2)}(\vec{r}', \vec{r}_{n+1})
    \, d\vec{r}' \\
    &=
    H^{(n+1)} * K^{(2)}(\vec{r}^{n+1}).
  \end{split}
\end{equation*}
In this form the Ornstein-Zernike equation can be written.
\begin{equation*}
  \begin{split}
    \delta(\vec{r}_1 - \vec{r}_2)
    &=
    \int
    \frac{\delta \rho^{(1)}(\vec{r}_1)}{\delta \psi(\vec{r}')}
    \frac{\delta \psi(\vec{r}')}{\delta \rho^{(1)}(\vec{r}_2)}
    \, d\vec{r}' \\
    &=
    \int
    H^{(2)}(\vec{r}_1, \vec{r}') K^{(2)}(\vec{r}', \vec{r}_2)
    \, d\vec{r}' \\
    &=
    H^{(2)} * K^{(2)} (\vec{r}^2)
  \end{split}
\end{equation*}
Taking functional derivatives of this expression gives us a hierarchy of generalised Ornstein-Zernike equations.
For example, the next equation in the hierarchy is
\begin{equation*}
  H^{(2)} * K^{(3)} (\vec{r}^3) +
  H^{(3)} * K^{(2)} * K^{(2)} (\vec{r}^3)
  = 0
\end{equation*}
The next functional derivative
\begin{equation}
  \begin{split}
  H^{(2)} * K^{(4)} (\vec{r}^4) & \\
  + \; 3 H^{(3)} * K^{(3)} * K^{(2)} (\vec{r}^4) & \\
  + \; H^{(4)} * K^{(2)} * K^{(2)} * K^{(2)} (\vec{r}^4)
  &= 0
  \end{split}
\end{equation}
And the next one%
\todo{Can we find a general formula? I notice some constraints on the indices: the sum of the indices $\{m\}$ in the $K^{(m)}$ terms must add up to $n-1$ so that the right number of independent variables are returned (the extra one is provided by the $H^{(l)}$ function giving the $\vec{r}^n$ total.}
\begin{equation}
  \begin{split}
    H^{(2)} * K^{(5)} (\vec{r}^5) & \\
    + \; 3 H^{(3)} * K^{(3)} * K^{(3)} (\vec{r}^5) & \\
    + \; 4 H^{(3)} * K^{(4)} * K^{(2)} (\vec{r}^5) & \\
    + \; 6 H^{(4)} * K^{(3)} * K^{(2)} * K^{(2)} (\vec{r}^5) & \\
    + \; H^{(5)} * K^{(2)} * K^{(2)} * K^{(2)} * K^{(2)} (\vec{r}^5)
    &=
    0
  \end{split}
\end{equation}

\subsection{Fundamental measure theory}

\todo{Finish this section}
FMT: a recent review \cite{Roth2010}.
Also mention \cite{Lutsko2010} for more of a focus on crystallisation.
This exposition follows \cite{Roth2010} mainly.

%\subsection{Many-body correlations from fundamental measure theory}

\begin{itemize}
\item $d+1$ weight functions
\item Exact decomposition of Mayer-f bond
\end{itemize}

We have 4 scalar weight functions
\begin{subequations}
  \begin{align}
    \omega_3^i(\vec{r}) &= \Theta(R_i - r), \\
    \omega_2^i(\vec{r}) &= \delta(R_i - r), \\
    \omega_1^i(\vec{r}) &= \frac{\omega_2^i(\vec{r})}{4\pi R_i}, \\
    \omega_0^i(\vec{r}) &= \frac{\omega_2^i(\vec{r})}{4\pi R_i^2},
  \end{align}
\end{subequations}
and 2 vector weight functions
\begin{subequations}
  \begin{align}
    \vec{\omega}_2^i(\vec{r}) &=
    \frac{\vec{r}}{r} \delta(R_i - r), \\
    \vec{\omega}_1^i(\vec{r}) &=
    \frac{\vec{\omega}_1^i(\vec{r})}{4\pi R_i}.
  \end{align}
\end{subequations}

Fourier transforming weight functions:
\begin{equation}
  \widetilde{\omega}_\alpha(\vec{k}) =
  \int \omega_\alpha (\vec{r}) e^{-i \vec{k}\cdot\vec{r}} \, d\vec{r}
\end{equation}
%% \begin{align}
%%   \widetilde{\omega}_3(\vec{k}) &=
%%   4\pi \frac{\sin{kR} - kR\cos{kR}}{k^3} \\
%%   \widetilde{\omega}_2(\vec{k}) &=
%%   4\pi R^2 \frac{\sin{kR}}{kR} \\
%%   \widetilde{\vec{\omega}}_2(\vec{k}) &=
%%   - i \vec{k} \widetilde{\omega}_3(\vec{k})
%% \end{align}
\begin{equation}
  \begin{aligned}
    c_{22}(\vec{r} = \vec{r}_2 - \vec{r}_1) &=
    \frac{1}{(2\pi)^3}
    \iint
    e^{-i (\vec{k}_1\cdot\vec{r}_1 + \vec{k}_2\cdot\vec{r}_2)}
    \widetilde{\omega}_2(\vec{k}_1)
    \widetilde{\omega}_2(\vec{k}_2)
    \delta{(\vec{k}_1 + \vec{k}_2)}
    \, d\vec{k}_1 d\vec{k}_2 \\
    &=
    \frac{1}{(2\pi)^3}
    \int
    e^{i \vec{k}_1 \cdot (\vec{r}_2 - \vec{r}_1)}
    \widetilde{\omega}_2(\vec{k}_1)
    \widetilde{\omega}_2(-\vec{k}_1)
    \, d\vec{k}_1 \\
    &=
    \frac{1}{(2\pi)^3}
    \int
    e^{i \vec{k} \cdot \vec{r}}
    \widetilde{\omega}_2(\vec{k})^2
    \, d\vec{k}
    =
    \frac{8\pi^2 R^2}{(2\pi)^3}
    \int
    e^{i \vec{k} \cdot \vec{r}}
    \left( \frac{\sin{kR}}{k} \right)^2
    \, d\vec{k} \\
    &=
    \frac{R^2}{\pi}
    \int
    e^{i k r \cos\theta}
    \left( \frac{\sin{kR}}{k} \right)^2
    \, 2\pi k^2 \sin\theta \, dr d\theta \\
    &=
    2 R^2
    \int
    e^{i k r \cos\theta}
    \sin^2{(kR)}
    \, \sin\theta \, dr d\theta
  \end{aligned}
\end{equation}

Following \cite{Rosenfeld1990} we have
\begin{equation}\label{eq:fmt-direct-correlations}
  \begin{aligned}
    c^{(n)}(\vec{r}^n) &=
    - \frac{
      \delta^n \beta F_{ex}
    }{
      \delta \rho^{(1)}(\vec{r}_1) \cdots \delta \rho(\vec{r}_n)
    }
    \\ &=
    - \sum_{\alpha_1, \alpha_2, \cdots, \alpha_n}
    \int d\vec{r}'
    \prod_{i=1}^n \Big( \omega_{\alpha_i}(\vec{r}' - \vec{r}_i) \Big)
    \partial^n_{\alpha_1, \alpha_2, \cdots, \alpha_n} \beta\Phi_{ex}(\vec{r}')
  \end{aligned}
\end{equation}
where
\begin{equation*}
  \partial^n_{\alpha_1, \alpha_2, \cdots, \alpha_n} \beta\Phi_{ex}(\vec{r}') =
  \left.
  \frac{\partial^n \beta\Phi_{ex}}{\partial n_{\alpha_1} \partial n_{\alpha_2} \cdots \partial n_{\alpha_n}}
  \right|_{\{n_\alpha\} = \{n_\alpha(\vec{r}')\}}.
\end{equation*}
At uniform density $\partial^n_{\alpha_1, \alpha_2, \cdots, \alpha_n} \beta\Phi_{ex}$ is position independent, so \eqref{eq:fmt-direct-correlations} becomes
\begin{equation}\label{eq:fmt-direct-correlations-uniform-density}
  \begin{aligned}
    c^{(n)}(\vec{r}^n) &=
    - \sum_{\alpha_1, \alpha_2, \cdots, \alpha_n}
    \partial^n_{\alpha_1, \alpha_2, \cdots, \alpha_n} \beta\Phi_{ex}
    \int d\vec{r}'
    \prod_{i=1}^n \Big( \omega_{\alpha_i}(\vec{r}' - \vec{r}_i) \Big) \\
    &=
    - \sum_{\alpha_1, \alpha_2, \cdots, \alpha_n}
    \partial^n_{\alpha_1, \alpha_2, \cdots, \alpha_n} \beta\Phi_{ex} \;
    \Big(
    \omega_{\alpha_1} \otimes \omega_{\alpha_2} \otimes \cdots \otimes \omega_{\alpha_n}
    (\vec{r}^n)
    \Big)
  \end{aligned}
\end{equation}
where the $\otimes$-notation in the latter step denotes the $n$-body convolution.
For example, the two body convolution would be written%
\todo{This is not the standard convolution! See \cite{Rosenfeld1997}.}
%\marginfootnote{The `standard' convolution, i.e.\ $f \otimes g(\vec{r} \equiv \vec{r}_1 - \vec{r}_2)$, is recovered after transforming to the new integration variable $\vec{r}'' = \vec{r}' - \vec{r}_1$.}
\begin{equation*}
  f \otimes g(\vec{r}_1, \vec{r}_2) =
  \int d\vec{r}' f(\vec{r}' - \vec{r}_1) g(\vec{r}' - \vec{r}_2).
\end{equation*}
\todo{Also, the standard convolution symbol is an asterisk, not an outer product.}
Applying the convolution theorem allows the Fourier transform of \eqref{eq:fmt-direct-correlations-uniform-density} to be written rather succinctly as
\begin{equation}
  \tilde{c}^{(n)}(\vec{k}^n) =
  - \sum_{\alpha_1, \alpha_2, \cdots, \alpha_n}
  \partial^n_{\alpha_1, \alpha_2, \cdots, \alpha_n} \beta\Phi_{ex} \;
  \left( \prod_{i=1}^n \widetilde{\omega}_{\alpha_i}(\vec{k}_i) \right)
  \delta(\vec{k}_1 + \vec{k}_2 + \cdots + \vec{k}_n).
\end{equation}
The delta function enforces the `ring' condition $\sum_{i=1}^n \vec{k}_i = 0$ which emerges from translational symmetry of the weight functions%
\marginfootnote{In the previous note this occurred by change of integration variables to a relative displacement, which was only possible because of this translational symmetry.}[-3cm],
reducing the dimensionality of the domain by $d$.
A further $d(d-1)/2$ degrees of freedom%
\marginfootnote{This many degrees of freedom can be removed for general $n \ge d$, but we expect fewer for $n < d$.
  For example, $n=2$ arrangements (a dimer) are isomorphic to a line so they possess $d-1$ rotational degrees of freedom.}
can be removed by exploiting rotational symmetry.

From \cite{Rosenfeld1990}:
\begin{align}
  \widetilde{\omega}_0 &= \cos{(kR)} \\
  \widetilde{\omega}_1 &= 2\frac{\sin{(kR)}}{k} \\
  \widetilde{\omega}_0 &= \cos{(kR)}
\end{align}

White bear I:
\begin{equation}
  c^{(2)}(r)
\end{equation}

\begin{tcolorbox}[title=Percus-Yevick theory in hard spheres]
  Virial route
  \begin{equation}
    \frac{\beta p^\mathrm{PY-V}}{\rho}
    =
    \frac{1 + 2 \eta + 3 \eta^2}{(1 - \eta)^2}
  \end{equation}
  Compressibility route
  \begin{equation}\label{eq:pyc-pressure}
    \frac{\beta p^\mathrm{PY-C}}{\rho}
    =
    \frac{1 + \eta + \eta^2}{(1 - \eta)^3}
  \end{equation}
  The interpolation
  \begin{equation}\label{eq:cs-pressure}
    \begin{split}
      \frac{\beta p^\mathrm{CS}}{\rho}
      &=
      \frac{1}{3} \frac{\beta p^\mathrm{PY-V}}{\rho}
      + \frac{2}{3} \frac{\beta p^\mathrm{PY-C}}{\rho}
      \\ &=
      \frac{1 + \eta + \eta^2 - \eta^3}{(1-\eta)^3},
    \end{split}
  \end{equation}
\end{tcolorbox}


%% \begin{SCfigure}[H]
%%   \missingfigure[figwidth=\linewidth]{}
%%   \caption{Static structure factor for convolution, Rosenfeld and White Bear closures.}
%% \end{SCfigure}

%% \begin{SCfigure}[H]
%%   \missingfigure[figwidth=\linewidth]{}
%%   \caption{Triplet static structure factors for convolution, Rosenfeld and White Bear closures.}
%% \end{SCfigure}

\subsection{Superposition and convolution approximations}

\todo{Finish this section}
In the Kirkwood superposition approximation \cite{Kirkwood1935} many-body correlations are expressed as pairwise products of the two-body correlation function, i.e.
\begin{equation}
  g^{(n)}(\vec{r}^n) =
  \prod_{i < j} g^{(2)}(\vec{r}_i, \vec{r}_j),
\end{equation}
which correctly satisfies the hard-core condition, but violates the sum rule
\begin{equation}
  \begin{aligned}
    \rho^{(n)}(\vec{r}^n) &=
    \frac{1}{\Xi} \sum_{N=n}^\infty \frac{z^N}{(N-n)!} \int e^{-\beta U_N} \, d\vec{r}^{(N-n)} \\
    &=
    \int d\vec{r}_n \left(
    \frac{1}{\Xi} \sum_{N=n}^\infty \frac{z^N}{(N+1 - (n+1))!} \int e^{-\beta U_N} \, d\vec{r}^{(N-(n+1))}
    \right) \\
    &=
    \int d\vec{r}_n \left(
    \frac{1}{\Xi} \sum_{N=n}^\infty \frac{z^N}{(N - (n+1))!} \int e^{-\beta U_N} \, d\vec{r}^{(N-(n+1))}
    \right) \\
    &=
    \frac{1}{N-n}
    \int \rho^{(n+1)}(\{\vec{r}^n, \vec{r}_{n+1}\}) d\vec{r}_{n+1},
  \end{aligned}
\end{equation}
and the related convolution approximation \cite{Jackson1962,Ichimaru1970,Barrat1988}%
\todo{Check this expression is correct - it almost certainly is not.}
\begin{equation}
  S^{(n)}(\vec{k}^n) =
  (1 + \tilde{c}^{(n)}(\vec{k}^n))
  \prod_{i < j} S^{(2)}(\vec{k}_i, \vec{k}_j)
\end{equation}
satisfies the sum rule but fails to satisfy the hard-core condition.

%% In equilibrium
%% \begin{equation}
%%   c^{(n)}(\vec{r}^n) =
%%   \left.
%%   \frac{\delta^n \beta F_{ex}}{\delta \rho(\vec{r}_1)\delta \rho(\vec{r}_2) \cdots \delta \rho(\vec{r}_n)}
%%   \right|_{\rho(\vec{r})=\rho}
%% \end{equation}

\subsection{Variational formulation}

In density functional theory the ...
\todo{Finish this section}

Properties:
\begin{equation}
  \Omega[\rho(\vec{r})] > \Omega[\rho^{(1)}(\vec{r})]
\end{equation}
and
\begin{equation}\label{eq:dft-equilibrium}
  \left.
  \frac{
    \delta \Omega[\rho(\vec{r})]
  }{
    \delta \rho
  }
  \right|_{\rho(\vec{r})=\rho^{(1)}(\vec{r})} = 0
\end{equation}

\begin{align*}
  \Omega &\to \Omega[\rho]
  \\
  \mathcal{F}^\mathrm{id} &\to \mathcal{F}^\mathrm{id}[\rho]
  \\
  \mathcal{F}^\mathrm{ex} &\to \mathcal{F}^\mathrm{ex}[\rho]
\end{align*}

Which gives the equilibrium density as
\begin{equation}
  \rho^{(1)}(\vec{r})
  =
  \frac{
    \exp{\left(\beta\psi(\vec{r}) + c^{(1)}(\vec{r})\right)}
  }{ \Lambda^d }
\end{equation}


\section{Summary}

\begin{SCtable}
  \begin{minipage}[b]{\linewidth}
    \centering
    \begin{tabular}{cc}
      \toprule
      \multicolumn{2}{c}{Dual hierarchies} \\
      \midrule
      correlation function & dual function \\
      \midrule
      distribution $g^{(n)}$ \eqref{eq:n-particle-distribution} &
      cluster $h^{(n)}$ \eqref{eq:cluster-correlation-functions} \\
      density moments $\langle \hat{\rho}_1 \cdots \hat{\rho}_n \rangle$ \eqref{eq:density-moments} &
      density-density $H^{(n)}$ \eqref{eq:density-density-correlations} \\
      Boltzmann weight $e^{-\beta U_n}$ &
      Ursell $W^{(n)}$ \eqref{eq:ursell-functions} \\
      \midrule
      \multicolumn{2}{c}{Other correlation functions} \\
      \midrule
      \multicolumn{2}{c}{$n$-particle density $\rho^{(n)}$ \eqref{eq:n-particle-density}} \\
      \multicolumn{2}{c}{static structure factor $S^{(n)}$ \eqref{eq:static-structure-factor}} \\
      \multicolumn{2}{c}{direct $c^{(n)}$ \eqref{eq:direct-correlations}} \\
      \bottomrule
    \end{tabular}
  \end{minipage}
  \caption{Summary of the various correlation functions in liquid state theory.}
  \label{table:correlation-functions}
\end{SCtable}

We have seen that integral geometry provides the formalism for describing sizes, and thus defines an \emph{ansatz} for strictly extensive variables.
Specifically, such variables must be a linear combination of the $(d+1)$ intrinsic volumes, which are the only meaningful size measures in $d$-dimensions in the sense of rigid-motion invariance, additivity and continuity.
We expect in general that thermodynamic potentials are extensive in the thermodynamic limit, so this formalism provides a good starting point for their description.
However, it is worth noting that extensive quantities could contain subleading terms which disappear in the thermodynamic limit; these would not be captured by an integral geometric \emph{ansatz}.

We introduced liquid state theory, with a particular focus on the various correlation functions which are summarised in Table \ref{table:correlation-functions}.
A key result for hard spheres was the highly accurate Carnahan-Starling equation of state \eqref{eq:cs-pressure}, which we will use throughout to treat the hard sphere liquid.
We saw how the intrinsic volumes could be applied to liquid state theory to construct accurate free energy functionals for hard particle systems.
We obtained the morphometric approach, an expansion of the chemical potential with the properties of sizes outlined above, as a special case of this theory.
In chapters \ref{chapter:morphometric-framework} and \ref{chapter:resummation} we will provide additional justifications of this approach, and we will apply it to predict the concentrations of local structures in chapter \ref{chapter:morphometric-applications}.

In the next chapter we will discuss the phenomenology of supercooled liquids and the glass transition, which will provide the key context for the treatment of the high density hard sphere liquid in subsequent chapters.

\ifdefined\includebibliography
  \newgeometry{margin=1in}
  \printbibliography
\fi

\end{document}
