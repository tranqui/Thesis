\documentclass[12pt]{report}
\usepackage{preamble}
\setcounter{chapter}{5}

\begin{document}
\chapter{Drying kinetics and nucleation of aerosol droplets}
A shorter chapter.

\section{Motivation}
Connection with rest of thesis: here we attempt ot model nucleation of salt crystals inside of aerosol droplets, whereas in previous chapters we will have attempted to model nucleation of hard spheres.
Presently, the nucleation rates of the droplets are taken from phenomenological fits to available data assuming classical nucleation theory.
A connection with the morphometric work could be made if the applications to nucleation in hard spheres is succesful: this could provide a route to a proper first-principles treatment of nucleation in aerosols avoiding fitting parameters.
As optimistic as this scenario is, it is worth making \emph{some} connection.

\section{Diffusion model for a drying droplet}
\subsection{Evolution equation}
A moving boundary problem with diffusion
\subsection{Treating evaporation}
\subsection{Evolution of the concentration profile}

\section{Nucleation model}
\subsection{Classical nucleation theory}
\subsection{Density dependence of nucleation rate}
\subsection{Decay rates with droplet evolution}

\section{Appendix: finite difference methods for numerical integration}
Second-order scheme (Crank-Nicolson)

\end{document}
