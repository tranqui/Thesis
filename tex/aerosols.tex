\documentclass[12pt]{report}
\usepackage{preamble}
\setcounter{chapter}{5}
\graphicspath{{../img/}}

\begin{document}
\chapter{Drying kinetics and nucleation of aerosol droplets}
A shorter chapter.

\section{Motivation}
Connection with rest of thesis: here we attempt ot model nucleation of salt crystals inside of aerosol droplets, whereas in previous chapters we will have attempted to model nucleation of hard spheres.
Presently, the nucleation rates of the droplets are taken from phenomenological fits to available data assuming classical nucleation theory.
A connection with the morphometric work could be made if the applications to nucleation in hard spheres is succesful: this could provide a route to a proper first-principles treatment of nucleation in aerosols avoiding fitting parameters.
As optimistic as this scenario is, it is worth making \emph{some} connection.

\section{Diffusion model for a drying droplet}
A moving boundary problem with diffusion.

\subsection{Evolution equation}

We consider a droplet of radius $R(t)$ and consider concentration profiles at radius $r$ from its centre.
The droplet consists of a binary fluid, with solute and solvent (mass) concentration profiles labelled $c_s(r)$ and $c_w(r)$ respectively.
The mass fraction of each component is written
\begin{equation}\label{eq:mass-fraction}
  \phi_i(r) = \frac{c_i(r)}{\sum_j c_j(r)} = \frac{c_i(r)}{c_w(r) + c_s(r)} \qquad i=\{w,s\}
\end{equation}
Clearly $\phi_w = 1 - \phi_s$ as they must sum to unity, so we can choose either to describe the fluid;
we will henceforth use solute mass fraction $\phi_s$.
The density of the droplet is
\begin{equation}
  \rho(r) = \sum_i c_i(r) = c_w(r) + c_s(r) = \frac{c_s(r)}{\phi_s(r)},
\end{equation}
but $c_s$ and $\phi_s$ are related by an equation of state so we could equivalently write:
\begin{equation}
  \rho(r) = \frac{c_s(r)}{\phi_s(c_s(r))} = \frac{c_s(\phi_s(r))}{\phi_s(r)}.
\end{equation}
We use an empirical form for this density by fitting it to a polynomial i.e. the leading order terms in its Taylor series expansion
\begin{equation}
  \rho(c_s) = \sum_{n=0}^{\infty} \frac{\rho^{(n)}(c_s^{(0)})}{n!} (c_s - c_s^{(0)})^n,
\end{equation}
where $c_s^{(0)}$ is the reference state point with known density and concentrations.
While a high order polynomial would normally be used in the full numerical calculations, we will
truncate this series to its linear term in order to estimate leading order terms in evolution
equations giving
\begin{equation}\label{eq:density-expansion}
  \rho(c_s) = \rho(c_s^{(0)}) + \rho'(c_s^{(0)}) \Delta c_s + \mathcal{O}(\Delta c_s^2)
\end{equation}
where $\Delta c_s = c_s - c_s^{(0)}$.
With any order of polynomial it follows that
\begin{equation}\label{eq:linear-density-regime}
  \frac{\partial \rho}{\partial t}
  = \frac{\partial \rho}{\partial \phi_s} \frac{\partial \phi_s}{\partial t}
  = \frac{\partial \rho}{\partial c_s} \frac{\partial c_s}{\partial t}
\end{equation}

The concentration of solute evolves according to the diffusion equation
\begin{equation}\label{eq:diffusion-equation}
  \frac{\partial c_s}{\partial t}
  = D_s \nabla^2 c_s
  = D_s \frac{1}{r^2} \frac{\partial}{\partial r} \left( r^2 \frac{\partial c_s}{\partial r} \right)
\end{equation}
The total solute mass is obtained from this concentration profile as
\begin{equation}\label{eq:solute-mass-equation}
  m_s(t) = 4\pi \int_0^{R(t)} c_s(r,t) r^2 \, dr.
\end{equation}
Differentiating this expression using Leibniz's rule gives
\begin{equation}
  \begin{aligned}
    \frac{dm_s}{dt} &=
    4\pi \left(
    c_s(R) R^2 \frac{dR}{dt} + \int_0^R \frac{\partial c_s}{\partial t} r^2 dr
    \right) \\
    &=
    4\pi R^2 \left(
    c_s(R) \frac{dR}{dt} + D_s \left.\frac{\partial c_s}{\partial r}\right|_{r=R}
    \right),
  \end{aligned}
\end{equation}
where we used \eqref{eq:diffusion-equation} in the final step.
Assuming there is no solute flux outside the droplet $m_s = const$ we obtain the outer boundary condition
\begin{equation}
  \left.\frac{\partial c_s}{\partial r}\right|_{r=R} = -\frac{c_s(R)}{D_s} \frac{dR}{dt}
\end{equation}

The total droplet mass is obtained from the internal density profile as
\begin{equation}\label{eq:mass-equation}
  m(t) = 4\pi \int_0^{R(t)} \rho(r,t) r^2 \, dr.
\end{equation}
Differentiating this expression using Leibniz's rule gives
\begin{equation}
  \frac{dm}{dt} = 4\pi \left(
  \rho(R) R^2 \frac{dR}{dt} + \int_0^R \frac{\partial \rho}{\partial t} r^2 dr
  \right),
\end{equation}
and rearranging for the evolution of the boundary gives
\begin{equation}
  \frac{dR}{dt} =
  \frac{1}{\rho(R) R^2} \left(
  \frac{1}{4\pi} \frac{dm}{dt} - \int_0^R \frac{\partial \rho}{\partial t} r^2 dr
  \right).
\end{equation}
Writing the density in terms of the concentration profile and using \eqref{eq:diffusion-equation} gives
\begin{equation}\label{eq:radial-evolution}
  \frac{dR}{dt} =
  \frac{\phi_s(c_s(R))}{c_s(R)} \frac{1}{R^2} \left(
  \frac{1}{4\pi} \frac{dm}{dt} -
  D_s \int_0^R
  \frac{\partial \rho}{\partial c_s}
  \frac{\partial}{\partial r} \left( r^2 \frac{\partial c_s}{\partial r} \right) dr
  \right).
\end{equation}
%% In the linear density regime given by equation \eqref{eq:linear-density-regime} this expression reduces to
%% \begin{equation}
%%   \frac{dR}{dt} =
%%   \frac{\phi_s(c_s(R))}{c_s(R)} \left(
%%   \frac{1}{4\pi R^2} \frac{dm}{dt} -
%%   \rho'(c_s^{(0)}) \left.\frac{\partial c_s}{\partial r}\right| _{r=R}
%%   + \mathcal{O}(\Delta c_s)
%%   \right).
%% \end{equation}

\subsection{Updated notation}

Book describes 6 vaporisation models that differ primarily by their treatment of heat/temp of fluid phase.
In order of increasing complexity these are:
\begin{enumerate}
\item const $T$ (enough to get $\Delta R^2 \propto t$ law)
\item Uniform but varying $T$
\item Conduction limit
\item Effective-conduction: treats internal circulation/convection of heat in an ad hoc manner
\item Vortex models
\item Navier-Stokes: in principle exact
\end{enumerate}

Some general notes:
\begin{itemize}
\item Generally, the thermal diffusivity of gas phase is much larger than that in the liquid phase (except near a critical point where physical properties become identical).
  Therefore transient liquid heating takes longer, and can often treat gas-phase as quasi-steady.
\item Internal heat circulation ignored in first three models above.
\item Gas has mass diffusion $\sim$ heat diffusion whereas in the liquid mass diffusion is much slower than heat diffusion
\end{itemize}


\subsection{Evaporation condition}

Kulmala equation for mass flow:
\begin{equation}
  I(R, \phi_s)%; \; T_\infty, S_\infty, P)
  \equiv \frac{dm}{dt}
  \simeq -2\pi \, S_h \, R \frac{S_{\infty} - a_w(\phi_s(R))}{\kappa_1(\phi_s) + \kappa_2(R, \phi_s)}
\end{equation}
with
\begin{align}
  \kappa_1(\phi_s) &= \frac{R_{mol} \, T_\infty}{M_w \, \beta_M \, D_w^{(air)} \, p_{eq} \, A(\phi_s)} \\
  \kappa_2(R, \phi_s) &= \frac{L^2 \, M_w}{K \, R_{mol} \, \beta_T \, T_\infty^2} a_w(\phi_s(R))
\end{align}
The water activity $a_w$ is a fitted function of $\phi_s$ using experimental data.
The constants I used are given as
\begin{itemize}
\item Sherwood number: $S_h \simeq 2.15479$
\item Relative humidity of gas phase: $S_\infty = 0$ (dry experiments)
\item Gas temperature: $T_\infty$ (\SI{}{\kelvin})
\item Gas pressure: $p(T_\infty) \simeq \frac{T_\infty}{\SI{293}{\kelvin}} \SI{101325}{\newton}$
\item Gas thermal conductivity (nitrogen): $K(T_\infty) = \left(\sum_{n=0}^2 \alpha_n T_\infty^n \right) \SI{}{\watt\per\meter\per\kelvin}$
  \begin{itemize}
  \item $\alpha_0 = \SI{3.95e-4}{}$
  \item $\alpha_1 = \SI{9.805e-5}{\per\kelvin}$
  \item $\alpha_2 = \SI{-4.3032e-8}{\per\square\kelvin}$
  \end{itemize}
\item Molar gas constant: $R_{mol} = \SI{8.314}{\joule\kelvin\per\mol}$
\item $M_w = \SI{0.018016}{\kilogram\per\mol}$
\item Diffusion constant of solvent through gas:
  $D_w^{(air)} \simeq \left(\frac{T_\infty}{\SI{273.15}{\kelvin}}\right)^{1.81} \frac{T_\infty}{\SI{293}{\kelvin}} \, \SI{2.19e-5}{\square\meter\per\second}$
\item Equilibrium water vapour pressure: $p_{eq} = \SI{2338.77}{\newton\per\meter}$
\item Stefan flow correction: $A(\phi_s) = 1 + \frac{S_\infty + a_w(\phi_s)}{2} \frac{p_{eq}}{p(T_\infty)}$
\item Latent heat of evaporation (water): $L = \SI{2455595}{\joule\per\kilogram}$
\item Mass flux correction: $\beta_M \simeq 0.996175$
\item Heat flux correction: $\beta_T \simeq 0.995228$
\end{itemize}

\subsubsection{Vapour phase analysis}

Starting from the continuity equation\marginfootnote{A key reference for this section is: Sirignano, W. A. \emph{Fluid dynamics and transport of droplets and sprays - Second edition} (2010).}:
\begin{equation}\label{eq:continuity-eqn}
  \frac{\partial \rho}{\partial t}
  + \vec{\nabla} \cdot \vec{j} = 0,
\end{equation}
where the current is
\begin{equation}
  \vec{\nabla} \cdot \vec{j} =
  \frac{1}{r^2} \frac{\partial (r^2 \rho u)}{\partial r}
\end{equation}
If we assume a steady state the time derivative in \eqref{eq:continuity-eqn} disappears leaving
\begin{equation}
  r ^2 \rho u = \textrm{constant} = \frac{\dot{m}}{4\pi}.
\end{equation}
For motion of the evaporating solvent we have
\begin{align}
  \frac{\partial (\rho\phi_s)}{\partial t}
  + \vec{\nabla} \cdot \vec{j_s} &= 0, \\
  \vec{\nabla} \cdot \vec{j}_s &=
  \rho (u \phi_s - D \phi'_s) \\
  \frac{\partial (\rho\phi_s)}{\partial t}
  + \frac{1}{r^2} \frac{\partial (r^2 \rho u \phi_s)}{\partial r}
  - \frac{1}{r^2} \frac{\partial (r^2 \rho D \phi'_s)}{\partial r} &= 0
\end{align}

We wind up with the ordinary differential equation\marginfootnote{We are using $\phi$ to mean mass fraction, but this could get confused with terms elsewhere in the thesis.
  Perhaps a glossary of terms at the start also?}[-1cm]:
\begin{equation}
  (\phi(r) - \phi_\infty) \frac{\dot{m}}{4\pi r^2 \rho D}
    = \frac{\partial \phi}{\partial r}
\end{equation}
Integrating this expression gives\marginfootnote{The easiest way to do this is to change to intermediate variable $\Delta\phi(r) = \phi(r) - \phi_\infty$, then multiply through by the integrating factor $\exp{\left(-\int_r^\infty \frac{\dot{m}}{4\pi (r')^2 \rho D} \, dr'\right)}$.}:
\begin{equation}
  \phi(r) - \phi_\infty =
  e^{-\frac{\dot{m}}{4\pi} \int_r^\infty \frac{dr'}{\rho D (r')^2}}
\end{equation}
Taking $\rho$ and $D$ as constants we obtain\marginfootnote{I think we need to subtract 1 from the right hand side to make this properly normalised: we don't want any evaporation when there is a homogeneous vapour profile.}[1cm]
\begin{equation}
  \phi(r) - \phi_\infty =
  e^{-\frac{\dot{m}}{4\pi \rho D r}}
\end{equation}
This gives us our final expression for quasistatic evaporation:
\begin{equation}
  \dot{m} = - 4\pi \rho D R \ln{(\phi(R) - \phi_\infty)}
\end{equation}

\subsection{Quasistatic expressions}

In the quasistatic limit the density profile relaxes instantaneously leaving only
\begin{equation}
  \rho_{hom}(t) = \lim_{D\to\infty}{\rho(r,t)}
\end{equation}
which inserted back into \eqref{eq:mass-equation} gives the droplet mass in this limit as
\begin{equation}
  m_{hom}(t) = 4\pi \rho_{hom}(t) \int_0^{R(t)} r^2 \, dr = \frac{4\pi R(t)^3}{3} \rho_{hom}(t)
\end{equation}
We can rearrange this relation for $R(t)$
\begin{equation}\label{eq:homogeneous-droplet-radius}
  R(t) = \left( \frac{3 \, m_{hom}(t)}{4\pi \rho_{hom}(t)} \right)^\frac{1}{3}
\end{equation}
where the corrections arise due to deviations from the homogeneous density profile $\Delta \rho = \rho(r) - \rho_{hom}$.
The solute mass fraction \eqref{eq:mass-fraction} reduces to
\begin{equation}\label{eq:homogeneous-solute-mass-fraction}
  \phi_{s,hom}(t) = \frac{m_s}{m_{hom}(t)}.
\end{equation}
In what follows we will drop the subscript $hom$ in all expressions

The trick to this method is realising that at short times the Kulmala coefficients remain approximately constant, leading to the following ODE and solution:
\begin{equation*}
  \frac{dR}{dt} = \frac{c}{R(t)} \quad \implies R(t) = \sqrt{R(0)^2 + 2ct}.
\end{equation*}

Differentiating \eqref{eq:homogeneous-droplet-radius} gives
\begin{equation}
  \frac{1}{3} \frac{d}{dt} \left( R^3 \right)
  = R^2 \frac{dR}{dt}
  = \frac{1}{4\pi} \frac{d}{dt} \left( \frac{m}{\rho} \right)
  = \frac{1}{4\pi} \left(
  \frac{1}{\rho} \frac{d m}{dt}
  - \frac{m}{\rho^2} \frac{d \rho}{dt}
  \right).
\end{equation}

We use \eqref{eq:homogeneous-solute-mass-fraction} to differentiate the density as
\begin{equation*}
  \frac{d\rho}{dt}
  = \rho'(\phi_s) \frac{d\phi_s}{dt}
  = -\rho'(\phi_s) \frac{m_s}{m^2} \frac{dm}{dt}
  = -\rho'(\phi_s) \frac{\phi_s}{m} \frac{dm}{dt}.
\end{equation*}
\begin{equation}
  \implies \quad \frac{dR}{dt}
  = \frac{1}{4\pi \rho} \left(
  1 + \phi_s \frac{\rho'(\phi_s)}{\rho}
  \right)
  \underbrace{\frac{1}{R^2} \frac{dm}{dt}}_{\propto 1/R}
\end{equation}

Giving the time integral from the ODE solution:
\begin{equation}
  \begin{aligned}
    R(t + \Delta t)
    &= \sqrt{R(t)^2 + \frac{1}{2\pi \rho R(t)}
      \left(
      1 + \phi_s \frac{\rho'(\phi_s)}{\rho}
      \right)
      \frac{dm}{dt} \Delta t} \\
    &= R(t) \left(
    1 + \frac{1}{4\pi \rho R(t)^3}
    \left(
    1 + \phi_s \frac{\rho'(\phi_s)}{\rho}
    \right)
    \frac{dm}{dt} \Delta t
    \right)
    + \mathcal{O}(\Delta t^2) \\
    &= R(t) \left(
    1 +
    \left(
    1 + \phi_s \frac{\rho'(\phi_s)}{\rho}
    \right)
    \frac{\dot{m}}{3m} \Delta t
    \right)
    + \mathcal{O}(\Delta t^2).
  \end{aligned}
\end{equation}

We model the diffusion constant by calibrating dynamic viscosity measurements
\marginfootnote{Need a Reid group citation} against molecular dynamics measures of the diffusion constant \marginfootnote{Lyubartsev and Laaksonen citation here}[2cm].
We assume the Stokes-Einstein form
\begin{equation}
  D(c;T) = D_0 \frac{\eta_0}{\eta(c)} \frac{T}{T_0}
\end{equation}
where $\eta_0$ is the dynamic viscosity of water in ambient conditions, $T_0 = \SI{293}{\kelvin}$ and $D_0 = \SI{1266}{\micro\metre^2\per\second}$ is a constant determined by fitting the two datasets.

\subsection{Treating evaporation}
\subsection{Evolution of the concentration profile}

\section{Nucleation model}
\subsection{Classical nucleation theory}

Numerical values taken from capilliary paper.

Nucleation rate per unit volume:
\begin{equation}
  \Gamma = \kappa \exp{\left(-\frac{\Delta G^{*}}{k_B T}\right)}
\end{equation}
where $\kappa$ is a kinetic prefactor.
CNT equation:
\begin{equation}
  \Delta G(R) = \frac{4}{3} \pi R^3 \rho_s \Delta \mu + 4\pi R^2 \gamma
\end{equation}
Defining $\Delta G^{*} = \max{(\Delta G)}$ and $R^{*} = \argmax{(\Delta G)}$ we have
\begin{align}
  \Delta G^{*} &= \frac{4}{3} \pi (R^{*})^2 \gamma \\
  (R^{*}) &= -\frac{2\gamma}{\rho_s \Delta\mu}
\end{align}
NB: $\Delta \mu < 0$.

Chemical potential expressed in terms of mean ionic activity:
\begin{equation}
  \begin{aligned}
  \exp{-\frac{\Delta \mu}{k_B T}} &=
  \nu \ln{\left( \frac{a_\pm}{a_{0\pm}} \right)} \\
  &=
  \nu \ln{\left( \frac{\gamma_\pm}{\gamma_{0\pm}} \frac{m}{m_0} \right)}
  \end{aligned}
\end{equation}
$nu$ is sum of ions (2 for NaCl), $a_\pm$ is mean ionic activity, $\gamma_\pm$ is mean ionic activity coefficient, $m$ and $m_0$ are molarities at equilibrium and crystal stabilisation respectively, no idea what $a_{0\pm}$ and $\gamma_{0\pm}$ are.

Kinetic prefactor:
\begin{align}
  \kappa &= \rho_l j Z \\
  j &\sim \rho_l D_l R^* \\
  Z &\sim (n^*)^{-\tfrac{2}{3}}
\end{align}
$\rho_l$ is liquid phase density, $j$ is rate of aggregation, $Z$ is Zeldovich factor, $n^*$ is excess number of molecules in critical nucleus (check references!).

\subsection{Density dependence of nucleation rate}
\subsection{Decay rates with droplet evolution}

Nucleation rate/unit volume is $\Gamma(Y_S)$.
Total nucleation rate (nucleation events per second across whole droplet):
\begin{equation}
  W = \int \Gamma dV
  = 4\pi \int_0^{R(t)} \Gamma(Y_S(r); t) \, dr
\end{equation}

In the volume element $r \to r + \Delta r$ Between $t \to t+\Delta t$ we have a Poisson process.
Mean events:
\begin{equation}
  \begin{aligned}
    \mu(r) &= \frac{4\pi}{3} ( (r+\Delta r)^3 - r^3)
    \int_t^{t+\Delta t} \Gamma(r,t') \, dt' \\
    &\simeq
    4\pi r^2 \Delta r \Gamma(r, t) \Delta t
  \end{aligned}
\end{equation}
Poisson process so we have probability of exactly $n$ events as
\begin{equation}
  P(n) = \frac{e^{-\mu} \mu^n}{n!}
\end{equation}
Giving probability of no event (`survival') as $P(0) = e^{-\mu}$.
Survival across all $r$ in this time period:
\begin{equation}
  \begin{aligned}
  \textrm{Probability}\left( \textrm{no event in } t \to t + \Delta t \right)
  &= \prod_i P(0;i) \\
  &= \prod_i \exp{\left(-4\pi (i\Delta r)^2 \Gamma(i\Delta r) \Delta t \Delta r\right)} \\
  &= \exp{\left(
    -4\pi \int_t^{t+\Delta t} dt' \int_0^{R(t')} dr r^2 \Gamma(r, t')
    \right)}
  \end{aligned}
\end{equation}

We have $\Gamma(r,t') = \Gamma(Y_S(r,t'), T(r,t'))$ so need a phenomenological model (CNT) for this.
Initially ignore temperature so we have $\Gamma(Y_S(r,t'))$.
We know: nucleation impossible below saturation limit.

\section{Appendix: finite difference methods for numerical integration}
Second-order scheme (Crank-Nicolson)

\end{document}
