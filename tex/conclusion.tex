%TC: macro \marginfootnote [other]
%TC: envir SCfigure [] other
%TC: macrocount beginSCfigure [figure]
\documentclass[11pt,twoside]{report}
\usepackage{preamble}
\setcounter{chapter}{7}
\graphicspath{{../img/}}
\def\includebibliography{}

\externaldocument{background}
\externaldocument{supercooled-liquids}
\externaldocument{morphometric-framework}
\externaldocument{morphometric-applications}
\externaldocument{resummation}
\externaldocument{aerosols}

\begin{document}
\chapter{Conclusion}
\epigraph{We demand rigidly defined areas of doubt and uncertainty!}{Douglas Adams, \emph{The Hitchhiker's Guide to the Galaxy}, (1979).}

At high densities, the dynamical properties of supercooled liquids show marked complexity distinguishing them from their ordinary counterparts at lower densities.
We have attempted to develop methods to better understand the nature of dynamical arrest and the glass transition, and nucleation of crystals within the metastable liquid.
Most of this work has focused on the first question, so we will emphasise that topic here and return to nucleation towards the end.

%with hard spheres as the model system, and to address nucleation of the crystals in a real system.
%Specifically, supercooled liquids and the glass transition, and nucleation of the crystal.

Liquids approaching their glass transition display a dramatic slowdown in their relaxation behaviour, while showing no obvious structural change at the level of pair correlations.
In chapter \ref{chapter:glass} we summarised the various scenarios posited to explain this phenomenon, highlighting the potential role of amorphous order in mean-field theories and structural order in geometric theories.
These thermodynamic scenarios in particular posit solely static mechanisms for dynamic arrest, which we argued could be detected through the many-body correlation functions.
Developing from this observation, in chapter \ref{chapter:morphometric-framework} we showed that the correlation functions could be expressed in terms of a potential of mean force, itself dependent on the interaction potential and a chemical potential term.
The key approximation we employed was the \emph{morphometric approach}, where the chemical potential is expressed as an expansion in size measures: the volume, surface area and integrated curvature measures.
Using the many-body correlation functions we attempted to explore the energy landscape of local structures in chapter \ref{chapter:morphometric-applications}, to look for features which could be connected with dynamical arrest.

%Disappointingly we have few results.

We have presented three justifications of the morphometric approach for hard particle systems.
First, we derived it in the usual way as a limit of fundamental measure theory (section \ref{sec:fmt}).
Second, we argued for the morphometric \emph{ansatz} as the natural generalisation of scaled particle theory (chapter \ref{chapter:morphometric-framework}); furthermore, we used integral geometry to argue that this \emph{ansatz} generalises the form of an extensive quantity (section \ref{sec:generalised-intrinsic-volumes-d}).
And last, we derived the morphometric form for the chemical potential by resumming a component of the virial series (chapter \ref{chapter:resummation}).

The latter two justifications are in principle new arguments, though we suspect neither will be of any surprise to the liquid state community; in some sense, these routes are all equivalent because they fundamentally reduce to integral geometric arguments.
%, and properties of the intrinsic volumes.
The primary advancements of this work have thus been technological, rather than fundamental, in nature.
In particular, we introduced the trick of imposing self-consistency of the virial theorem to derive a new set of morphometric coefficients in hard spheres (chapter \ref{chapter:morphometric-framework}).
These coefficients yield a theory for chemical potentials capable of producing highly accurate correlation functions, even at high densities.
Although we have made some modest contributions to treating local structure in chapter \ref{chapter:morphometric-applications} (described below), fundamentally the accuracy of all subsequent results depended on application of this trick.

In chapter \ref{chapter:morphometric-applications} we introduced methods to extend the formalism of energy landscapes, normally applied to soft potentials, to local structures of hard spheres; many adaptations were required to handle the singularity of the hard sphere interaction potential.
%Concerning morphometric approach: most of the developments have been in methods.
We explored a method of predicting the concentrations of structural motifs within the liquid, and developed a route to do similar calculations along saddles%
\marginfootnote{These calculations used techniques from Bayesian inference which were a late addition to this thesis, and are subsequently the least well-explained.
  Their inclusion provides a route to extending calculations to dynamical phenomena, so I felt it important to include them for a more complete perspective of available options.}
for connecting with dynamics.
Notably, we found a bimodality in the distribution of energy states corresponding to distinct structural symmetries, of potential importance to structural viewpoints of dynamical arrest.
This work lays the groundwork for a quantitative assessment of the landscape properties of local regions within the liquid, which could explore the validity of random first-order transition and related theories.

There are two limits to the accuracy of the morphometric approach: the thermodynamic coefficients entering the theory, and the limitations \emph{ansatz} itself.
We found that improving the coefficients was enough to obtain accurate results in chapters \ref{chapter:morphometric-framework} and \ref{chapter:morphometric-applications}, though the theory was not exact and so there is room for improvement particularly at high densities.
We attempted to provide some insight into the theory behind the morphometric approach in chapter \ref{chapter:resummation}, with a view to potentially supplementing the \emph{ansatz} with additional terms.
There, we found a contribution to the chemical potential which is rigorously of morphometric form, the thermodynamic coefficients of which capture most of the bulk free energy in hard spheres up to moderate densities; this observation potentially explains why the approach works well in the first place.
Furthermore, the resulting theory applies to arbitrary mixtures of hard convex particles without modification of the morphometric \emph{ansatz}, suggesting that extensions to other hard particle systems are possible.

%% Generalisations:
%% Hard particle extension is done.
%% We tested the theory up to mode-coupling, finding it accurate.
%% Polydisperse ready to go: direct comparison with the high density swap data could be made.
%% Could be a challenge to do the virial route coefficients, 
%% By introducing polydispersity into the theory we should obtain better quantitative agreement with experiments and simulations.

In Refs.\ \cite{DamascenoS2012,DamascenoAN2012} Glotzer and coworkers showed that hard polyhedra have all the richness in phase behaviour of the periodic table.
For example, the propensity for glassformation can be increased by inducing competition between polyhedra of different symmetries, which form competing domains of incompatible crystal structures \cite{TeichNC2019}.
Subsequent developments have introduced methods for tailoring the assembly into target crystal structures \cite{YoungACIE2013,SchultzAN2015,VanAndersAN2015}, many of which have been observed in simulation and experiment \cite{MisztaNM2011,HenzieNM2012,QiJCP2013}.
Polyhedral particles are intended as representatives of anisotropic particles (e.g.\ nanoparticles and colloids), in the same way that hard spheres are the starting point in simple liquids.
In current theories \cite{YoungACIE2013} effective entropic forces are imagined between parallel faces of adjacent polyhedra, which becomes exact at asymptotically high pressures.
The morphometric approach thus has potential to significantly improve descriptions of these interactions, and become a quantitatively predictive theory for nanoparticle and colloidal self-assembly.
%This would probably be the most interesting extension of the morphometric calculations.

For the \emph{isotropic} phase the current morphometric approach should readily extend to arbitrary mixtures of convex polyhedra; the form of the Carnahan-Starling equation for mixtures \eqref{eq:cs-mixtures} should even give a reasonable description of the liquid pressure.
However, in practice the computational geometry required to actually extend morphometric calculations to non-spherical particles would present a considerable challenge.
For \emph{anisotropic} phases (e.g.\ liquid crystal phases which form for highly elongated polyhedra) further theoretical developments would be required.
Fundamental measure theory has been extended to anisotropic phases \cite{Hansen-GoosPRL2009,Hansen-GoosJPCM2010,WittmannEL2015,WittmannPRE2015,WittmannJPCM2016}, so it is likely that a similar programme could be achieved for the simpler morphometric approach.

Continuing on the theme of self-assembly, a fundamental development of interest would be to the kinetics of protein folding.
The entropy of the surrounding water is argued to be a major thermodynamic contribution for aqueous proteins \cite{HaranoCPL2004,HaranoBJ2005,KinoshitaCES2006}.
The morphometric approach would thus be desirable to avoid explicit solvent modelling, and it has been used with some success \cite{HaranoCPL2006,RothPRL2006,KodamaJCP2011}.
Most of the literature on the morphometric approach concerns hard particles so, while it seems to treat depletion/exclusion interactions accurately, it is less clear how well it would perform for softer interaction potentials with attractions which better represent real systems.
Notably, the presence of attractions can induce non-analytic contributions through surface phase transitions \cite{EvansELE2003,EvansJCP2004}, which cannot be captured by the morphometric \emph{ansatz}.
Moreover, with a soft potential it is not clear \emph{a priori} how to define the surface geometry.
Despite these concerns, the potential computational benefit of the morphometric approach makes this area worth exploring.

Finally, we studied the nucleation of salt crystals inside atmospheric aerosols in chapter \ref{chapter:aerosols}.
The chemistry involved was too complex to treat the nucleation kinetics with a hard sphere model, so we used a continuum diffusion model to understand experimental data.
We found that classical nucleation theory had mixed success for the systems studied, suggesting more complex nucleation pathways than a simple one-dimensional model.

Were the morphometric approach to be extended to more realistic potentials, we could imagine bridging the gap between the microscopic models of the hard sphere chapters and the continuum models of the final chapter.
Crystal nucleation occurs by spontaneous formation of a crystal domain inside the bulk liquid, so the free energy calculations of chapter \ref{chapter:morphometric-applications} could be used for crystal geometries to assess the thermodynamic driving forces of nucleation.
The morphometric approach would offer a route to access nucleation pathways of much greater complexity than the simple one-dimensional projection normally considered in classical nucleation theory.
This would be another logical avenue of the morphometric approach, of interest even for hard spheres.

\ifdefined\includebibliography
  \newgeometry{margin=1in}
  \printbibliography
\fi

\end{document}
