\section{Statistical mechanics}

I will assume the reader has a background in physics, and is familiar enough with statistical ensembles that I do not need to explain the origin of the following equations in detail.
Briefly, these emerge by considering typical fluctuations of thermodynamic quantities for a subsystem within a macroscopic system called the \emph{ensemble}; the properties of this larger system define average quanties of the subsystem \cite{standard-text}.
Alternatively, from a Bayesian perspective the probability distributions given below emerge from maximising the entropy%
\marginfootnote{The entropy represents a thermodynamic quantity in the former picture, whereas it represents our own \emph{uncertainty} about the system in the latter.
  While these two approaches are formally equivalent, the first interpretation is more common in the physics literature.
  That being said, the derivations of the ensembles within an information theoretic framework are remarkably simple and will probably be more accessible to the non-physicist with a background in probability theory.}[-3cm]
subject to the constraint of average energy and (optionally) the average particle number \cite{Jaynes1957a,Jaynes1957b}.

A $d$-dimensional system of $N$ particles consists of $\vec{r}^N = \{\vec{r}_1, \cdots, \vec{r}_N\} \in \mathbb{R}^{dN}$ coordinates and $\vec{p}^N = \{\vec{p}_1, \cdots, \vec{p}_N\} \in \mathbb{R}^{dN}$ momenta.
The classical Hamiltonian can be decomposed into kinetic and potential terms as in
\begin{equation}
  \mathcal{H}(\vec{r}^N, \vec{p}^N) = K_N(\vec{p}^N) + U_N(\vec{r}^N)
\end{equation}
in the absence of an external field.
Further, we constrain the coordinates inside the region $\mathbb{V}^d$ of volume%
\marginfootnote{Physicists would typically use the same symbol $V$ for these two concepts.
  In keeping with the formal language of the previous chapter we would write $\vec{r}_i \in \mathbb{V}^d$ with $V = \mu_d(\mathbb{V}^d)$.}
$V$.
The \emph{canonical} ensemble describes an equilibrium system at constant temperature $T$ with probability measure
\begin{equation}
  f^{(N)}(\vec{r}^N, \vec{p}^N) \propto e^{-\beta{\mathcal{H}(\vec{r}^N, \vec{p}^N)}}
\end{equation}
where $\beta = (k_B T)^{-1}$ with Boltzmann constant $k_B$.
The proportionality constant ensures the probability distribution is properly normalised, leading to the canonical partition function
\begin{equation}
  Q_N =
  \int_{\mathbb{R}^{dN}} \int_{\mathbb{V}^{dN}}
  e^{-\beta{\mathcal{H}(\vec{r}^N, \vec{p}^N)}}
  \, d\vec{r}^N d\vec{p}^N.
\end{equation}
Or the classical kinetic energy can be written
\[ K_N(\vec{p}^N) = \sum_{i=1}^N \frac{|\vec{p}|^2}{2m_i} \]
then the momenta can be integrated out leaving only the configurational integral
\begin{equation}
  Q_N = \frac{\Lambda^{dN} Z_N}{N!}
\end{equation}
with
\begin{equation*}
  Z_N =
  \int_{\mathbb{V}^{dN}}
  \exp{-\beta U_N}
  \, d\vec{r}^N.
\end{equation*}
%% and confiprobability measure
%% \begin{equation}
%%   g^{(N)}(\vec{r}^N) = \frac{e^{-\beta U_N(\vec{r}^N)}}{Z_N}
%% \end{equation}

The grand canonical ensemble is taken.

The phase space measure is now
The partition function is given as
\begin{equation}
  \Xi
  = \sum_{N=0}^\infty \frac{z^N Z_N}{N!} \, d\vec{r}^N
  = \sum_{N=0}^\infty \frac{z^N}{N!} \int e^{-\beta U_N} \, d\vec{r}^N,
\end{equation}
where the activity is written in terms of the thermal de Broglie wavelength $\Lambda$ as $z = \exp{\beta\mu} / \Lambda^d$.
Accordingly, average quantities are found via
\begin{equation}
  \left< \cdots \right> =
  \frac{1}{\Xi} \sum_{N=0}^\infty \frac{z^N}{N!} \int \left(\cdots\right) e^{-\beta U_N} \, d\vec{r}^N,
\end{equation}

We will work almost exlusively in the \emph{grand canonical ensemble} as it is the most convenient for liquid state descriptions%
\marginfootnote{Notably the free energy is extensive without the need to invoke Stirling's approximation for $N!$, making the thermodynamics properly self-consistent even at finite system volumes.}.

\begin{equation}
  \mu_{id} = k_B T \ln{\Lambda^d \rho}
\end{equation}
\begin{equation}
  \frac{F_{id}}{N} = k_B T (\ln{\Lambda^d \rho} - 1)
\end{equation}
