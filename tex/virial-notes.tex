\section{Bulk virial approach}

\subsection{Outline}

As far as I know this is an original calculation, however it should be emphasised that this is merely an application of ideas from FMT to the (simpler) uniform system.
\todo{Emphasise that the resulting equations of state should be bad for anisotropic particles: it won't find any liquid crystalline transitions, and we expect multiple intersections to be a general rule for highly anistropic particles.}
My aim here is to introduce the ideas of FMT gradually in order to emphasise the role played by integral geometry.

Note that in mean field $d \gg 1$ it is argued that the dominant contributions come from the ring diagrams, i.e.\
\begin{equation*}
  \lim_{d \to \infty}
  \frac{\beta F_{ex}}{N} =
  \rho^2 \mayerdiagram{2} +
  \rho^3 \mayerdiagram{3} +
  \rho^4 \mayerdiagram[|.||.|]{4} +
  \rho^5 \, \mayerdiagram[|..||..|.|]{5} +
  \mathcal{O}(\rho^6)
\end{equation*}
whereas the contributing diagrams in for $d=1$ are the fully connected Ree-Hoover diagrams, i.e.\
\begin{equation*}
  \begin{aligned}
    \lim_{d \to 1}
    \frac{\beta F_{ex}}{N} &=
    \rho^2 \mayerdiagram{2} +
    \rho^3 \mayerdiagram{3} +
    \rho^4 \mayerdiagram{4} +
    \rho^5 \, \mayerdiagram{5} +
    \mathcal{O}(\rho^6) \\
    &=
    \rho^2 \fmtdiagram{2} +
    \rho^3 \fmtdiagram{3} +
    \rho^4 \fmtdiagram{4} +
    \rho^5 \fmtdiagram{5} +
    \mathcal{O}(\rho^6)
  \end{aligned}
\end{equation*}

\subsection{Exact result in one dimensions: hard rods}

Spheres in one dimensions are just line segments%
\marginfootnote{In fact, line segments are the only possible convex shape in one dimensions so hard rods are the one dimensional starting point for arbitrary (convex) shapes in higher dimensions.},
or rods, so one dimensional hard spheres are conventionally called hard rods.
The hard rod system is sufficiently simple that all of its equilibrium properties can be determined exactly, so it provides a good starting point for theories in higher dimensions.
Here we show how integral geometry can be used to determine the exact free energy (and thus the equation of state) for the hard rod system; this suggests approximations in higher dimensions to be discussed in the next section.

\begin{tcolorbox}[title=Multiple intersections of hard rods]
  Where every pairwise combination of $n$ rods overlap, there must be a region where all of them intersect, i.e.\
  \begin{equation}
    \prod_{i \le j} \mu_0(B_i \cap B_j) =
    \mu_0(\cap_{i=1}^n B_i)
  \end{equation}
  \emph{Proof:} Consider three rods whose centers are placed at $x_1, x_2, x_3$.
  Each rod overlaps so we have
  \begin{subequations}
    \begin{align}
      |x_2 - x_1| &\le \sigma \\
      |x_3 - x_1| &\le \sigma \\
      |x_3 - x_2| &\le \sigma
    \end{align}
  \end{subequations}
  The region of overlap between the first two particles is
  \begin{equation*}
    B_1 \cap B_2 =
    [\max{(x_1,x_2)} - \sigma, \min{(x_1,x_2)} + \sigma].
  \end{equation*}
  which implies
  \begin{equation}
    \max{(x_1,x_2)} - \sigma \le x_3 \le \min{(x_1,x_2)} + \sigma
  \end{equation}
  or $x_3 \in (B_1 \cap B_2)$.
\end{tcolorbox}

\begin{SCfigure}[H]
  \missingfigure[figwidth=\linewidth]{}
  \caption{If each pair of three hard rods overlap then this implies a region where all three mutually intersect.}
\end{SCfigure}


Therefore the diagram for the third virial coefficient becomes
\begin{equation}\label{eq:third-virial-diagram-replacement-1d}
  \mayerdiagram{3} = \fmtdiagram{3}
\end{equation}
where the latter \emph{intersection} diagram denotes an integral of the form
\begin{equation}
  \fmtdiagram{3} =
  \int_{\mathbb{E}_d^3}
  \mu_0 (g_1 B^d \cap g_2 B^d \cap g_3 B^d)
  \, dg_1 dg_2 dg_3.
\end{equation}
The result of \eqref{eq:third-virial-diagram-replacement-1d} generalises straightforwardly to the other fully connected diagrams.
For example, the fourth and fifth terms in the virial series are replaced by
\begin{align*}
  \mayerdiagram{4} &= \fmtdiagram{4} =
  \int_{\mathbb{E}^4}
  \mu_0 (\cap_{i=1}^4 g_i B^1)
  \, dg^4
  \\
  \mayerdiagram{5} &= \fmtdiagram{5} =
  \int_{\mathbb{E}^5}
  \mu_0 (\cap_{i=1}^5 g_i B^1)
  \, dg^5,
\end{align*}
and in general the $n$th fully connected diagram can be replaced by the integral
\begin{equation}
  \int_{\mathbb{E}_d^n}
  \mu_0 (\cap_{i=1}^n g_i B^1)
  \, dg^n \equiv \Gamma_n,
\end{equation}
without approximation.

We make the replacement
\begin{equation}\label{eq:third-virial-diagram-replacement-general}
  \mayerdiagram{3} \approx \fmtdiagram{3}
\end{equation}
which is now approximate as can be seen from the counter example in Figure \ref{fig:lost-cases}.

\begin{SCfigure}[H]
  \missingfigure[figwidth=\linewidth]{}
  \caption{Lost cases of FMT.}
  \label{fig:lost-cases}
\end{SCfigure}

Ree-Hoover diagrams.
From the relation:
\begin{equation*}
  1 = e_{ij} - f_{ij}
\end{equation*}
we have the graphical representation of this rule as
\begin{equation*}
  \mayerdiagram[.]{2} =
  \mayerdiagram[:]{2} -
  \mayerdiagram[|]{2}
\end{equation*}
so
\begin{equation*}
  \begin{aligned}
    \mayerdiagram[|.||.|]{4} &=
    \mayerdiagram[|.||:|]{4} -
    \mayerdiagram[|.||||]{4} \\
    &=
    \mayerdiagram[|:||:|]{4} +
    \mayerdiagram[||||||]{4} -
    \mayerdiagram[|:||||]{4} -
    \mayerdiagram[||||:|]{4} \\
    &=
    \mayerdiagram[|:||:|]{4} +
    \mayerdiagram[||||||]{4} -
    2 \mayerdiagram[|:||||]{4}
  \end{aligned}
\end{equation*}
after using permutation invariance in the final step.
We will extend this by considering only the fully connected Ree-Hoover diagrams, as in
\begin{equation}
  \frac{\beta F^{ex}}{N} \approx
  \rho^2 \mayerdiagram{2} +
  \rho^3 \mayerdiagram{3} +
  \rho^4 \mayerdiagram{4} +
  \rho^5 \mayerdiagram{5}
  + \mathcal{O}(\rho^6)
\end{equation}
and then approximate these by considering only the subset of $n$-particle intersections as in
\begin{equation}
  \frac{\beta F^{ex}}{N} \approx
  \rho^2 \fmtdiagram{2} +
  \rho^3 \fmtdiagram{3} +
  \rho^4 \fmtdiagram{4} +
  \rho^5 \fmtdiagram{5}
  + \mathcal{O}(\rho^6)
\end{equation}
