%TC: macro \marginfootnote [other]
%TC: envir SCfigure [] other
%TC: macrocount beginSCfigure [figure]
\documentclass[11pt,twoside]{report}
\usepackage{preamble}
\setcounter{chapter}{0}
\graphicspath{{../img/}}
\def\includebibliography{}
\renewcommand{\chaptername}{Appendix}
\renewcommand{\thechapter}{\Alph{chapter}}

\externaldocument{introduction}
\externaldocument{background}
\externaldocument{morphometric-framework}
\externaldocument{morphometric-applications}
\externaldocument{resummation}
\externaldocument{aerosols}

\begin{document}
\chapter{Existence of singularities in the chemical potential}
\label{appendix:spt-singularities}

This appendix is intended to supplement chapters \ref{chapter:morphometric-framework} and \ref{chapter:resummation} by emphasising the fundamental limitations of the theories we advocate there.
In those chapters we advanced the \emph{morphometric approach}, an approximate theory for solvation in hard particle systems.
This approximation scheme expresses a chemical potential as an expansion in terms of geometric properties.
This expansion is found in any one of equations \eqref{eq:extensive-integral-geometry}, \eqref{eq:extensive-integral-geometry-d}, \eqref{eq:fmt-morphometric-2}, \eqref{eq:morph-ansatz}, or \eqref{eq:morphometric-approach-from-virial}.
This approach is numerically very accurate, within its regime of validity, however there are fundamental limitations which prevent it from being readily extended.
Here we revisit old arguments made by Reiss and coworkers \cite{ReissJCP1959,ReissJCP1960} which demonstrate that the cost of inserting a solute is not generally analytic in geometric measures.
This argument demonstrates that \emph{any} regular expansion of $\Delta \Omega$ in terms of geometrical measures like curvature must necessarily be approximate.

We consider a single-component hard sphere liquid with particles of diameter $\sigma$, and we imagine inserting a hard spherical solute of radius $R$.
A sphere of radius $R + \sigma/2$ around the solute is then excluded to the centers of solvent particles.
We write this excluded volume as $V_\mathrm{ex}$, which is clearly the minimum size of cavity required to contain the solute.
The insertion cost is simply the probability that a randomly chosen position for insertion contains such a cavity, i.e.\
\begin{equation}\label{eq:insertion-from-p}
  \Delta \Omega(R)
  =
  -k_B T \ln p_0(R)
\end{equation}
where $p_0$ is the probability that the excluded region is empty.
This can be determined as \cite{ReissJCP1959}
\begin{equation}\label{eq:spt-zero-cavity-p}
  p_0(R) = 1 + \sum_{n=1}^\infty (-1)^n F^{(n)}(R)
\end{equation}
where $F^{(n)}(R)$ is the average number of $n$-tuples of solvent particles contained in the excluded region, as in
\begin{equation}\label{eq:spt-tuple-function}
  F^{(n)}(R)
  =
  \frac{\rho^n}{n!}
  \int_{V_\mathrm{ex}^n} g^{(n)}(\vec{r}^n) \, d\vec{r}^n.
\end{equation}
Clearly $F^{(n)}(R) = 0$ for $R < R_n$, the minimum radius capable of containing $n$ hard spheres.
At any given state point $g^{(n)}$ will be bounded from above by some finite number, so we can write the inequality%
\marginfootnote{Typically, we would expect this to occur where the maximum number of particles are in contact, however that is not a necessary assumption for this argument.}
\begin{equation}\label{eq:spt-tuple-function-upper-bound}
  \begin{split}
    F^{(n)}(R) &\le
    \frac{\rho^n}{n!}
    \max_{\mathbb{R}^{dn}}{\left(g^{(n)}\right)}
    \int_{V_\mathrm{ex}} \, d\vec{r}^n \\
    &=
    \frac{\rho^n}{n!}
    \max_{\mathbb{R}^{dn}}{\left(g^{(n)}\right)}
    (V_\mathrm{ex})^n,
  \end{split}
\end{equation}
but because of the hard core interaction there will be heavy restrictions on allowable values of $n$ for any $R$.
Defining $R_n$ as the smallest $R$ such that $n$ particles can be accommodated, we expect
\begin{equation}\label{eq:F-scaling}
  F^{(n)}(R) =
  \begin{cases}
    0 & \quad R < R_n \\
    \mathcal{O}\left( \left(V_\mathrm{ex}\right)^n \right) & \quad R > R_n
  \end{cases}
\end{equation}
where the latter polynomial is motivated by the same argument as used in \eqref{eq:spt-tuple-function-upper-bound}.
Noting that $V_\mathrm{ex} \propto R^d$ this can be expressed alternatively as a polynomial in $R$.
\begin{equation*}
  F^{(n)}(R) =
  \begin{cases}
    0 & \quad R < R_n \\
    \mathcal{O}\left( R^{dn} \right) & \quad R > R_n
  \end{cases}
\end{equation*}
Applying this bound to \eqref{eq:spt-zero-cavity-p} gives bounds on the scaling of $p_0$ as
\begin{equation}\label{eq:spt-p-scaling}
  p_0(R) =
  \mathcal{O}\left( R^{dn} \right)
  \quad R_n < R < R_{n+1}.
\end{equation}
From \eqref{eq:spt-p-scaling} we expect to see singular behaviour at the points $\{R_n\}$.
To look at this in more detail we define
\begin{equation*}
  p_0^{(n)}(R) := 1 + \sum_{m=1}^n (-1)^m F^{(m)}(R)
\end{equation*}
such that $p_0 = p_0^{(n)}$ for $R \le R_{n+1}$.
Approaching the singular point $R_n$ we find the deviation from the solution for $R < R_n$ is thus
\begin{equation*}
  \begin{split}
    \Delta p_0(R) :=& \;
    p_0(R) - p_0^{(n-1)}(R)
    \\ =& \;
    (-1)^n F^{(n)}(R)
    \qquad R < R_{n+1}
  \end{split}
\end{equation*}
i.e.\ the singular behaviour is entirely contained in $F^{(n)}$.
Noting that polynomials of degree $n$ have vanishing $(n+1)$th derivatives, and $V_\mathrm{ex} \propto R^d$, we thus expect a discontinuity in the $dn$th derivative of $p_0$ about $R=R_n$; from \eqref{eq:insertion-from-p} we find $\Delta \Omega$ will similarly feature a discontinuity in its $dn$th derivative at $R=R_n$.
A summary of the first few singularities in three-dimensions is given in Table \ref{table:spt-singularities}.

Again, these singularities have long been known since the first papers on scaled particle theory \cite{ReissJCP1959,ReissJCP1960}.
They are worth reiterating because any geometric expansion in simple powers of $R$ will not capture these singularities; the morphometric approach is an example of such an expansion, albeit generalised beyond spherical solutes, so it must necessarily be approximate.

\begin{SCtable}
  \begin{minipage}[b]{\linewidth}
    \centering
    \begin{tabular}{ccc}
      \toprule
      $n$ & $R_n / \sigma$ & Discontinuity \\
      \midrule
      %% 2 & $\frac{\sigma}{2}$ & $\Delta \Omega'''(R)$
      %% \\
      %% 3 & $\frac{\sigma}{\sqrt{3}}$ &
      %% $\dfrac{\partial^6 \Delta \Omega}{\partial R^6}$
      %% \vspace{0.25em} \\
      %% 4 & ${\sqrt{\frac{3}{8}}} \sigma$ &
      %% $\dfrac{\partial^9 \Delta \Omega}{\partial R^9}$
      2 & 0 & $\Delta \Omega'''(R)$
      \\
      3 & $\dfrac{1}{\sqrt{3}} - \dfrac{1}{2}$ &
      $\dfrac{\partial^6 \Delta \Omega}{\partial R^6}$
      \vspace{0.25em} \\
      4 & $\sqrt{\dfrac{3}{8}} - \dfrac{1}{2}$ &
      $\dfrac{\partial^9 \Delta \Omega}{\partial R^9}$ \\
      \bottomrule
    \end{tabular}
  \end{minipage}
  \caption[First few singularities in the insertion cost]{
    First few singularities in the cost of inserting a sphere of radius $R$ into a one-component hard sphere liquid of diameter $\sigma$ in $d=3$.
    $R = R_n$ is the minimum radius required for a sphere to contain $n$ spheres, and the corresponding singularity is determined from equations \eqref{eq:insertion-from-p} and \eqref{eq:spt-p-scaling}.}
  \label{table:spt-singularities}
\end{SCtable}

\ifdefined\includebibliography
  \printbibliography
\fi

\end{document}
