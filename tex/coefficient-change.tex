\subsection{Coefficient change for parallel surfaces}
\hl{Assume}
\begin{equation}\label{eq:dilated-volumes}
  V_k(A \minkplus \epsilon B_d)
  =
  \sum_{i=0}^d b_k^i V_i(A)
\end{equation}
\hl{Considering parallel bodies the intrinsic volumes}
\begin{equation*}
  V_k (\epsilon K)
  =
  \epsilon^k V_k.
\end{equation*}
\hl{The coefficients are independent of the geometry of} $A$, \hl{so we can consider a simple spherical geometry.
Therefore}
\begin{equation*}
  \begin{split}
  %V_k (B_d \minkplus \epsilon B_d)
  &=
  V_k \left[ (1+\epsilon) B_d \right]
  \\ &=
  (1+\epsilon)^k V_k(B_d)
  \\ &=
  \sum_{i=0}^k {k \choose i} \epsilon^{k-i} V_k(B_d)
  \\ &=
  \sum_{i=0}^k {d-i \choose d-k}
  \frac{\omega_{d-i}}{\omega_{d-k}}
  \epsilon^{k-i} V_i(B_d)
  \end{split}
\end{equation*}
using \eqref{eq:intrinsic-volume-ball} to reach the last line.
Comparison with \eqref{eq:dilated-volumes} allows us to read off the coefficients
\begin{equation}\label{eq:dilated-volume-coefficients}
  b_k^i =
  \begin{cases}
    { d-i \choose d-k }
    \frac{\omega_{d-i}}{\omega_{d-k}}
    \epsilon^{k-i}
    & \; i \le k \\
    0 & \; i > k
  \end{cases}
\end{equation}
giving the classical Steiner's formula for convex bodies in the case that $k=d$.
Finally we have
\begin{equation}
  %V_k(A \minkplus \epsilon B_d) =
  \sum_{i=0}^k { d-i \choose d-k }
  \frac{\omega_{d-i}}{\omega_{d-k}}
  \epsilon^{k-i} V_i(A)
\end{equation}
i.e.
\begin{equation}
  \begin{split}
    \sum_{k=0}^d a_k V_k(A)
    &=
    %\sum_{k=0}^d a_k' V_k(A \minkplus \epsilon B_d)
    \\ =
    \sum_{k=0}^d a_k' \sum_{i=0}^k b_k^i V_i(A)
    &=
    \sum_{k=0}^d V_k(A) \sum_{i=k}^d a_i' b_i^k
  \end{split}
\end{equation}
or
\begin{equation}
  a_k = \sum_{i=k}^d a_i' b_i^k,
\end{equation}
which can be inverted for $a_i'$.
In $d=2$ this yields
\begin{subequations}
  \begin{align}
    a_2' &= a_2 \\
    a_1' &= a_1 - 2 \epsilon a_2 \\
    a_0' &= a_0 - \pi \epsilon a_1 + \pi \epsilon^2 a_2,
  \end{align}
\end{subequations}
or for $d=3$
\begin{subequations}
  \begin{align}
    a_3' &= a_3 \\
    a_2' &= a_2 - 2 \epsilon a_3 \\
    a_1' &= a_1 - \pi \epsilon a_2 + \pi \epsilon^2 a_3 \\
    a_0' &=
    a_0 - 4 \epsilon a_1 +
    2 \pi \epsilon^2 a_2 - \frac{4}{3} \pi \epsilon^3 a_3.
  \end{align}
\end{subequations}
For example, transforming \eqref{eq:cs-spt-coefficients} to the parallel surface $\epsilon = \frac{\sigma}{2}$ gives
\begin{subequations}\label{eq:cs-spt-coefficients-parallel}
  \begin{align}
    \beta a_0'
    &=
    - \frac{\eta(4 - 11\eta + 13\eta^2 - 4\eta^3)}{3(1-\eta)^3}
    - 4 \ln{(1-\eta)},
    \\
    \beta a_1'
    &=
    \frac{1}{2\sigma} \left(
    \frac{\eta(4 - 10\eta + 20\eta^2 - 8\eta^3)}{(1-\eta)^3}
    + 4 \ln{(1-\eta)}
    \right),
    \\
    \beta a_2'
    &=
    - \frac{2}{\pi \sigma^2}
    \left(
    \frac{\eta(1 + 2\eta + 8\eta^2 - 5\eta^3)}{(1-\eta)^3}
    + \ln{(1-\eta)}
    \right),
    \\
    \frac{\beta p}{\rho}
    &=
    \frac{1 + \eta + \eta^2 - \eta^3}{(1-\eta)^3},
  \end{align}
\end{subequations}
which are identical to those in Ref.\ \cite{Hansen-GoosJPCM2006}, up to a different normalisation of the geometric parameters; we use the intrinsic volumes $\{V_0, V_1, V_2, V_3\}$ whereas in Ref.\ \cite{Hansen-GoosJPCM2006} $\{X, C, A, V\}$ are used.
The transformations between these two sets of geometric properties are given in Table~\ref{table:geometric-quantities}.
